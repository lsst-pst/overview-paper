\section{Summary and Conclusions}
\label{Sec:conclusions}
%edited by Tony Tyson 11/13/17

Until recently, most astronomical investigations have focused on small
samples of cosmic sources or individual objects. Over the past few decades,
however, advances in technology have made it possible to move beyond the
traditional observational paradigm and to undertake large-scale sky
surveys, such as SDSS, 2MASS, GALEX, Gaia, and others. This observational
progress, based on a synergy of advances in telescope construction, detectors,
and above all, information technology, has had a dramatic impact on nearly all
fields of astronomy, many areas of fundamental physics, and society in
general. The LSST builds on the experience of these surveys and addresses
the broad goals stated in several nationally endorsed reports by the U.S.\
National Academy of Sciences. The 2010 report ``New Worlds, New Horizons
in Astronomy and Astrophysics'' by the Committee for a Decadal Survey of Astronomy and
Astrophysics
%\footnote{\url{http://www.nap.edu/catalog.php?record_id=12951}}
ranked LSST as its top priority for large ground-based programs.
The LSST will be unique: the combination of large aperture and large field of view,
coupled with the needed computation power and database technology, will
enable simultaneously fast and wide and deep imaging of the sky, addressing in
one sky survey the broad scientific community's needs in both the
time domain and deep universe.

The realization of the LSST involves extraordinary engineering and
technological challenges: the fabrication of large, high-precision optics;
construction of a huge, highly-integrated array of sensitive, wide-band
imaging sensors; and the operation of a data management facility
handling tens of terabytes of data each day. The design, development
and construction
effort has been underway since 2006 and will continue through the
onset of full survey operations.  This work involves hundreds of
personnel at institutions all over the US, Chile, and the rest of the
world.
%, includes structural, thermal, and optical analyses
%of all key hardware subsystems, vendor interactions to determine
%manufacturability, explicit prototyping of high-risk elements, prototyping
%and development of data management systems, and extensive systems engineering
%studies. Over 200 technical personnel at a range of institutions are currently
%engaged in this program.

In December 2013, LSST passed the NSF Final Design Review for construction,
and in May 2014 the National Science Board approved the project.
The primary/tertiary mirror was cast in 2008, and the polished mirror
was completed in 2015.
In 2014 LSST transitioned from the design and development phase to
construction, and the Associated Universities for Research in
Astronomy (AURA) has formal responsibility for the LSST project since 2011.
At this writing,  the project is near the peak of the construction
effort, and is preparing for the transition to
commissioning and operations.

The construction cost of LSST is being borne by the US National Science
Foundation, the Department of Energy, generous contributions from several
private foundations and institutions, and the member institutions of the
LSST Corporation. The LSST budget includes a significant Education and
Public Outreach program (\S~\ref{Sec:impact}).
The U.S.\ Department of Energy is supporting the cost of constructing the
camera. LSST has high visibility in the high-energy physics community,
both at universities and government laboratories. The telescope will
see first light with a commissioning camera in late 2019, and the
project is scheduled to begin regular survey operations by 2022.
%(it will take about five years from the start of the federal construction phase
%to full system integration and the beginning of the commissioning period).

The LSST survey will open a movie-like window on objects that
change brightness, or move, on timescales ranging from 10 seconds to 10 years.
The survey will have a raw data rate of about 15 TB per night (about the same as one
complete Sloan Digital Sky Survey per night), and will collect about 60 PB
of data over its lifetime, resulting in an incredibly rich and extensive
public archive that will be a treasure trove for breakthroughs in many areas
of astronomy and physics. About 20 billion galaxies and a similar number of stars
will be detected -- for the first time in history, the number of cataloged
celestial objects will exceed the number of living people! About a thousand
observations of each position across half of the Celestial Sphere will
represent the greatest movie of all time.
% : assuming HDTV resolution (2.1 Mpix/frame) with 30 frames per second, and 270 color images per position
% (e.g., the $ugr$ and $izy$ color-composites, c.f. Table~\ref{tab:baseline}), it would take
% about eleven uninterrupted months to view this color movie (equivalent
% to about 4,000 feature films).


%The integrated cost of LSST will be {\$638M} for construction (NSF and DOE contributions),
%and {\$37M} per year for operations (in 2014 U.S.\ dollars).

%Beyond the improved image quality, LSST will be much more cost
%efficient than the current state-of-the-art massive optical survey, the
%SDSS. For example, for each \$100 spent, the SDSS obtained
%single-epoch $ugriz$ photometry for $\sim$200 sources and spectra for
%two sources. In contrast, for \$100 LSST will obtain $ugrizy$
%photometry for about 4,000 sources (to a limit 5 magnitudes deeper
%than SDSS) and a 800-point multi-color light curve for $\sim$400 sources. Compared
%to the SDSS, LSST will produce three orders of magnitude larger data volume with
%an increase in total cost of less than a factor of ten.

Alerts of transient, variable, and moving objects will be issued worldwide within
60 seconds of detection.
An extensive public outreach program will provide a new view of the sky to
curious minds of all ages worldwide.
We are working with prospective foreign partners to make all of the LSST science data
more broadly available worldwide.  As of 2017, 34 institutions from 23 countries
have signed Memoranda of Agreement to contribute significantly to
the LSST operating costs, in exchange for participation in the science collaborations
and data access.  The software which processes the pixels
and creates the LSST database is open source.
LSST will be a significant milestone in the globalization of the information revolution.
The vast relational database of 32 trillion observations of 40 billion objects
will be mined for the unexpected and used for precision experiments in astrophysics.
LSST will be in some sense an internet telescope:
the ultimate network peripheral device to explore the Universe, and
a shared resource for all humanity.
