
\begin{abstract}
  We describe here the most ambitious survey currently planned in the optical,
  the Large Synoptic Survey Telescope (LSST). The LSST design is driven by four
  main science themes: probing dark energy and dark matter, taking an inventory
  of the solar system, exploring the transient optical sky, and mapping the Milky Way.
  LSST will be a large, wide-field ground-based system designed to obtain repeated
  images covering the sky visible from Cerro Pach'{o}n in northern Chile. The telescope
  will have an 8.4 m (6.5 m effective) primary mirror, a 9.6 deg$^2$ field of view, a
  3.2-gigapixel camera, and six filters ($ugrizy$) covering the wavelength range
  320--1050 nm. The project is in the construction phase and will begin regular survey
  operations by 2022. About 90\% of the observing time will be devoted to a deep-wide-fast
  survey mode that will uniformly observe a 18,000 deg$^2$ region about 800 times
  (summed over all six bands) during the anticipated 10~yr of operations and will
  yield a co-added map to $r\sim27.5$. These data will result in databases including about
  32 trillion observations of 20 billion galaxies and a similar number of stars, and they will
  serve the majority of the primary science programs. The remaining 10\% of the observing
  time will be allocated to special projects such as a Very Deep and Very Fast time domain
  surveys, whose details are currently under discussion. We illustrate how the LSST science
  drivers led to these choices of system parameters, and we describe the expected data products
  and their characteristics.
\end{abstract}

\keywords{
  surveys ---
  methods: observational ---
  astrometry ---
  cosmology: observations ---
  Galaxy: general ---
  stars: general
}

% latex lsst; dvips -Ppdf -o lsst.ps lsst.dvi; ps2pdf14 lsst.ps lsst.pdf
