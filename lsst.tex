\documentclass{emulateapj}


\slugcomment{A living LSST document (arXiv:0805.2366); version 3.1 of Aug 29, 2014}


\newcommand\x         {\hbox{$\times$}}
\newcommand\othername {\hbox{$\dots$}}
\def\eq#1{\begin{equation} #1 \end{equation}}
\def\eqarray#1{\begin{eqnarray} #1 \end{eqnarray}}
\def\eqarraylet#1{\begin{mathletters}\begin{eqnarray} #1 
                  \end{eqnarray}\end{mathletters}}
\def\mic              {\hbox{$\mu{\rm m}$}}
\def\about            {\hbox{$\sim$}}
\def\Mo               {\hbox{$M_{\odot}$}}
\def\Lo               {\hbox{$L_{\odot}$}}
\def\comm#1           {{\tt (COMMENT: #1)}}
\def\kms   {\hbox{km s$^{-1}$}}

\usepackage{graphicx}
\usepackage{xspace}
\usepackage{hyperref}

\usepackage[usenames]{color} 
%\newcommand{\B}[1]{{\color{blue} #1}}
%\newcommand{\R}[1]{{\color{red} #1}}
\newcommand{\G}[1]{{\color{red} #1}}
\newcommand{\B}[1]{{#1}}
\newcommand{\R}[1]{{\color{red} #1}}

\begin{document}
\title{LSST: from Science Drivers to Reference Design and Anticipated Data Products} 

%% DO NOT EDIT THIS FILE. IT IS GENERATED FROM db2authors.py"
%% Regenerate using:
%%    python support/db2authors.py > authors.tex


\author[0000-0001-5250-2633]{\v{Z}eljko~Ivezi\'{c}}
\affiliation{University of Washington, Dept.\ of Astronomy, Box 351580, Seattle, WA 98195}

\author{Steven~M.~Kahn}
\affiliation{LSST Project Office, 950 N.\ Cherry Avenue, Tucson, AZ  85719}
\affiliation{Kavli Institute for Particle Astrophysics and Cosmology, SLAC National Accelerator Laboratory, Stanford University, Stanford, CA 94025}

\author[0000-0002-9242-8797]{J.~Anthony~Tyson}
\affiliation{Physics Department, University of California, One Shields Avenue, Davis, CA 95616}

\author{Bob~Abel}
\affiliation{Olympic College, 1600 Chester Ave., Bremerton, WA 98337-1699}

\author{Emily~Acosta}
\affiliation{LSST Project Office, 950 N.\ Cherry Avenue, Tucson, AZ  85719}

\author{Robyn~Allsman}
\affiliation{LSST Project Office, 950 N.\ Cherry Avenue, Tucson, AZ  85719}

\author{David~Alonso}
\affiliation{Department of Physics, University of Oxford, Denys Wilkinson Building, Keble Road, Oxford, OX1 3RH, UK}

\author{Yusra~AlSayyad}
\affiliation{Department of Astrophysical Sciences, Princeton University, Princeton, NJ 08544}

\author{Scott~F.~Anderson}
\affiliation{University of Washington, Dept.\ of Astronomy, Box 351580, Seattle, WA 98195}

\author{John~Andrew}
\affiliation{LSST Project Office, 950 N.\ Cherry Avenue, Tucson, AZ  85719}

\author{James~Roger~P.~Angel}
\affiliation{Steward Observatory, The University of Arizona, 933 N Cherry Ave., Tucson, AZ 85721}

\author{George~Z.~Angeli}
\affiliation{Giant Magellan Telescope Organization (GMTO), 465 N.\ Halstead Street, Suite 250, Pasadena, CA 91107}

\author{Reza~Ansari}
\affiliation{Laboratoire de l'Acc\'{e}l\'{e}rateur Lin\'{e}aire, CNRS/IN2P3, Universit\'{e} de Paris-Sud, B.P.\ 34, 91898 Orsay Cedex, France}

\author{Pierre~Antilogus}
\affiliation{Laboratoire de Physique Nucl\'{e}aire et des Hautes Energies, Universit\'{e} Pierre et Marie Curie, Universit\'{e} Paris Diderot, CNRS/IN2P3, 4 place Jussieu, 75005 Paris, France}

\author{Constanza~Araujo}
\affiliation{LSST Project Office, 950 N.\ Cherry Avenue, Tucson, AZ  85719}

\author{Robert~Armstrong}
\affiliation{Department of Astrophysical Sciences, Princeton University, Princeton, NJ 08544}

\author[0000-0002-6826-8340]{Kirk~T.~Arndt}
\affiliation{Department of Physics, University of Oxford, Denys Wilkinson Building, Keble Road, Oxford, OX1 3RH, UK}

\author{Pierre~Astier}
\affiliation{Laboratoire de Physique Nucl\'{e}aire et des Hautes Energies, Universit\'{e} Pierre et Marie Curie, Universit\'{e} Paris Diderot, CNRS/IN2P3, 4 place Jussieu, 75005 Paris, France}

\author{\'{E}ric~Aubourg}
\affiliation{AstroParticule et Cosmologie, Universit\'{e} Paris Diderot, CNRS/IN2P3, CEA/lrfu, Observatoire de Paris, Sorbonne Paris Cit\'{e}, 10, rue Alice Domon et L\'{e}onie Duquet, Paris Cedex 13, France}

\author{Nicole~Auza}
\affiliation{LSST Project Office, 950 N.\ Cherry Avenue, Tucson, AZ  85719}

\author[0000-0002-5722-7199]{Tim~S.~Axelrod}
\affiliation{Steward Observatory, The University of Arizona, 933 N Cherry Ave., Tucson, AZ 85721}

\author{Deborah~J.~Bard}
\affiliation{SLAC National Accelerator Laboratory,  2575 Sand Hill Rd, Menlo Park CA 94025}

\author{Jeff~D.~Barr}
\affiliation{LSST Project Office, 950 N.\ Cherry Avenue, Tucson, AZ  85719}

\author{Aurelian~Barrau}
\affiliation{Laboratoire de Physique Subatomique et de Cosmologie, Universit\'{e} Grenoble-Alpes, CNRS/IN2P3, 53 av.\ des Martyrs, 38026 Grenoble cedex, France}

\author{James~G.~Bartlett}
\affiliation{AstroParticule et Cosmologie, Universit\'{e} Paris Diderot, CNRS/IN2P3, CEA/lrfu, Observatoire de Paris, Sorbonne Paris Cit\'{e}, 10, rue Alice Domon et L\'{e}onie Duquet, Paris Cedex 13, France}

\author[0000-0001-9037-6981]{Amanda~E.~Bauer}
\affiliation{LSST Project Office, 950 N.\ Cherry Avenue, Tucson, AZ  85719}

\author{Brian~J.~Bauman}
\affiliation{Lawrence Livermore National Laboratory, 7000 East Avenue, Livermore, CA 94550}

\author{Sylvain~Baumont}
\affiliation{Sorbonne Universit\'{e}s, UPMC Univ Paris 06, UMR 7585, LPNHE, F-75005, Paris, France}
\affiliation{Laboratoire de Physique Nucl\'{e}aire et des Hautes Energies, Universit\'{e} Pierre et Marie Curie, Universit\'{e} Paris Diderot, CNRS/IN2P3, 4 place Jussieu, 75005 Paris, France}

\author[0000-0001-6661-3043]{Andrew~C.~Becker}
\affiliation{University of Washington, Dept.\ of Astronomy, Box 351580, Seattle, WA 98195}

\author{Jacek~Becla}
\affiliation{SLAC National Accelerator Laboratory,  2575 Sand Hill Rd, Menlo Park CA 94025}

\author{Cristina~Beldica}
\affiliation{NCSA, University of Illinois at Urbana-Champaign, 1205 W.\ Clark St., Urbana, IL 61801}

\author{Steve~Bellavia}
\affiliation{Brookhaven National Laboratory, Upton, NY 11973}

\author[0000-0003-1953-8727]{Federica~B.~Bianco}
\affiliation{Center for Urban Science \& Progress, New York University, Brooklyn, NY 11021}
\affiliation{Center for Cosmology \& Particle Physics, New York University, New York, 10012}

\author{Rahul~Biswas}
\affiliation{Oskar Klein Centre, Department of Physics, Stockholm University, SE 106 91 Stockholm, Sweden}

\author{Guillaume~Blanc}
\affiliation{Laboratoire de l'Acc\'{e}l\'{e}rateur Lin\'{e}aire, CNRS/IN2P3, Universit\'{e} de Paris-Sud, B.P.\ 34, 91898 Orsay Cedex, France}
\affiliation{Universit\'{e} Paris Diderot, Sorbonne Paris Cit\'{e}, F-75013 Paris, France}

\author{Jonathan~Blazek}
\affiliation{Center for Cosmology and Astro-Particle Physics, The Ohio State University, Columbus, OH 43210, USA}
\affiliation{Institute of Physics, Laboratory of Astrophysics, \'{E}cole Polytechnique Fed\`{e}rale de Lausanne (EPFL), Observatoire de Sauverny, 1290 Versoix, Switzerland}

\author{Roger~D.~Blandford}
\affiliation{Kavli Institute for Particle Astrophysics and Cosmology, SLAC National Accelerator Laboratory, Stanford University, Stanford, CA 94025}

\author{Josh~S.~Bloom}
\affiliation{Astronomy Department,  University of California, 601 Campbell Hall, Berkeley, CA 94720}

\author{Joanne~Bogart}
\affiliation{Kavli Institute for Particle Astrophysics and Cosmology, SLAC National Accelerator Laboratory, Stanford University, Stanford, CA 94025}

\author{Tim~W.~Bond}
\affiliation{SLAC National Accelerator Laboratory,  2575 Sand Hill Rd, Menlo Park CA 94025}

\author{Anders~W.~Borgland}
\affiliation{SLAC National Accelerator Laboratory,  2575 Sand Hill Rd, Menlo Park CA 94025}

\author{Kirk~Borne}
\affiliation{School of Physics, Astronomy and Computational Sciences, George Mason University, 4400 University Drive, Fairfax, VA  22030}

\author[0000-0003-2759-5764]{James~F.~Bosch}
\affiliation{Department of Astrophysical Sciences, Princeton University, Princeton, NJ 08544}

\author[0000-0003-4887-2150]{Dominique~Boutigny}
\affiliation{Universit\'{e} Grenoble-Alpes, Universit\'{e} Savoie Mont Blanc, CNRS/IN2P3 Laboratoire d'Annecy-le-Vieux de Physique des Particules, 9 Chemin de Bellevue -- BP 110, 74940 Annecy-le-Vieux Cedex, France}

\author{Craig~A.~Brackett}
\affiliation{SLAC National Accelerator Laboratory,  2575 Sand Hill Rd, Menlo Park CA 94025}

\author{Andrew~Bradshaw}
\affiliation{Physics Department, University of California, One Shields Avenue, Davis, CA 95616}

\author{William~Nielsen~Brandt}
\affiliation{Department of Astronomy and Astrophysics, The Pennsylvania State University, 525 Davey Lab, University Park, PA 16802}

\author{Michael~E.~Brown}
\affiliation{Division of Geological and Planetary Sciences, California Institute of Technology, Pasadena, CA 91125}

\author{James~S.~Bullock}
\affiliation{Center for Cosmology, University of California, Irvine, CA 92697}

\author{Patricia~Burchat}
\affiliation{Kavli Institute for Particle Astrophysics and Cosmology, SLAC National Accelerator Laboratory, Stanford University, Stanford, CA 94025}

\author{David~L.~Burke}
\affiliation{Kavli Institute for Particle Astrophysics and Cosmology, SLAC National Accelerator Laboratory, Stanford University, Stanford, CA 94025}

\author{Gianpietro~Cagnoli}
\affiliation{Laboratoire des Materiaux Avances (LMA), CNRS/IN2P3, Universit\'{e} de Lyon, F-69622 Villeurbanne, Lyon, France}

\author{Daniel~Calabrese}
\affiliation{LSST Project Office, 950 N.\ Cherry Avenue, Tucson, AZ  85719}

\author{Shawn~Callahan}
\affiliation{LSST Project Office, 950 N.\ Cherry Avenue, Tucson, AZ  85719}

\author{Alice~L.~Callen}
\affiliation{SLAC National Accelerator Laboratory,  2575 Sand Hill Rd, Menlo Park CA 94025}

\author{Srinivasan~Chandrasekharan}
\affiliation{Department of Computer Science, The University of Arizona, 1040 E 4th St, Tucson, AZ 85719}

\author{Glenaver~Charles-Emerson}
\affiliation{LSST Project Office, 950 N.\ Cherry Avenue, Tucson, AZ  85719}

\author{Steve~Chesley}
\affiliation{Jet Propulsion Laboratory, California Institute of Technology, Pasadena, CA 91109}

\author{Elliott~C.~Cheu}
\affiliation{Department of Physics, University of Arizona, 1118 E.\ Fourth Street, Tucson, AZ 85721}

\author[0000-0002-1181-1621]{Hsin-Fang~Chiang}
\affiliation{NCSA, University of Illinois at Urbana-Champaign, 1205 W.\ Clark St., Urbana, IL 61801}

\author{James~Chiang}
\affiliation{Kavli Institute for Particle Astrophysics and Cosmology, SLAC National Accelerator Laboratory, Stanford University, Stanford, CA 94025}

\author{Carol~Chirino}
\affiliation{LSST Project Office, 950 N.\ Cherry Avenue, Tucson, AZ  85719}

\author{Derek~Chow}
\affiliation{SLAC National Accelerator Laboratory,  2575 Sand Hill Rd, Menlo Park CA 94025}

\author{David~R.~Ciardi}
\affiliation{IPAC, California Institute of Technology, MS 100-22, Pasadena, CA 91125}

\author{Charles~F.~Claver}
\affiliation{LSST Project Office, 950 N.\ Cherry Avenue, Tucson, AZ  85719}

\author[0000-0001-9022-4232]{Johann~Cohen-Tanugi}
\affiliation{Laboratoire Univers et Particules de Montpellier (LUPM), Universit\'{e} de Montpellier, CNRS/IN2P3, Place Eug\`{e}ne Bataillon, 34095 Montpellier, France}

\author{Joseph~J.~Cockrum}
\affiliation{LSST Project Office, 950 N.\ Cherry Avenue, Tucson, AZ  85719}

\author[0000-0002-4774-9364]{Rebecca~Coles}
\affiliation{Center for Cosmology and Astro-Particle Physics, The Ohio State University, Columbus, OH 43210, USA}

\author[0000-0001-5576-8189]{Andrew~J.~Connolly}
\affiliation{University of Washington, Dept.\ of Astronomy, Box 351580, Seattle, WA 98195}

\author{Kem~H.~Cook}
\affiliation{Cook Astronomical Consulting, 220 Duxbury CT, San Ramon, CA 94583, USA}

\author{Asantha~Cooray}
\affiliation{Center for Cosmology, University of California, Irvine, CA 92697}

\author{Kevin~R.~Covey}
\affiliation{Western Washington University, 516 High Street, Bellingham, WA 98225}

\author{Chris~Cribbs}
\affiliation{NCSA, University of Illinois at Urbana-Champaign, 1205 W.\ Clark St., Urbana, IL 61801}

\author{Wei~Cui}
\affiliation{Department of Physics and Astronomy, Purdue University, 525 Northwestern Ave., West Lafayette, IN  47907}

\author{Roc~Cutri}
\affiliation{IPAC, California Institute of Technology, MS 100-22, Pasadena, CA 91125}

\author{Philip~N.~Daly}
\affiliation{University of Arizona, Tucson, AZ 85721}

\author{Scott~F.~Daniel}
\affiliation{University of Washington, Dept.\ of Astronomy, Box 351580, Seattle, WA 98195}

\author{Felipe~Daruich}
\affiliation{LSST Project Office, 950 N.\ Cherry Avenue, Tucson, AZ  85719}

\author{Guillaume~Daubard}
\affiliation{Laboratoire de Physique Nucl\'{e}aire et des Hautes Energies, Universit\'{e} Pierre et Marie Curie, Universit\'{e} Paris Diderot, CNRS/IN2P3, 4 place Jussieu, 75005 Paris, France}

\author{Greg~Daues}
\affiliation{NCSA, University of Illinois at Urbana-Champaign, 1205 W.\ Clark St., Urbana, IL 61801}

\author{William~Dawson}
\affiliation{Lawrence Livermore National Laboratory, 7000 East Avenue, Livermore, CA 94550}

\author{Francisco~Delgado}
\affiliation{LSST Project Office, 950 N.\ Cherry Avenue, Tucson, AZ  85719}

\author{Alfred~Dellapenna}
\affiliation{Brookhaven National Laboratory, Upton, NY 11973}

\author{Robert~de~Peyster}
\affiliation{SLAC National Accelerator Laboratory,  2575 Sand Hill Rd, Menlo Park CA 94025}

\author[0000-0002-0455-9384]{Miguel~de~Val-Borro}
\affiliation{Department of Astrophysical Sciences, Princeton University, Princeton, NJ 08544}

\author{Seth~W.~Digel}
\affiliation{SLAC National Accelerator Laboratory,  2575 Sand Hill Rd, Menlo Park CA 94025}

\author{Peter~Doherty}
\affiliation{Department of Physics, Harvard University, 17 Oxford St, Cambridge MA 02138}

\author{Richard~Dubois}
\affiliation{SLAC National Accelerator Laboratory,  2575 Sand Hill Rd, Menlo Park CA 94025}

\author[0000-0003-1598-6979]{Gregory~P.~Dubois-Felsmann}
\affiliation{IPAC, California Institute of Technology, MS 100-22, Pasadena, CA 91125}

\author{Josef~Durech}
\affiliation{Astronomical Institute, Charles University, Praha, Czech Republic}

\author[0000-0002-8333-7615]{Frossie~Economou}
\affiliation{LSST Project Office, 950 N.\ Cherry Avenue, Tucson, AZ  85719}

\author{Tim~Eifler}
\affiliation{Steward Observatory, The University of Arizona, 933 N Cherry Ave., Tucson, AZ 85721}

\author{Michael~Eracleous}
\affiliation{Department of Astronomy and Astrophysics, The Pennsylvania State University, 525 Davey Lab, University Park, PA 16802}

\author{Henry~Ferguson}
\affiliation{Space Telescope Science Institute, 3700 San Martin Drive, Baltimore, MD 21218}

\author{Enrique~Figueroa}
\affiliation{LSST Project Office, 950 N.\ Cherry Avenue, Tucson, AZ  85719}

\author[0000-0001-9440-8960]{Merlin~Fisher-Levine}
\affiliation{Department of Astrophysical Sciences, Princeton University, Princeton, NJ 08544}

\author{Warren~Focke}
\affiliation{SLAC National Accelerator Laboratory,  2575 Sand Hill Rd, Menlo Park CA 94025}

\author{Michael~D.~Foss}
\affiliation{SLAC National Accelerator Laboratory,  2575 Sand Hill Rd, Menlo Park CA 94025}

\author{James~Frank}
\affiliation{Brookhaven National Laboratory, Upton, NY 11973}

\author{Michael~D.~Freemon}
\affiliation{NCSA, University of Illinois at Urbana-Champaign, 1205 W.\ Clark St., Urbana, IL 61801}

\author[0000-0001-6728-1423]{Emmanuel~Gangler}
\affiliation{Universit\'e Clermont Auvergne, CNRS, Laboratoire de Physique de Clermont, F-63000 Clermont-Ferrand, France}

\author{Eric~Gawiser}
\affiliation{Department of Physics and Astronomy, Rutgers University, 136 Frelinghuysen Rd, Piscataway, NJ 08854}

\author{John~C.~Geary}
\affiliation{Smithsonian Astrophysical Observatory, 60 Garden St., Cambridge MA 02138}

\author{Perry~Gee}
\affiliation{Physics Department, University of California, One Shields Avenue, Davis, CA 95616}

\author{Marla~Geha}
\affiliation{Astronomy Department, Yale University, New Haven, CT 06520}

\author{Charles~J.~B.~Gessner}
\affiliation{LSST Project Office, 950 N.\ Cherry Avenue, Tucson, AZ  85719}

\author{Robert~R.~Gibson}
\affiliation{University of Washington, Dept.\ of Astronomy, Box 351580, Seattle, WA 98195}

\author{D.~Kirk~Gilmore}
\affiliation{Kavli Institute for Particle Astrophysics and Cosmology, SLAC National Accelerator Laboratory, Stanford University, Stanford, CA 94025}

\author{Thomas~Glanzman}
\affiliation{SLAC National Accelerator Laboratory,  2575 Sand Hill Rd, Menlo Park CA 94025}

\author{William~Glick}
\affiliation{NCSA, University of Illinois at Urbana-Champaign, 1205 W.\ Clark St., Urbana, IL 61801}

\author{Tatiana~Goldina}
\affiliation{IPAC, California Institute of Technology, MS 100-22, Pasadena, CA 91125}

\author{Daniel~A.~Goldstein}
\affiliation{Astronomy Department,  University of California, 601 Campbell Hall, Berkeley, CA 94720}
\affiliation{Lawrence Berkeley National Laboratory, 1 Cyclotron Road, Berkeley, CA 94720, USA}

\author{Iain~Goodenow}
\affiliation{LSST Project Office, 950 N.\ Cherry Avenue, Tucson, AZ  85719}

\author[0000-0002-9154-3136]{Melissa~L.~Graham}
\affiliation{University of Washington, Dept.\ of Astronomy, Box 351580, Seattle, WA 98195}

\author{William~J.~Gressler}
\affiliation{LSST Project Office, 950 N.\ Cherry Avenue, Tucson, AZ  85719}

\author{Philippe~Gris}
\affiliation{Universit\'e Clermont Auvergne, CNRS, Laboratoire de Physique de Clermont, F-63000 Clermont-Ferrand, France}

\author[0000-0003-0800-8755]{Leanne~P.~Guy}
\affiliation{LSST Project Office, 950 N.\ Cherry Avenue, Tucson, AZ  85719}

\author{Augustin~Guyonnet}
\affiliation{Department of Physics, Harvard University, 17 Oxford St, Cambridge MA 02138}

\author{Gunther~Haller}
\affiliation{SLAC National Accelerator Laboratory,  2575 Sand Hill Rd, Menlo Park CA 94025}

\author{Ron~Harris}
\affiliation{National Optical Astronomy Observatory, 950 N.\ Cherry Ave, Tucson, AZ 85719}

\author{Patrick~A.~Hascall}
\affiliation{SLAC National Accelerator Laboratory,  2575 Sand Hill Rd, Menlo Park CA 94025}

\author{Justine~Haupt}
\affiliation{Brookhaven National Laboratory, Upton, NY 11973}

\author[0000-0001-7203-2552]{Fabio~Hernandez}
\affiliation{CNRS, CC-IN2P3, 21 avenue Pierre de Coubertin, CS70202, 69627 Villeurbanne cedex, France}

\author{Sven~Herrmann}
\affiliation{SLAC National Accelerator Laboratory,  2575 Sand Hill Rd, Menlo Park CA 94025}

\author{Edward~Hileman}
\affiliation{LSST Project Office, 950 N.\ Cherry Avenue, Tucson, AZ  85719}

\author[0000-0002-5292-5879]{Joshua~Hoblitt}
\affiliation{LSST Project Office, 950 N.\ Cherry Avenue, Tucson, AZ  85719}

\author{John~A.~Hodgson}
\affiliation{SLAC National Accelerator Laboratory,  2575 Sand Hill Rd, Menlo Park CA 94025}

\author{Craig~Hogan}
\affiliation{Department of Astronomy and Astrophysics, University of Chicago, 5640 South Ellis Avenue, Chicago, IL 60637}

\author{Dajun~Huang}
\affiliation{Brookhaven National Laboratory, Upton, NY 11973}

\author{Michael~E.~Huffer}
\affiliation{Kavli Institute for Particle Astrophysics and Cosmology, SLAC National Accelerator Laboratory, Stanford University, Stanford, CA 94025}

\author{Patrick~Ingraham}
\affiliation{LSST Project Office, 950 N.\ Cherry Avenue, Tucson, AZ  85719}

\author{Walter~R.~Innes}
\affiliation{Kavli Institute for Particle Astrophysics and Cosmology, SLAC National Accelerator Laboratory, Stanford University, Stanford, CA 94025}

\author{Suzanne~H.~Jacoby}
\affiliation{LSST Project Office, 950 N.\ Cherry Avenue, Tucson, AZ  85719}

\author{Bhuvnesh~Jain}
\affiliation{Department of Physics \& Astronomy, University of Pennsylvania, 209 South 33rd Street, Philadelphia, PA 19104-6396}

\author{Fabrice~Jammes}
\affiliation{Universit\'e Clermont Auvergne, CNRS, Laboratoire de Physique de Clermont, F-63000 Clermont-Ferrand, France}

\author{James~Jee}
\affiliation{Physics Department, University of California, One Shields Avenue, Davis, CA 95616}

\author[0000-0001-5982-167X]{Tim~Jenness}
\affiliation{LSST Project Office, 950 N.\ Cherry Avenue, Tucson, AZ  85719}

\author{Garrett~Jernigan}
\affiliation{Space Sciences Lab, University of California, 7 Gauss Way, Berkeley, CA 94720-7450}

\author{Darko~Jevremovi\'{c}}
\affiliation{Astronomical Observatory, Volgina 7, P.O.\ Box 74, 11060 Belgrade, Serbia}

\author{Kenneth~Johns}
\affiliation{Department of Physics, University of Arizona, 1118 E.\ Fourth Street, Tucson, AZ 85721}

\author[0000-0002-5729-2716]{Anthony~S.~Johnson}
\affiliation{SLAC National Accelerator Laboratory,  2575 Sand Hill Rd, Menlo Park CA 94025}

\author{Margaret~W.~G.~Johnson}
\affiliation{NCSA, University of Illinois at Urbana-Champaign, 1205 W.\ Clark St., Urbana, IL 61801}

\author[0000-0001-5916-0031]{R.~Lynne~Jones}
\affiliation{University of Washington, Dept.\ of Astronomy, Box 351580, Seattle, WA 98195}

\author{Claire~Juramy-Gilles}
\affiliation{Laboratoire de Physique Nucl\'{e}aire et des Hautes Energies, Universit\'{e} Pierre et Marie Curie, Universit\'{e} Paris Diderot, CNRS/IN2P3, 4 place Jussieu, 75005 Paris, France}

\author[0000-0003-1996-9252]{Mario~Juri\'{c}}
\affiliation{University of Washington, Dept.\ of Astronomy, Box 351580, Seattle, WA 98195}

\author{Jason~S.~Kalirai}
\affiliation{Space Telescope Science Institute, 3700 San Martin Drive, Baltimore, MD 21218}

\author{Nitya~J.~Kallivayalil}
\affiliation{Department of Astronomy, University of Virginia, Charlottesville, VA 22904}

\author[0000-0002-6825-5283]{Bryce~Kalmbach}
\affiliation{University of Washington, Dept.\ of Astronomy, Box 351580, Seattle, WA 98195}

\author{Jeffrey~P.~Kantor}
\affiliation{LSST Project Office, 950 N.\ Cherry Avenue, Tucson, AZ  85719}

\author{Pierre~Karst}
\affiliation{Aix Marseille Univ, CNRS/IN2P3, CPPM, Marseille, France}

\author[0000-0001-7970-0760]{Mansi~M.~Kasliwal}
\affiliation{Astronomy Department, California Institute of Technology, 1200 East California Blvd., Pasadena CA 91125}

\author{Heather~Kelly}
\affiliation{SLAC National Accelerator Laboratory,  2575 Sand Hill Rd, Menlo Park CA 94025}

\author{Richard~Kessler}
\affiliation{Department of Astronomy and Astrophysics, University of Chicago, 5640 South Ellis Avenue, Chicago, IL 60637}

\author{Veronica~Kinnison}
\affiliation{LSST Project Office, 950 N.\ Cherry Avenue, Tucson, AZ  85719}

\author{David~Kirkby}
\affiliation{Department of Physics and Astronomy, University of California, 4129 Frederick Reines Hall, Irvine, CA 92697}

\author{Lloyd~Knox}
\affiliation{Physics Department, University of California, One Shields Avenue, Davis, CA 95616}

\author{Ivan~V.~Kotov}
\affiliation{Brookhaven National Laboratory, Upton, NY 11973}

\author{Victor~L.~Krabbendam}
\affiliation{LSST Project Office, 950 N.\ Cherry Avenue, Tucson, AZ  85719}

\author[0000-0002-4410-7868]{K.~Simon~Krughoff}
\affiliation{LSST Project Office, 950 N.\ Cherry Avenue, Tucson, AZ  85719}

\author{Petr~Kub\'{a}nek}
\affiliation{Institute of Physics, Academy of Sciences of the Czech Republic, Na Slovance 2, 182 21 Praha 8, Czech Republic}

\author{John~Kuczewski}
\affiliation{Brookhaven National Laboratory, Upton, NY 11973}

\author{Shri~Kulkarni}
\affiliation{Astronomy Department, California Institute of Technology, 1200 East California Blvd., Pasadena CA 91125}

\author{John~Ku}
\affiliation{SLAC National Accelerator Laboratory,  2575 Sand Hill Rd, Menlo Park CA 94025}

\author{Nadine~R.~Kurita}
\affiliation{SLAC National Accelerator Laboratory,  2575 Sand Hill Rd, Menlo Park CA 94025}

\author{Craig~S.~Lage}
\affiliation{Physics Department, University of California, One Shields Avenue, Davis, CA 95616}

\author{Ron~Lambert}
\affiliation{LSST Project Office, 950 N.\ Cherry Avenue, Tucson, AZ  85719}
\affiliation{Cerro Tololo InterAmerican Observatory, La Serena, Chile}

\author{Travis~Lange}
\affiliation{SLAC National Accelerator Laboratory,  2575 Sand Hill Rd, Menlo Park CA 94025}

\author{J.~Brian~Langton}
\affiliation{SLAC National Accelerator Laboratory,  2575 Sand Hill Rd, Menlo Park CA 94025}

\author{Laurent~Le~Guillou}
\affiliation{Sorbonne Universit\'{e}s, UPMC Univ Paris 06, UMR 7585, LPNHE, F-75005, Paris, France}
\affiliation{Laboratoire de Physique Nucl\'{e}aire et des Hautes Energies, Universit\'{e} Pierre et Marie Curie, Universit\'{e} Paris Diderot, CNRS/IN2P3, 4 place Jussieu, 75005 Paris, France}

\author{Deborah~Levine}
\affiliation{IPAC, California Institute of Technology, MS 100-22, Pasadena, CA 91125}

\author{Ming~Liang}
\affiliation{LSST Project Office, 950 N.\ Cherry Avenue, Tucson, AZ  85719}

\author[0000-0002-6338-6516]{Kian-Tat~Lim}
\affiliation{SLAC National Accelerator Laboratory,  2575 Sand Hill Rd, Menlo Park CA 94025}

\author{Chris~J.~Lintott}
\affiliation{Department of Physics, University of Oxford, Denys Wilkinson Building, Keble Road, Oxford, OX1 3RH, UK}

\author{Kevin~E.~Long}
\affiliation{Longhorn Industries, Ellensburg, WA 98926}

\author{Margaux~Lopez}
\affiliation{SLAC National Accelerator Laboratory,  2575 Sand Hill Rd, Menlo Park CA 94025}

\author{Paul~J.~Lotz}
\affiliation{LSST Project Office, 950 N.\ Cherry Avenue, Tucson, AZ  85719}

\author[0000-0003-1666-0962]{Robert~H.~Lupton}
\affiliation{Department of Astrophysical Sciences, Princeton University, Princeton, NJ 08544}

\author[0000-0002-4122-9384]{Nate~B.~Lust}
\affiliation{Department of Astrophysical Sciences, Princeton University, Princeton, NJ 08544}

\author{Lauren~A.~MacArthur}
\affiliation{Department of Astrophysical Sciences, Princeton University, Princeton, NJ 08544}

\author{Ashish~Mahabal}
\affiliation{Astronomy Department, California Institute of Technology, 1200 East California Blvd., Pasadena CA 91125}

\author{Rachel~Mandelbaum}
\affiliation{McWilliams Center for Cosmology, Department of Physics, Carnegie Mellon University, Pittsburgh, PA 15213, USA}

\author{Darren~S.~Marsh}
\affiliation{SLAC National Accelerator Laboratory,  2575 Sand Hill Rd, Menlo Park CA 94025}

\author{Philip~J.~Marshall}
\affiliation{Kavli Institute for Particle Astrophysics and Cosmology, SLAC National Accelerator Laboratory, Stanford University, Stanford, CA 94025}

\author{Stuart~Marshall}
\affiliation{Kavli Institute for Particle Astrophysics and Cosmology, SLAC National Accelerator Laboratory, Stanford University, Stanford, CA 94025}

\author{Morgan~May}
\affiliation{Brookhaven National Laboratory, Upton, NY 11973}

\author{Robert~McKercher}
\affiliation{LSST Project Office, 950 N.\ Cherry Avenue, Tucson, AZ  85719}

\author{Michelle~McQueen}
\affiliation{Brookhaven National Laboratory, Upton, NY 11973}

\author[0000-0002-2308-4230]{Joshua~Meyers}
\affiliation{Department of Astrophysical Sciences, Princeton University, Princeton, NJ 08544}

\author{Myriam~Migliore}
\affiliation{Laboratoire de Physique Subatomique et de Cosmologie, Universit\'{e} Grenoble-Alpes, CNRS/IN2P3, 53 av.\ des Martyrs, 38026 Grenoble cedex, France}

\author{Michelle~Miller}
\affiliation{National Optical Astronomy Observatory, 950 N.\ Cherry Ave, Tucson, AZ 85719}

\author{David~J.~Mills}
\affiliation{LSST Project Office, 950 N.\ Cherry Avenue, Tucson, AZ  85719}

\author{Connor~Miraval}
\affiliation{Brookhaven National Laboratory, Upton, NY 11973}

\author[0000-0001-5820-3925]{Joachim~Moeyens}
\affiliation{University of Washington, Dept.\ of Astronomy, Box 351580, Seattle, WA 98195}

\author{David~G.~Monet}
\affiliation{U.S.\ Naval Observatory Flagstaff Station, 10391 Naval Observatory Road, Flagstaff, AZ 86001}

\author{Marc~Moniez}
\affiliation{Laboratoire de l'Acc\'{e}l\'{e}rateur Lin\'{e}aire, CNRS/IN2P3, Universit\'{e} de Paris-Sud, B.P.\ 34, 91898 Orsay Cedex, France}

\author{Serge~Monkewitz}
\affiliation{IPAC, California Institute of Technology, MS 100-22, Pasadena, CA 91125}

\author{Christopher~Montgomery}
\affiliation{LSST Project Office, 950 N.\ Cherry Avenue, Tucson, AZ  85719}

\author{Fritz~Mueller}
\affiliation{SLAC National Accelerator Laboratory,  2575 Sand Hill Rd, Menlo Park CA 94025}

\author{Gary~P.~Muller}
\affiliation{LSST Project Office, 950 N.\ Cherry Avenue, Tucson, AZ  85719}

\author{Freddy~Mu\~noz~Arancibia}
\affiliation{LSST Project Office, 950 N.\ Cherry Avenue, Tucson, AZ  85719}

\author{Douglas~R.~Neill}
\affiliation{LSST Project Office, 950 N.\ Cherry Avenue, Tucson, AZ  85719}

\author{Scott~P.~Newbry}
\affiliation{SLAC National Accelerator Laboratory,  2575 Sand Hill Rd, Menlo Park CA 94025}

\author{Jean-Yves~Nief}
\affiliation{CNRS, CC-IN2P3, 21 avenue Pierre de Coubertin, CS70202, 69627 Villeurbanne cedex, France}

\author{Andrei~Nomerotski}
\affiliation{Brookhaven National Laboratory, Upton, NY 11973}

\author{Martin~Nordby}
\affiliation{SLAC National Accelerator Laboratory,  2575 Sand Hill Rd, Menlo Park CA 94025}

\author{Paul~O'Connor}
\affiliation{Brookhaven National Laboratory, Upton, NY 11973}

\author{John~Oliver}
\affiliation{Department of Physics, Harvard University, 17 Oxford St, Cambridge MA 02138}
\affiliation{Department of Astronomy, Center for Astrophysics, Harvard University, 60 Garden St., Cambridge, MA 02138}

\author{Scot~S.~Olivier}
\affiliation{Lawrence Livermore National Laboratory, 7000 East Avenue, Livermore, CA 94550}

\author{Knut~Olsen}
\affiliation{National Optical Astronomy Observatory, 950 N.\ Cherry Ave, Tucson, AZ 85719}

\author[0000-0003-4141-6195]{William~O'Mullane}
\affiliation{LSST Project Office, 950 N.\ Cherry Avenue, Tucson, AZ  85719}

\author{Sandra~Ortiz}
\affiliation{LSST Project Office, 950 N.\ Cherry Avenue, Tucson, AZ  85719}

\author{Shawn~Osier}
\affiliation{SLAC National Accelerator Laboratory,  2575 Sand Hill Rd, Menlo Park CA 94025}

\author{Russell~E.~Owen}
\affiliation{University of Washington, Dept.\ of Astronomy, Box 351580, Seattle, WA 98195}

\author{Reynald~Pain}
\affiliation{Laboratoire de Physique Nucl\'{e}aire et des Hautes Energies, Universit\'{e} Pierre et Marie Curie, Universit\'{e} Paris Diderot, CNRS/IN2P3, 4 place Jussieu, 75005 Paris, France}

\author{Paul~E.~Palecek}
\affiliation{Brookhaven National Laboratory, Upton, NY 11973}

\author{John~K.~Parejko}
\affiliation{University of Washington, Dept.\ of Astronomy, Box 351580, Seattle, WA 98195}

\author{James~B.~Parsons}
\affiliation{NCSA, University of Illinois at Urbana-Champaign, 1205 W.\ Clark St., Urbana, IL 61801}

\author[0000-0002-9701-5975]{Nathan~M.~Pease}
\affiliation{SLAC National Accelerator Laboratory,  2575 Sand Hill Rd, Menlo Park CA 94025}

\author[0000-0002-6564-6246]{J.~Matt~Peterson}
\affiliation{LSST Project Office, 950 N.\ Cherry Avenue, Tucson, AZ  85719}

\author{John~R.~Peterson}
\affiliation{Department of Physics and Astronomy, Purdue University, 525 Northwestern Ave., West Lafayette, IN  47907}

\author{Donald~L.~Petravick}
\affiliation{NCSA, University of Illinois at Urbana-Champaign, 1205 W.\ Clark St., Urbana, IL 61801}

\author{M.~E.~Libby~Petrick}
\affiliation{LSST Project Office, 950 N.\ Cherry Avenue, Tucson, AZ  85719}

\author{Cathy~E.~Petry}
\affiliation{LSST Project Office, 950 N.\ Cherry Avenue, Tucson, AZ  85719}

\author{Francesco~Pierfederici}
\affiliation{Instituto de Radioastronom\'ia Milim\'{e}trica, Av.\ Divina Pastora 7, N\'{u}cleo Central, E-18012 Granada, Spain}

\author{Stephen~Pietrowicz}
\affiliation{NCSA, University of Illinois at Urbana-Champaign, 1205 W.\ Clark St., Urbana, IL 61801}

\author{Rob~Pike}
\affiliation{Google Inc., 1600 Amphitheatre Parkway Mountain View, CA 94043}

\author{Philip~A.~Pinto}
\affiliation{Steward Observatory, The University of Arizona, 933 N Cherry Ave., Tucson, AZ 85721}

\author{Raymond~Plante}
\affiliation{NCSA, University of Illinois at Urbana-Champaign, 1205 W.\ Clark St., Urbana, IL 61801}

\author{Stephen~Plate}
\affiliation{Brookhaven National Laboratory, Upton, NY 11973}

\author{Paul~A.~Price}
\affiliation{Department of Astrophysical Sciences, Princeton University, Princeton, NJ 08544}

\author{Michael~Prouza}
\affiliation{Institute of Physics, Academy of Sciences of the Czech Republic, Na Slovance 2, 182 21 Praha 8, Czech Republic}

\author{Veljko~Radeka}
\affiliation{Brookhaven National Laboratory, Upton, NY 11973}

\author{Jayadev~Rajagopal}
\affiliation{National Optical Astronomy Observatory, 950 N.\ Cherry Ave, Tucson, AZ 85719}

\author{Andrew~P.~Rasmussen}
\affiliation{Department of Physics and Astronomy, Purdue University, 525 Northwestern Ave., West Lafayette, IN  47907}

\author{Nicolas~Regnault}
\affiliation{Laboratoire de Physique Nucl\'{e}aire et des Hautes Energies, Universit\'{e} Pierre et Marie Curie, Universit\'{e} Paris Diderot, CNRS/IN2P3, 4 place Jussieu, 75005 Paris, France}

\author{Kevin~A.~Reil}
\affiliation{SLAC National Accelerator Laboratory,  2575 Sand Hill Rd, Menlo Park CA 94025}

\author{David~J.~Reiss}
\affiliation{University of Washington, Dept.\ of Astronomy, Box 351580, Seattle, WA 98195}

\author[0000-0003-3881-8310]{Michael~A.~Reuter}
\affiliation{LSST Project Office, 950 N.\ Cherry Avenue, Tucson, AZ  85719}

\author{Stephen~T.~Ridgway}
\affiliation{National Optical Astronomy Observatory, 950 N.\ Cherry Ave, Tucson, AZ 85719}

\author{Vincent~J.~Riot}
\affiliation{Lawrence Livermore National Laboratory, 7000 East Avenue, Livermore, CA 94550}

\author{Steve~Ritz}
\affiliation{Santa Cruz Institute for Particle Physics and Physics Department, University of California--Santa Cruz, 1156 High St., Santa Cruz, CA 95064}

\author{Sean~Robinson}
\affiliation{Brookhaven National Laboratory, Upton, NY 11973}

\author{William~Roby}
\affiliation{IPAC, California Institute of Technology, MS 100-22, Pasadena, CA 91125}

\author[0000-0001-5326-3486]{Aaron~Roodman}
\affiliation{SLAC National Accelerator Laboratory,  2575 Sand Hill Rd, Menlo Park CA 94025}

\author{Wayne~Rosing}
\affiliation{Las Cumbres Observatory, 6740 Cortona Dr.\ Suite 102, Santa Barbara, CA 93117}

\author{Cecille~Roucelle}
\affiliation{AstroParticule et Cosmologie, Universit\'{e} Paris Diderot, CNRS/IN2P3, CEA/lrfu, Observatoire de Paris, Sorbonne Paris Cit\'{e}, 10, rue Alice Domon et L\'{e}onie Duquet, Paris Cedex 13, France}

\author{Matthew~R.~Rumore}
\affiliation{Brookhaven National Laboratory, Upton, NY 11973}

\author{Stefano~Russo}
\affiliation{SLAC National Accelerator Laboratory,  2575 Sand Hill Rd, Menlo Park CA 94025}

\author{Abhijit~Saha}
\affiliation{National Optical Astronomy Observatory, 950 N.\ Cherry Ave, Tucson, AZ 85719}

\author{Benoit~Sassolas}
\affiliation{Laboratoire des Materiaux Avances (LMA), CNRS/IN2P3, Universit\'{e} de Lyon, F-69622 Villeurbanne, Lyon, France}

\author{Terry~L.~Schalk}
\affiliation{Santa Cruz Institute for Particle Physics and Physics Department, University of California--Santa Cruz, 1156 High St., Santa Cruz, CA 95064}

\author[0000-0002-8324-0880]{Pim~Schellart}
\affiliation{Department of Astrophysical Sciences, Princeton University, Princeton, NJ 08544}
\affiliation{Department of Astrophysics/IMAPP, Radboud University Nijmegen, P.O.\ Box 9010, 6500 GL Nijmegen, The Netherlands}

\author{Rafe~H.~Schindler}
\affiliation{Kavli Institute for Particle Astrophysics and Cosmology, SLAC National Accelerator Laboratory, Stanford University, Stanford, CA 94025}

\author{Samuel~Schmidt}
\affiliation{Physics Department, University of California, One Shields Avenue, Davis, CA 95616}

\author{Donald~P.~Schneider}
\affiliation{Department of Astronomy and Astrophysics, The Pennsylvania State University, 525 Davey Lab, University Park, PA 16802}

\author{Michael~D.~Schneider}
\affiliation{Lawrence Livermore National Laboratory, 7000 East Avenue, Livermore, CA 94550}

\author{William~Schoening}
\affiliation{LSST Project Office, 950 N.\ Cherry Avenue, Tucson, AZ  85719}

\author{German~Schumacher}
\affiliation{LSST Project Office, 950 N.\ Cherry Avenue, Tucson, AZ  85719}
\affiliation{Cerro Tololo InterAmerican Observatory, La Serena, Chile}

\author{Megan~E.~Schwamb}
\affiliation{Gemini Observatory, Northern Operations Center, 670 North A'ohoku Place, Hilo, HI 96720, USA}

\author{Jacques~Sebag}
\affiliation{LSST Project Office, 950 N.\ Cherry Avenue, Tucson, AZ  85719}

\author{Brian~Selvy}
\affiliation{LSST Project Office, 950 N.\ Cherry Avenue, Tucson, AZ  85719}

\author{Glenn~H.~Sembroski}
\affiliation{Department of Physics and Astronomy, Purdue University, 525 Northwestern Ave., West Lafayette, IN  47907}

\author{Lynn~G.~Seppala}
\affiliation{Lawrence Livermore National Laboratory, 7000 East Avenue, Livermore, CA 94550}

\author{Andrew~Serio}
\affiliation{LSST Project Office, 950 N.\ Cherry Avenue, Tucson, AZ  85719}

\author{Eduardo~Serrano}
\affiliation{LSST Project Office, 950 N.\ Cherry Avenue, Tucson, AZ  85719}

\author{Richard~A.~Shaw}
\affiliation{Space Telescope Science Institute, 3700 San Martin Drive, Baltimore, MD 21218}

\author{Ian~Shipsey}
\affiliation{Department of Physics, University of Oxford, Denys Wilkinson Building, Keble Road, Oxford, OX1 3RH, UK}

\author[0000-0003-3001-676X]{Jonathan~Sick}
\affiliation{LSST Project Office, 950 N.\ Cherry Avenue, Tucson, AZ  85719}

\author{Nicole~Silvestri}
\affiliation{University of Washington, Dept.\ of Astronomy, Box 351580, Seattle, WA 98195}

\author[0000-0002-0558-0521]{Colin~T.~Slater}
\affiliation{University of Washington, Dept.\ of Astronomy, Box 351580, Seattle, WA 98195}

\author{J.~Allyn~Smith}
\affiliation{Austin Peay State University, Clarksville, TN 37044}

\author{R.~Chris~Smith}
\affiliation{Cerro Tololo InterAmerican Observatory, La Serena, Chile}

\author{Shahram~Sobhani}
\affiliation{Belldex IT Consulting, Tucson, AZ 85742}

\author{Christine~Soldahl}
\affiliation{SLAC National Accelerator Laboratory,  2575 Sand Hill Rd, Menlo Park CA 94025}

\author{Lisa~Storrie-Lombardi}
\affiliation{IPAC, California Institute of Technology, MS 100-22, Pasadena, CA 91125}

\author{Edward~Stover}
\affiliation{LSST Project Office, 950 N.\ Cherry Avenue, Tucson, AZ  85719}

\author[0000-0002-0106-7755]{Michael~A.~Strauss}
\affiliation{Department of Astrophysical Sciences, Princeton University, Princeton, NJ 08544}

\author{Rachel~A.~Street}
\affiliation{Las Cumbres Observatory, 6740 Cortona Dr.\ Suite 102, Santa Barbara, CA 93117}

\author[0000-0003-0347-1724]{Christopher~W.~Stubbs}
\affiliation{Department of Physics, Harvard University, 17 Oxford St, Cambridge MA 02138}
\affiliation{Department of Astronomy, Center for Astrophysics, Harvard University, 60 Garden St., Cambridge, MA 02138}

\author[0000-0001-8708-251X]{Ian~S.~Sullivan}
\affiliation{University of Washington, Dept.\ of Astronomy, Box 351580, Seattle, WA 98195}

\author{Donald~Sweeney}
\affiliation{LSST Project Office, 950 N.\ Cherry Avenue, Tucson, AZ  85719}

\author[0000-0001-9445-1846]{John~D.~Swinbank}
\affiliation{University of Washington, Dept.\ of Astronomy, Box 351580, Seattle, WA 98195}
\affiliation{Department of Astrophysical Sciences, Princeton University, Princeton, NJ 08544}

\author{Alexander~Szalay}
\affiliation{Department of Physics and Astronomy, The John Hopkins University, 3701 San Martin Drive, Baltimore, MD 21218}

\author{Peter~Takacs}
\affiliation{Brookhaven National Laboratory, Upton, NY 11973}

\author{Stephen~A.~Tether}
\affiliation{SLAC National Accelerator Laboratory,  2575 Sand Hill Rd, Menlo Park CA 94025}

\author{Jon~J.~Thaler}
\affiliation{University of Illinois, Physics and Astronomy Departments, 1110 W.\ Green St., Urbana, IL  61801}

\author{John~Gregg~Thayer}
\affiliation{SLAC National Accelerator Laboratory,  2575 Sand Hill Rd, Menlo Park CA 94025}

\author{Sandrine~Thomas}
\affiliation{LSST Project Office, 950 N.\ Cherry Avenue, Tucson, AZ  85719}

\author{Vaikunth~Thukral}
\affiliation{SLAC National Accelerator Laboratory,  2575 Sand Hill Rd, Menlo Park CA 94025}

\author{Jeffrey~Tice}
\affiliation{SLAC National Accelerator Laboratory,  2575 Sand Hill Rd, Menlo Park CA 94025}

\author{David~E.~Trilling}
\affiliation{Department of Physics and Astronomy, Northern Arizona University, PO Box 6010, Flagstaff, AZ 86011, USA}

\author{Max~Turri}
\affiliation{SLAC National Accelerator Laboratory,  2575 Sand Hill Rd, Menlo Park CA 94025}

\author{Richard~Van~Berg}
\affiliation{SLAC National Accelerator Laboratory,  2575 Sand Hill Rd, Menlo Park CA 94025}
\affiliation{Department of Physics \& Astronomy, University of Pennsylvania, 209 South 33rd Street, Philadelphia, PA 19104-6396}

\author{Daniel~Vanden~Berk}
\affiliation{Saint Vincent College, Department of Physics, 300 Fraser Purchase Road, Latrobe, PA 15650}

\author{Kurt~Vetter}
\affiliation{Brookhaven National Laboratory, Upton, NY 11973}

\author{Francoise~Virieux}
\affiliation{AstroParticule et Cosmologie, Universit\'{e} Paris Diderot, CNRS/IN2P3, CEA/lrfu, Observatoire de Paris, Sorbonne Paris Cit\'{e}, 10, rue Alice Domon et L\'{e}onie Duquet, Paris Cedex 13, France}

\author{Tomislav~Vucina}
\affiliation{LSST Project Office, 950 N.\ Cherry Avenue, Tucson, AZ  85719}

\author{William~Wahl}
\affiliation{Brookhaven National Laboratory, Upton, NY 11973}

\author[0000-0003-2918-8687]{Lucianne~Walkowicz}
\affiliation{Library of Congress, 101 Independence Ave SE, Washington, DC 20540}
\affiliation{The Adler Planetarium, 1300 South Lakeshore Ave, Chicago, IL 60605, USA}

\author{Brian~Walsh}
\affiliation{Brookhaven National Laboratory, Upton, NY 11973}

\author[0000-0003-2035-2380]{Christopher~W.~Walter}
\affiliation{Department of Physics, Duke University, Durham, NC 27708}

\author{Daniel~L.~Wang}
\affiliation{SLAC National Accelerator Laboratory,  2575 Sand Hill Rd, Menlo Park CA 94025}

\author{Shin-Yawn~Wang}
\affiliation{IPAC, California Institute of Technology, MS 100-22, Pasadena, CA 91125}

\author{Michael~Warner}
\affiliation{Cerro Tololo InterAmerican Observatory, La Serena, Chile}

\author{Oliver~Wiecha}
\affiliation{LSST Project Office, 950 N.\ Cherry Avenue, Tucson, AZ  85719}

\author[0000-0003-2892-9906]{Beth~Willman}
\affiliation{LSST Project Office, 950 N.\ Cherry Avenue, Tucson, AZ  85719}
\affiliation{Steward Observatory, The University of Arizona, 933 N Cherry Ave., Tucson, AZ 85721}

\author{Scott~E.~Winters}
\affiliation{Lawrence Livermore National Laboratory, 7000 East Avenue, Livermore, CA 94550}

\author{David~Wittman}
\affiliation{Physics Department, University of California, One Shields Avenue, Davis, CA 95616}

\author{Sidney~C.~Wolff}
\affiliation{LSST Project Office, 950 N.\ Cherry Avenue, Tucson, AZ  85719}

\author[0000-0001-7113-1233]{W.~Michael~Wood-Vasey}
\affiliation{Department of Physics and Astronomy, University of Pittsburgh, 3941 O'Hara Street, Pittsburgh PA 15260}

\author{Xiuqin~Wu}
\affiliation{IPAC, California Institute of Technology, MS 100-22, Pasadena, CA 91125}

\author{Bo~Xin}
\affiliation{LSST Project Office, 950 N.\ Cherry Avenue, Tucson, AZ  85719}

\author[0000-0003-2874-6464]{Peter~Yoachim}
\affiliation{University of Washington, Dept.\ of Astronomy, Box 351580, Seattle, WA 98195}

\author{Hu~Zhan}
\affiliation{Key Laboratory of Optical Astronomy, National Astronomical Observatories, Chinese Academy of Sciences, 20A Datun Road, Chaoyang District, Beijing 100012, China}


\begin{abstract}
  We describe here the most ambitious survey currently planned in the optical,
  the Large Synoptic Survey Telescope (LSST). The LSST design is driven by four
  main science themes: probing dark energy and dark matter, taking an inventory
  of the solar system, exploring the transient optical sky, and mapping the Milky Way.
  LSST will be a large, wide-field ground-based system designed to obtain repeated
  images covering the sky visible from Cerro Pach'{o}n in northern Chile. The telescope
  will have an 8.4 m (6.5 m effective) primary mirror, a 9.6 deg$^2$ field of view, a
  3.2-gigapixel camera, and six filters ($ugrizy$) covering the wavelength range
  320--1050 nm. The project is in the construction phase and will begin regular survey
  operations by 2022. About 90\% of the observing time will be devoted to a deep-wide-fast
  survey mode that will uniformly observe a 18,000 deg$^2$ region about 800 times
  (summed over all six bands) during the anticipated 10~yr of operations and will
  yield a co-added map to $r\sim27.5$. These data will result in databases including about
  32 trillion observations of 20 billion galaxies and a similar number of stars, and they will
  serve the majority of the primary science programs. The remaining 10\% of the observing
  time will be allocated to special projects such as a Very Deep and Very Fast time domain
  surveys, whose details are currently under discussion. We illustrate how the LSST science
  drivers led to these choices of system parameters, and we describe the expected data products
  and their characteristics.
\end{abstract}

\keywords{
  surveys ---
  methods: observational ---
  astrometry ---
  cosmology: observations ---
  Galaxy: general ---
  stars: general
}

% latex lsst; dvips -Ppdf -o lsst.ps lsst.dvi; ps2pdf14 lsst.ps lsst.pdf




\section{INTRODUCTION}

Major advances in our understanding of the Universe have historically arisen
from dramatic improvements in our ability to ``see''. We have developed
progressively larger telescopes over the past century, allowing us 
to peer further into space, and further back in time. With the development of 
advanced instrumentation -- imaging, spectroscopic, and polarimetric -- we 
have been able to parse radiation detected from distant sources over the 
full electromagnetic spectrum in increasingly subtle ways. 
These data have provided the detailed information needed to construct physical 
models of planets, stars, galaxies, quasars, and larger structures. 

Until recently, most astronomical investigations have focused on small samples 
of cosmic sources or individual objects. This is because our largest telescope 
facilities typically had rather small fields of view, and those with large
fields of view could not detect very faint sources. With all of our existing 
telescope facilities, we have still surveyed only a small fraction of the 
observable Universe (except when considering the most luminous quasars). 

Over the past two decades, however, advances in technology have made it possible to 
move beyond the traditional observational paradigm and to undertake large-scale 
sky surveys. As vividly demonstrated by surveys such as the Sloan Digital Sky 
Survey (SDSS; York et al.~2000), the Two Micron All Sky Survey (2MASS;
Skrutskie et al.~2006), and the Galaxy Evolution Explorer (GALEX; Martin et
al.~2006), to name but a few, sensitive and accurate 
multi-color surveys over a large fraction of the sky enable an extremely broad range of 
new scientific investigations. These projects, based on a synergy of advances in 
telescope construction, detectors, and above all, information technology, 
have dramatically impacted nearly all fields of astronomy 
-- and many areas of fundamental physics. In addition, the world-wide attention 
received by Sky in Google Earth\footnote{http://earth.google.com/sky/}
(Scranton et al.~2007)  and the hundreds of thousands of volunteers
classifying galaxies in the Galaxy Zoo project (Lintott et al.~2011)
and its extensions 
demonstrate that the impact of sky surveys extends
far beyond fundamental science progress and reaches all of society.

Motivated by the evident scientific progress enabled by large sky surveys,
three nationally-endorsed reports by the U.S. National Academy of Sciences\footnote{
  Astronomy and Astrophysics in the New Millennium, NAS 2001; 
  Connecting Quarks with the Cosmos: Eleven Science Questions for the New Century, NAS 2003; 
  New Frontiers in the Solar System: An Integrated Exploration Strategy, NAS 2003.   
}
concluded that a dedicated ground-based wide-field imaging telescope with an effective aperture 
of 6--8 meters is a high priority for planetary science, astronomy, and physics 
over the next decade. The Large Synoptic Survey Telescope (LSST) described here is 
such a system. The LSST will be a large, wide-field ground-based telescope 
designed to obtain multi-band images over a substantial fraction of the sky
every few nights. The survey will yield contiguous overlapping imaging of over
half the sky in six optical bands, with each sky location visited close to 1000 times over 
10 years. The recent 2010 report ``New Worlds, New Horizons in Astronomy and Astrophysics''
by the Committee for a Decadal Survey of Astronomy and 
Astrophysics\footnote{http://www.nap.edu/catalog.php?record\_id=12951}
ranked LSST as its top priority for large ground-based projects, and in May 2014 the National 
Science Board approved the project for construction. 

The purpose of this paper is to provide an overall summary of the main LSST
science drivers and how they led to the current system design parameters 
(\S~\ref{Sec:refdesign}), to describe anticipated data products (\S~\ref{Sec:dataprod}), 
and to provide a few examples of the science programs that LSST will enable 
(\S~\ref{Sec:science}). The community involvement is discussed in \S~\ref{Sec:community}, 
and broad educational and societal impacts in \S~\ref{Sec:impact}. Concluding 
remarks are presented in \S~\ref{Sec:conclusions}. This publication will be maintained 
at the arXiv.org site\footnote{http://arxiv.org/abs/0805.2366}, and will also
be available from the LSST website (www.lsst.org). The latest arXiv version of this paper 
should be consulted and referenced for the most up-to-date information about the 
LSST system.

%\newpage
\section{FROM SCIENCE DRIVERS TO REFERENCE DESIGN}
\label{Sec:refdesign}

The most important characteristic that determines the speed at which a system can 
survey a given sky area to a given flux limit (i.e., its depth) is its \'etendue
(or grasp), the product of its primary mirror area and the angular
area of its field of view (for a given set of observing conditions, such as
seeing and sky brightness). 
The effective \'etendue for LSST will be greater than 300 m$^2$ deg$^2$, which
is more than an order of magnitude larger than that of any existing facility.  
For example, the SDSS, with its 2.5-m telescope (Gunn et al.~2006) and a
camera with 30 imaging CCDs (Gunn et al.~1998), has an effective \'etendue of 
only 5.9 m$^2$ deg$^2$.

The range of scientific investigations which will be enabled by such a 
dramatic improvement in survey capability is extremely broad. Guided by
the community-wide input assembled in the report of the Science Working Group of the
LSST in 2004\footnote{Available as 
http://www.lsst.org/files/docs/DRM2.pdf}, the LSST is designed to 
achieve goals set by four main science themes:

\begin{enumerate}
\item Probing Dark Energy and Dark Matter;
\item Taking an Inventory of the Solar System;
\item Exploring the Transient Optical Sky;
\item Mapping the Milky Way.
\end{enumerate}

Each of these four themes itself encompasses a variety of analyses, with 
varying sensitivity to instrumental and system parameters. These themes 
fully exercise the technical capabilities of the system, such as photometric 
and astrometric accuracy and image quality. About 90\% of the observing time 
will be devoted to a deep-wide-fast (main) survey mode. The working paradigm is that all 
scientific investigations will utilize a common database constructed from an optimized 
observing program (the main survey mode), such as that discussed in
\S~\ref{sec:baseline}. 
Here we briefly describe these science goals and the most challenging requirements for the 
telescope and instrument that are derived from those goals, which will
inform the overall system design decisions discussed below.
For a more detailed discussion, we refer the reader to the LSST Science Requirements 
Document\footnote{Available at 
http://www.lsst.org/files/docs/SRD.pdf}, 
\B{the LSST Science Book}
%\footnote{Available at http://www.lsst.org/lsst/SciBook and as
%arXiv:0912.0201} 
(2009; hereafter SciBook),
as well as to numerous LSST poster presentations at recent
meetings of the AAS\footnote{See
http://www.lsst.org/lsst/content/aas2013 and further links at
http://www.lsst.org/lsst/news .}. 


\subsection{The Main Science Drivers }

The main science drivers are used to optimize various system parameters.
Ultimately, in this high-dimensional parameter space, there is a 
manifold defined by the total project cost. The science
drivers must both justify this cost, as well as provide guidance
on how to optimize various parameters while staying within the cost envelope.

Here we summarize the dozen or so most important interlocking constraints on data 
and system properties placed by the four main science themes: 

\begin{enumerate}
\item  The depth of a single visit;
\item  Image quality;
\item  Photometric accuracy;
\item  Astrometric accuracy;
\item  Optimal exposure time;
\item  The filter complement;
\item  The distribution of revisit times (i.e., the cadence of observations),
                including the survey lifetime;
\item  The total number of visits to a given area of sky;
\item  The coadded survey depth;
\item  The distribution of visits on the sky, and the total sky coverage;
\item  The distribution of visits per filter; and 
\item  Data processing and data access (e.g., time delay for reporting
         transient sources and the software contribution to measurement errors).
\end{enumerate}

We present a detailed discussion of how these science-driven data properties are
transformed to system parameters below. 


\subsubsection{Probing Dark Energy and Dark Matter}
\label{sec:Dark_Energy}
 
Current models of cosmology require the existence of both dark matter and dark
energy to match observational constraints (Riess et al.~2007; Komatsu et al.~2009; 
Percival et al.~2010; LSST Dark Energy Science Collaboration 2012, and
references therein). Dark energy affects the cosmic history of both the Hubble expansion 
and mass clustering. Distinguishing competing models for the physical
nature of dark energy, or alternative explanations involving
modifications of the General Theory of Relativity, will require 
percent level measurements of both the cosmic expansion and the growth 
of dark matter structure as a function of redshift.  Any given
cosmological probe is sensitive to, and thus constrains degenerate
combinations of, several cosmological parameters.  Therefore the most robust
cosmological constraints are the result of using interlocking combinations
of probes. The most powerful probes include weak gravitational lens cosmic shear (WL), baryon 
acoustic oscillations (BAO), and photometry of type Ia supernovae (SN) -- all as functions of 
redshift. Using the cosmic microwave background fluctuations as the normalization, the 
combination of these probes can yield the needed precision to distinguish among models of dark 
energy (Zhan 2006, and references therein). In addition, strong galaxy 
and cluster lensing as a function of cosmic time probes the physics of dark matter,
because the positions and shapes of multiple images of a source galaxy 
depend sensitively on the total mass distribution, including the dark matter,
in the lensing object.

The three major programs from this science theme, WL, BAO and SN,  provide unique and 
independent constraints on the system design (SciBook Ch.~11--15). 

Weak lensing (WL) techniques can be used to map the distribution of
mass as a function of redshift and thereby trace the history of both
the expansion of the Universe and the growth of structure (e.g., Hu \&
Tegmark 1999; for reviews see Bartelmann \& Schneider 2001 and
Weinberg et al.~2013).  Measurements of cosmic shear as a function of
redshift allow determination of angular distances versus cosmic time,
providing multiple independent constraints on the nature of dark
energy.  These investigations require deep wide-area multi-color
imaging with stringent requirements on shear systematics in at least
two bands, and excellent photometry in at least five bands to measure
photometric redshifts (a requirement shared with BAO, and indeed all
extragalactic science drivers). The strongest constraints on the LSST
image quality arise from this science program. In order to control
systematic errors in shear measurement, the desired depth must be
achieved with many short exposures (allowing for systematics in the
measurement of galaxy shapes related
to the PSF and telescope pointing to be diagnosed and removed). Detailed simulations of
weak lensing techniques show that imaging over $\sim20,000$ deg$^2$ to
a 5$\sigma$ point-source depth of $r_{AB} \sim 27.5$ gives adequate
signal to measure shapes for of order 4 billion galaxies for weak
lensing.  These numbers are adequare to reach
Stage IV goals for dark energy, as defined by the Dark Energy Task
Force (Albrecht et al.~2006).   
%allow weak lensing 
%signal-to-noise ratio in the measurement of galaxy , in order to obtain a sample of
%$\sim$4 billion lensing galaxies, the coadded map must cover
%$\sim$20,000 deg$^2$, and reach a depth of $r\sim27.5$ (5$\sigma$ for
%point sources; AB magnitudes hereafter), with several hundred
%exposures per field and sufficient signal-to-noise ratio ($SNR$) in at
%least five other bands to obtain accurate photometric redshifts. 
This
depth, \B{and the corresponding deep surface brightness limit,}
optimize the number of galaxies with measured shapes in ground-based
seeing, and allow their detection in significant numbers to beyond a
redshift of two.  Analyzing these data will
require sophisticated data processing techniques.  For example, rather
than simply coadding all images in a given region of sky, the
individual exposures, each with their own PSF and noise
characteristics,  should be analyzed simultaneously to optimally
measure the shapes of galaxies (Tyson et al.~2008a;
Jee and Tyson, 2011).

Type Ia supernovae provided the first robust evidence that the expansion of the
Universe is accelerating (Riess et al.~1998; Perlmutter et al.~1999). To fully 
exploit the supernova science potential, light curves sampled in multiple
bands every few days over the course of a few months are required. This is
essential to search for systematic differences in supernova populations 
(e.g., due to differing progenitor channels) which 
may masquerade as cosmological effects, as well as to determine photometric 
redshifts from the supernovae themselves. Unlike other cosmological probes,
even a single object gives information on the relationship between
redshift and distance.  Thus a large 
number of SN across the sky allows one to search for any dependence 
of dark energy properties on direction, which 
would be an indicator of new physics. The results from this method can be compared 
with similar measures of anisotropy from the combination of WL and BAO
(Zhan et al.~2009).  
Given the expected SN flux distribution
at the redshifts where dark energy is important, the 
single visit depth should be at least $r\sim24$. Good image quality is
required to separate SN photometrically from 
their host galaxies. Observations in at least five photometric bands will allow
proper K-corrected light curves to be measured over a range of
redshift.  Carrying out these K-corrections requires that the 
calibration of the relative offsets in photometric zero points between filters and
the system response functions, especially near the edges of
bandpasses, must be accurate to about 1\% (Wood-Vasey et al.~2007),
similar to the requirements from photometric redshifts of galaxies. Deeper data 
($r>26$) for small areas of the sky can extend the discovery of SN to a mean 
redshift of 0.7 (from $\sim0.5$ for the main survey), with some objects beyond $z\sim$1
(Garnavich et al.~2005; Pinto et al.~2005; SciBook Ch.~11). The added statistical leverage 
on the ``pre-acceleration'' era ($z\ga1$) would improve constraints on the properties of 
dark energy as a function of redshift.

Finally, there will be powerful cross checks and complementarities with other planned or 
proposed surveys, such as Euclid (Laureijs et al.~2011) and WFIRST-AFTA
(Spergel et al.~2013)
%\footnote{http://wfirst.gsfc.nasa.gov}
(wide-field optical-IR imaging from space),
DESI (Levi et al.~2013)
%\footnote{http://desi.lbl.gov} 
and PFS (Takada et al.~2014)
(spectroscopic BAO with several 
millions of galaxies), and SKA\footnote{https://www.skatelescope.org} (radio).
Large survey volumes are key to probing dynamical dark energy models (with sub-horizon 
dark energy clustering or anisotropic stresses). The cross-correlation
of the three-dimensional
mass distribution (as probed by HI in SKA, or galaxies in DESI and PFS) with the gravitational growth 
probed by tomographic shear in LSST will be complementary way to constrain dark energy 
properties beyond simply characterizing its equation of state. 



\subsubsection{Taking an Inventory of the Solar System}


The small-body populations in the Solar System, such as asteroids, trans-Neptunian objects (TNOs) 
and comets, are remnants of its early assembly. The history of accretion, collisional grinding, and 
perturbation by existing and vanished giant planets is preserved in the orbital elements and size 
distributions of those objects. Cataloging the orbital parameters, size distributions, colors and light 
curves of these small-body populations requires a large number of observations in multiple filters, 
and will lead to insights into planetary formation and evolution by providing the basis and constraints 
for new theoretical models. In addition, collisions in the main asteroid belt between Mars and Jupiter 
still occur, and occasionally eject objects on orbits that may place them on a collision course with Earth. 
Studying the properties of main belt asteroids at sub-kilometer sizes is important for linking the near-Earth 
Object (NEO) population with its source in the main belt. About 20\% of NEOs, the potentially hazardous 
asteroids (PHAs), are in orbits that pass sufficiently close to Earth's orbit, to within 0.05 AU, that perturbations 
with time scales of a century can lead to intersections and the possibility of collision. In December 2005, 
the U.S. Congress directed\footnote{For details see http://neo.jpl.nasa.gov/neo/report2007.html} NASA to 
implement a NEO survey that would catalog 90\% of NEOs with diameters larger than 140 meters by 2020.  

Discovering and linking objects in the Solar System moving with a wide range of apparent velocities (from 
several degrees per day for NEOs to a few arc seconds per day for the most distant TNOs) places strong 
constraints on the cadence of observations, requiring closely spaced pairs of observations (two or preferably 
three times per lunation) in order to link detections unambiguously and derive orbits (SciBook Ch.~5). Individual 
exposures should be shorter than about 30 seconds each to minimize the effects of trailing for the majority of 
moving objects. The images must be well sampled to enable accurate astrometry, with absolute accuracy of at 
least 0.1 arcsec in order to measure orbital parameters of TNOs with enough precision to constrain theoretical 
models and enable prediction of occulations. The photometry should be
better than 1-2\% to measure asteroids' colors and thus determine
their types.  The different filters
should be observed over a short time span to reduce apparent 
variations in color due to changes in observing geometry, but should
be repeated over many lunations in order to determine phase curves and allow shape modeling. 

The Congressional mandate can be fulfilled with a 10-meter-class
telescope equipped with a multi-gigapixel camera, and a sophisticated  
and robust data processing system (Ivezi\'{c} et al.~2007a). The images should reach a depth of at 
least 24.5 (5$\sigma$ for point sources) in the $r$ band to reach high
completeness down to the 140 m mandate for NEOs.  Such an instrument
would probe the $\sim$100 m size range at main-belt distances, and
discover rare distant TNOs such as Sedna (Brown, Trujillo \&
Rabinowitz 2004) and 2012 VP113 (Trujillo \& Sheppard 2014). 


\subsubsection{ Exploring the Transient Optical Sky}

Recent surveys have shown the power of measuring variability for studying gravitational lensing, searching 
for supernovae, determining the physical properties of gamma-ray burst sources, probing the structure of 
active galactic nuclei, studying variable star populations, discovering exoplanets 
and many other subjects at the forefront of astrophysics (SciBook
Ch.~8; Law et al.~2009; Djorgovski et al.~2012; Rowe et al.~2014). Wide area, dense
temporal coverage to deep limiting magnitudes would enable the
discovery and  
analysis of rare and exotic objects such as neutron stars and black hole binaries, novae and stellar flares, 
gamma-ray bursts and X-ray flashes, active galactic nuclei, stellar disruptions by black holes (Bloom et al.~2011;
Gezari et al.~2012), and possibly new classes of transients, such as
binary mergers of black holes (Shields \& 
Bonning 2008; Thornton et al.~2013). Such a survey would likely detect numerous microlensing events in the 
Local Group and perhaps beyond, and open the possibility of discovering populations of binaries and planets 
via transits (e.g., Beaulieu et al.~2006; Drake et al.~2010; Choi et al.~2013; Batista et al.~2014), as well as obtaining 
spectra of lensed stars in distant galaxies.

Time-domain science requires large area coverage to enhance the probability of detecting rare events; good 
time sampling, because light curves are necessary to distinguish certain types of variables and in some cases 
to infer their properties (e.g. determining the intrinsic luminosity of Type Ia supernovae depends on measurements 
of their rate of decline); accurate color information to assist with the classification of variable objects; good 
image quality to enable differencing of images, especially in crowded fields; and rapid data reduction, classification 
and reporting to the community in order to flag interesting objects
for spectroscopic observations and follow-up in other wavebands.  

Time series ranging between one minute and ten years cadence should be probed over a significant fraction of 
the sky. Rapid cadence makes it possible to serendipitously catch very short-lived events, such as eclipses in 
ultra-compact double degenerate binary systems (Anderson et al.~2005), to constrain the properties of fast faint 
transients (such as optical flashes associated with gamma-ray bursts;
Bloom et al.~2008), to detect electromagnetic counterparts to gravitational waves (Nissanke, Kasliwal \& Georgieva 2013) and to further constrain the properties 
of new classes of transients discovered by programs such as the Deep
Lens Survey (Becker et al.~2004), the Catalina Real-time Transient Survey
(Drake et  
al.~2009) and the Palomar Transient Factory (Law et
al.~2009). Observations over a decade would enable the study of long period 
variables and quasars (Kaspi et al.~2007; MacLeod et al.~2010; Graham
et al.~2014).  The next frontier in this field will require measuring
the colors of fast transients, and to probe variability at faint
magnitudes.  Classification of transients in close to real time will
require access to the full photometric history of the objects, both
before and after the transient event (e.g., Mahabal et al.~2011). 


%\vskip 0.5in
\subsubsection{ Mapping the Milky Way}

A major challenge in extragalactic cosmology today concerns the formation of structure on sub-galactic scales, where 
baryon physics becomes important, and the nature of dark matter may manifest itself in observable ways (e.g., Weinberg 
et al.~2013). The Milky Way and its environment provide a unique dataset for understanding the detailed processes that 
shape galaxy formation and for testing the small-scale predictions of
our standard cosmological model. However, we still lack 
answers to two basic questions about our Galaxy:
\begin{itemize}
\item What is the detailed structure and accretion history of the Milky Way?
\item What are the fundamental properties of all the stars within 300 pc of the Sun?
\end{itemize}

Mapping the Galaxy requires large area coverage, excellent image
quality to maximize photometric and astrometric accuracy,
especially in crowded fields, photometric precision of at least 1\% to
separate main sequence and 
giant stars (e.g., Helmi et al.~2003) as well as to identify variable
stars such as RR Lyrae (Sesar et al.~2010, Sharma et al.~2011),
and astrometric precision of about 10 mas per observation to enable parallax and proper motion measurements 
(SciBook Ch.~6,7). In order to probe the halo out to its presumed edge at $\sim100$ kpc (Ivezi\'{c} et al.~2003) using
main-sequence stars, the total coadded depth must reach $r > 27$, with a similar depth in the $g$ band.   
The metallicity distribution of stars can be studied photometrically in the Sgr tidal stream (e.g., see 
Majewski et al.~2003, Chou et al.~2007) and other halo substructures
($\sim 30$ kpc, Carollo et al.~2007), yielding new insights into how
they formed.  These 
metallicities are limited by the coadded depth in the $u$ band.  To reach the limit of the stellar halo, this must reach 
$u\sim24.5$. To detect RR Lyrae stars beyond the Galaxy's tidal radius at $\sim 300$ kpc, the single-visit depth must 
be $r \sim  24.5$.

In order to measure the tangential velocity of stars at a distance of 10 kpc, where the halo dominates over the disk, to 
within 10 km s$^{-1}$ (comparable with the accuracy of
large-scale radial velocity surveys), the proper motion  
accuracy should be 0.2 mas yr$^{-1}$ or better. This is the same accuracy as will be delivered by the recently launched Gaia mission\footnote{http://sci.esa.int/gaia/} (Perryman et al.~2001; de Bruijne 2012) at its faint limit ($r \sim 20$). 
In order to measure distances to solar neighborhood stars out to a distance of 300 pc (the thin disk scale height), 
with geometric distance accuracy of at least 30\%, trigonometric parallax measurements accurate to 1 mas ($1\sigma$) 
are required over 10 years. To achieve the required proper motion and parallax accuracy with an assumed astrometric 
accuracy of 10 mas per observation per coordinate, approximately 1,000
separate observations are required. This requirement for a large
number of observations is in good agreement with that from minimizing
systematics in weak lensing observations (\S~\ref{sec:Dark_Energy}). 


%\subsubsection{A Summary and Synthesis of Science-driven Constraints on Data Properties}

\begin{figure}
\includegraphics[width=1.0\hsize,clip]{seeing2.pdf}
\caption{
The image quality distribution measured at the Cerro Pach\'{o}n site using 
a differential image motion monitor (DIMM) at $\lambda$ = 500 nm, and corrected 
using an outer scale parameter of 30 m over an 8.4 m aperture. For details 
about the outer scale correction see Tokovinin (2002). The observed distribution 
is well described by a log-normal distribution, with the parameters shown in 
the figure.} 
\label{Fig:seeing}
\end{figure}

\subsubsection{A Summary and Synthesis of Science-driven Constraints on Data Properties}

The goals of all the science programs discussed above 
(and many more, of course) can be accomplished by satisfying the 
minimal constraints listed below. For a more elaborate listing
of various constraints, including detailed specification of 
various probability density distribution functions, please see the LSST Science
Requirements Document\footnote{http://www.lsst.org/files/docs/SRD.pdf}
and the LSST Science Book. 

\begin{enumerate}
\item  {\it The single visit depth} should reach $r\sim24.5$. This limit is
   primarily driven by the search for NEOs, variable sources (e.g., SN,
   RR Lyrae stars), and by proper motion and trigonometric parallax 
   measurements for stars. Indirectly, it is also driven by the
   requirements on the coadded survey depth and the minimum number of 
   exposures required by WL science.  We will split a single visit
   into two exposures of equal length to identify and remove cosmic
   rays.  
\item  {\it Image quality} should maintain the limit set by the 
     atmosphere (the median free-air seeing is 0.65 arcsec in the $r$ band 
     at the chosen site, see Fig.~\ref{Fig:seeing}),
     and not be degraded appreciably by the hardware. In addition to stringent 
     constraints from weak lensing, good image quality is driven by the 
     required survey depth for point sources and by image differencing
     techniques. 
\item  {\it Photometric repeatability} should achieve 5 mmag precision
     at the bright end, with zeropoint stability across the sky of 10 mmag
     and band-to-band calibration errors not larger than 5 mmag.
     These requirements are driven by the need for high photometric redshift accuracy,
     the separation of stellar populations, detection of low-amplitude variable
     objects (such as eclipsing planetary systems), and the search for
     systematic effects in type Ia supernova light curves.
\item  {\it Astrometric precision} should maintain the limit set by 
     the atmosphere, of about 10 mas per visit at the bright end
     (on scales below 20 arcmin). This precision is driven by the desire to 
     achieve a proper motion accuracy of 0.2 mas yr$^{-1}$ and parallax accuracy of 
     1.0 mas over the course of a 10-year survey (see \S \ref{sec:astrom}).
\item  {\it The single visit exposure time} (including both exposures in a 
    visit, which are required for cosmic ray rejection) should be less than about a minute 
    to prevent trailing of fast moving objects and to aid control 
    of various systematic effects induced by the atmosphere. It should
    be longer than $\sim$20 seconds to avoid significant efficiency losses due to
    finite readout, slew time, and read noise.
\item  {\it The filter complement} should include at least six filters
    in the wavelength range limited by atmospheric absorption and 
    silicon detection efficiency (320--1050 nm), with roughly
    rectangular filters and no large gaps in the coverage, in order
    to enable robust and accurate photometric redshifts and stellar typing. An 
    SDSS-like $u$ band (Fukugita et al.~1996) is extremely important for separating 
    low-redshift quasars from hot stars, and for estimating the metallicities of 
    F/G main sequence stars. A bandpass with an effective wavelength of
    about 1 micron  would enable studies of sub-stellar objects, high-redshift 
    quasars (to redshifts of $\sim$7.5), and regions of the Galaxy that are obscured 
    by interstellar dust.
\item  {\it The revisit time distribution} should enable determination of
   orbits of Solar System objects and sample SN light curves every few days, 
   while accommodating constraints set by proper motion and trigonometric 
   parallax measurements.
\item  {\it The total number of visits} of any given area of sky, when accounting for all 
   filters, should be of the order of 1,000, as mandated by WL
   science, the search for NEOs, and proper motion and 
   trigonometric parallax measurements. Studies of transient sources 
   also benefit from a large number of visits.
\item  {\it The coadded survey depth} should reach 
    $r\sim27.5$, with sufficient signal-to-noise ratio in other bands
    to address both extragalactic and Galactic science drivers.
\item  {\it The distribution of visits per filter} should enable 
   accurate photometric redshifts, separation of stellar populations,
   and sufficient depth to enable detection of faint extremely red
   sources (e.g., brown dwarfs and high-redshift quasars). Detailed simulations of 
   photometric redshift estimates
   suggest an approximately flat distribution of visits among bandpasses
   (because the system throughput and atmospheric properties are 
    wavelength dependent, the achieved depths are different in different
    bands). The adopted time allocation
   (see Table 1) includes a slight preference to the $r$ and $i$ bands because of their
   dominant role in star/galaxy separation and weak lensing measurements. 
\item  {\it The distribution of visits on the sky} should extend over
   at least $\sim$20,000 deg$^2$ to obtain the required number of galaxies
   for WL studies, with attention paid to include ``special''
   regions such as the Ecliptic and Galactic planes, and the Large and Small
   Magellanic Clouds (if in the Southern Hemisphere).  Note that
   20,000 deg$^2$ is comparable to the full area observable from any
   given ground-based site at reasonably low airmass (the LSST survey area will 
   be contained within 30,000 deg$^2$ with $\delta<+34.5^\circ$).
\item  {\it Data processing, data products and data access} should
  result in data products that approach the statistical uncertainties
  in the raw data; i.e., the processing must be close to optimal.  
%enable efficient science analysis without a significant impact on the
%final uncertainties. 
To enable fast and efficient response to
   transient sources, the processing latency for variable sources should be less than a minute,
   with a robust and accurate preliminary characterization
   %classification 
   of all reported variables.
\end{enumerate}

Remarkably, even with these joint requirements, none of the 
individual science programs is severely over-designed, i.e., despite 
their significant scientific diversity, these programs are highly 
compatible in terms of desired data characteristics. Indeed, any one
of the four main science drivers could be removed, and the remaining 
three would still yield very similar requirements for most system 
parameters. As a result, the LSST system can adopt a highly 
efficient survey strategy where {\it a single dataset serves most science
programs} (instead of science-specific surveys executed in series). 
One can view this project as {\it massively parallel astrophysics}.
The vast majority (about 90\%) of the observing time will be devoted to 
a deep-wide-fast survey mode, with the remaining 10\% 
allocated to special programs which will also address multiple science 
goals. Before describing these surveys in detail, we discuss the main 
system parameters. 


\begin{table}
\caption{The LSST Baseline Design and Survey Parameters}
\begin{tabular}{|l|l|}
\hline  
   Quantity                         &     Baseline Design Specification    \\
\hline  
Optical Config.                           &  3-mirror modified Paul-Baker        \\
Mount Config.                            &  Alt-azimuth          \\
Final f-ratio, aperture                 &  f/1.234, 8.4 m                \\
Field of view, \'etendue              &  9.6 deg$^2$,   319 m$^2$deg$^2$     \\
Plate Scale                                  &  50.9 $\mu$m/arcsec (0.2'' pix)  \\
Pixel count                                  &  3.2 Gigapix  \\
Wavelength Coverage                   &  320 -- 1050 nm, $ugrizy$             \\
Single visit depths, design $^a$  &  23.9, 25.0, 24.7, 24.0, 23.3, 22.1    \\
Single visit depths, min.$^b$       &  23.4, 24.6, 24.3, 23.6, 22.9, 21.7    \\
Mean number of visits$^c$          &  56, 80, 184, 184, 160, 160               \\ 
Final (coadded) depths$^d$         &  26.1, 27.4, 27.5, 26.8, 26.1, 24.9     \\
\hline                         
\end{tabular}
\\ \vskip 0.05in
$^a$ Design specification from the Science Requirements Document (SRD) for 5$\sigma$ depths 
for point sources in the $ugrizy$ bands, respectively. The listed values are expressed on AB magnitude 
scale, and correspond to point sources and fiducial zenith observations (about 0.2 mag loss of depth 
is expected for realistic airmass distributions, see Table 2 for more details).
$^b$ Minimum specification from the Science Requirements Document for 5$\sigma$ depths.
$^c$ An illustration of the distribution of the number of visits as a function of bandpass, 
taken from Table 24 in the SRD.  
$^d$ Idealized depth of coadded images, based on design specification for 5$\sigma$ depth and 
the number of visits in the penultimate row (taken from Table 24 in the SRD). 
\vskip 0.2in          
\end{table}



\subsection{The Main System Design Parameters} 

Given the minimum science-driven constraints on the data properties listed 
in the previous section, we now discuss how they are translated into
constraints on the main system design parameters: the aperture size, 
the survey lifetime, the optimal exposure time, and the filter complement. 


\subsubsection{ The Aperture Size }
\label{Sec:apSize}
The product of the system's \'etendue and the survey lifetime, for given
observing conditions, determines
the sky area that can be surveyed to a given depth. 
%, where the \'etendue is the product of the primary mirror area and
%the field-of-view area. 
The 
LSST field-of-view area is maximized to its practical limit, $\sim$10 deg$^2$, 
determined by the requirement that the delivered image quality be dominated 
by atmospheric seeing at the chosen site (Cerro Pach\'{o}n in Northern Chile). 
A larger field-of-view would lead to unacceptable deterioration of the 
image quality. This constraint leaves the primary mirror diameter and survey lifetime 
as free parameters. The adopted survey lifetime of 10 years is a compromise 
between a shorter time that leads to an excessively large and expensive mirror (15 m for a 
3 year survey and 12 m for a 5 year survey) and not as effective proper motion
measurements, and a smaller telescope that would require more time to complete the 
survey, with the associated increase in operations cost.

The primary mirror size is a function of the required survey depth and the 
desired sky coverage. By and large, the anticipated science outcome scales 
with the number of detected sources. For practically all astronomical source 
populations, in order to maximize the number of detected sources, it is more 
advantageous to maximize the area first, and then 
the detection depth\footnote{ 
If the total exposure time is doubled and used to double the survey area, 
the number of sources increases by a factor of two. If the survey 
area is kept fixed, the increased exposure time will result in 
$\sim$0.4 mag deeper data (see eq.~\ref{m5}). For cumulative source 
counts described by $\log(N) = C + k*m$, the number of sources
will increase by more than a factor of two only if $k>0.75$. 
Apart from $z<2$ quasars, practically all populations 
have $k$ at most 0.6 (the Euclidean value), and faint stars 
and galaxies have $k<0.5$. For more details, please see Nemiroff
(2003).}. For this reason, the sky area for the main survey is 
maximized to its practical limit, 18,000 deg$^2$, determined by the 
requirement to avoid airmasses less than 1.5, 
%$X<1.5$, where approximately $X={\rm sec}(\theta)$  and $\theta$ is
%the zenith distance), 
which would substantially 
deteriorate the image quality and the survey depth (see eq.~\ref{m5}). 

With the adopted field-of-view area, the sky coverage and the survey lifetime 
fixed, the primary mirror diameter is fully driven by the required survey 
depth. There are two depth requirements: the final (coadded) survey depth, 
$r\sim27.5$, and the depth of a single visit, $r\sim24.5$. The two 
requirements are compatible if the number of visits is several hundred
per band, which is in good agreement with independent science-driven 
requirements on the latter.  

The required coadded survey depth provides a direct constraint, 
independent of the details of survey execution such as the exposure time per visit, 
on the minimum effective primary mirror diameter of 6.5 m, as illustrated in 
Fig.~\ref{Fig:coaddDepth}. 



\subsubsection{ The Optimal Exposure Time }

The single visit depth depends on both the primary mirror diameter and the
chosen exposure time, $t_{\rm vis}$. In turn, the exposure time 
determines the time interval to revisit a given sky position and the total 
number of visits, and each of these quantities has its own science 
drivers. We summarize these simultaneous constraints in terms of the 
single-visit exposure time:
\begin{itemize}
\item  The single-visit exposure time should not be longer than about a minute to 
         prevent trailing of fast Solar System moving objects, and to enable efficient
         control of atmospheric systematics.
\item  The mean revisit time (assuming uniform cadence) for a given position 
         on the sky, $n$, scales as 
\begin{equation}
  n = \left( {t_{\rm vis} \over 10  \, {\rm sec}} \right)
      \left( { A_{\rm sky} \over 10,000  \, {\rm deg}^2} \right)
      \left( {10 \, {\rm deg}^2 \over  A_{\rm FOV}} \right) {\rm days},
\end{equation}
where two visits per night are assumed (required for efficient detection of 
Solar System objects, see below), and the losses for realistic observing conditions 
have been taken into account (with the aid of the Operations Simulator described below). 
Science drivers such as supernova light curves and moving objects in the Solar System require 
that $n<4$ days, or equivalently $t_{vis} < 40$ seconds for the nominal values 
of $A_{sky} $ and $A_{FOV}$. 
\item  The number of visits to a given position on the sky, $N_{visit}$,
%again 
with losses for realistic observing conditions taken into account,
is given by
\begin{equation}
      N_{visit} = \left( {3000 \over n} \right)
                    \left( { T \over 10 \, {\rm yr}} \right).
\end{equation}
The requirement $N_{visit}>800$ again implies that $n<4$ and 
$t_{vis} < 40$ seconds if the survey lifetime, $T \sim 10$ years. 
\item  These three requirements place a firm upper limit on the 
optimal visit exposure time of $t_{vis} < 40$ seconds. Surveying
efficiency (the ratio of open-shutter time to the total
time spent per visit) considerations place a lower limit on 
$t_{vis}$ due to finite detector read-out and telescope slew time (the longest 
acceptable read-out time is set to 2 seconds, the shutter open-and-close
time is 2 seconds, and the slew and settle time is set to 5 seconds, including 
the read-out time for the second exposure in a visit):
\begin{equation}
      \epsilon = \left( {t_{vis} \over t_{vis} + 9 \, {\rm sec}}\right).
\end{equation}
To maintain efficiency losses below $\sim$30\% (i.e., at least below the 
limit set by the weather patterns), and to minimize the read noise
impact, $t_{vis} > 20$ seconds is required. 
\end{itemize}

Taking these constraints simultaneously into account, as summarized in 
Fig.~\ref{Fig:singleDepth}, 
yielded the following reference design:
\begin{enumerate}
\item A primary mirror effective diameter of $\sim$6.5 m. With the adopted optical 
design, described below, this effective diameter corresponds to a geometrical diameter 
of $\sim$8 m. Motivated by characteristics of the existing equipment at the
Steward Mirror Laboratory, which is fabricating the primary mirror, the adopted
geometrical diameter is set to 8.4 m. 
\item A visit exposure time of 30 seconds (using two 15 second exposures
to efficiently reject cosmic rays), yielding $\epsilon=77$\%.
\item A revisit time of 3 days on average for 10,000 deg$^2$ of sky, 
  with two visits per night.
\end{enumerate}

\begin{figure}[t]
\vskip -0.5in
\includegraphics[width=1.1\hsize,clip]{coaddedDepth.pdf}
\vskip -1.1in
\caption{The coadded depth in the $r$ band (AB magnitudes) vs. the effective aperture and 
the survey lifetime. It is assumed that 22\% of the total observing time (corrected for
weather and other losses) is allocated for the $r$ band, and that the ratio of 
the surveyed sky area to the field-of-view area is 2,000.} 
\label{Fig:coaddDepth}
\end{figure}

\begin{figure}[t]
\includegraphics[width=1.0\hsize,clip]{singleDepth.pdf}
\caption{The single-visit depth in the $r$ band (5$\sigma$ detection for 
point sources, AB magnitudes) vs. revisit time, $n$ (days), as a function of 
the effective aperture size. With a coverage of 10,000 deg$^2$ in two bands, 
the revisit time directly constrains the visit exposure time, $t_{vis}=10\,n$ 
seconds. In addition to direct constraints on optimal exposure time, $t_{vis}$ 
is also driven by requirements on the revisit time, $n$, the total number of visits 
per sky position over the survey lifetime, $N_{visit}$, and the survey efficiency,
$\epsilon$ (see eqs.1-3). Note that these constraints result in a fairly narrow range of 
allowed $t_{vis}$ for the main deep-wide-fast survey.} 
\label{Fig:singleDepth}
\end{figure}

To summarize, the chosen primary mirror diameter is the {\it minimum}
diameter that simultaneously satisfies the depth ($r\sim24.5$ for single visit and 
$r\sim27.5$ for coadded depth) and cadence (revisit time of 3-4 days, 
with 30 seconds per visit) constraints described above.

\subsection{System Design Trade-offs}

We note that the Pan-STARRS project (Kaiser et al.~2002), with similar science
goals as LSST, has adopted a distributed aperture design, where the total
system \'etendue is 
a sum of \'etendue values for an array of small telescopes (the prototype
PS1 telescope has an \'etendue 1/24$^{th}$ that of LSST). 
Similarly, the LSST system could perhaps be made as two smaller copies with 
6m mirrors, or 4 copies with 4m mirrors, or 16 copies with 2m mirrors. Each 
of these clones would have to have its own 3 Gigapixel camera (see below), and 
given the added risk and complexity (e.g., maintenance, data processing), the monolithic 
design seems advantageous for a system with such a large \'etendue as LSST. 

It is informative to consider the tradeoffs that would be required 
for a system with a smaller aperture, if the science requirements were
to be maintained. For this comparison, we consider a four-telescope version of
the Pan-STARRS survey (PS4). With an \'etendue about 6 times smaller 
than that of LSST (effective diameters of 6.5 m and 3.0 m, and a field-of-view area 
of 9.6 deg$^2$ vs. 7.2 deg$^2$), and all observing conditions being equal,
the PS4 system could in principle use an identical cadence as that of LSST. The
main difference in the datasets would be a faint limit shallower by about 
1 mag in a given survey lifetime. As a result, for Euclidean populations the 
sample sizes would go down by a factor of 4, while for shallower populations (e.g.,
galaxies around redshift of 1) the samples would be smaller by a factor 2-3. 
The distance limits for nearby sources, such as Milky Way stars, would drop to
60\% of their corresponding LSST values, and the NEO completeness level mandated by
the U.S. Congress would not be reached.  

If instead the survey coadded depth were to be maintained, then the survey sky 
area would have to be 6 times smaller ($\sim$3,500 deg$^2$). If the
survey single-visit depth were to be maintained, then the exposure
time would have to be about 6 times longer (ignoring the slight difference
in the field-of-view area and simply scaling by the \'etendue ratio), 
resulting in non-negligible trailing losses for Solar System objects, 
and either 
i) a factor of six smaller sky area observed within $n=3$ days, or 
ii) the same sky area revisited every $n=18$ days. 
Given these conflicts, one solution would be to split the observing time and 
allocate it to individual specialized programs (e.g., large sky area vs. 
deep coadded data vs. deep single-visit data vs. small $n$ data, etc.), 
as is being done by the PS1 Consortium\footnote{More information about 
Pan-STARRS is available from http://pan-starrs.ifa.hawaii.edu.}. 

In summary,
{\it given the science requirements as stated here, there is a
minimum \'etendue of $\sim$300 deg$^2$m$^2$ which enables our seemingly
disparate science goals to be addressed with a single data set.}
A system with a smaller \'etendue would require separate specialized surveys
to address the science goals, which results in a loss of surveying
efficiency\footnote{The converse is also true: for every \'etendue
there is a set of optimal science goals that such a system can 
address with a high efficiency.}. The LSST is designed to reach this 
minimum \'etendue for the science goals stated in its Science Requirements 
Document. 



\subsection{  The Filter Complement }

The LSST filter complement ($ugrizy$, see Fig.~\ref{Fig:filters}) is modeled after the Sloan
Digital Sky Survey 
(SDSS) system (Fukugita et al.~1996) because of its demonstrated success in a wide
variety of applications, including photometric redshifts of galaxies (Budav\'{a}ri
et al.~2003), separation of stellar populations (Lenz et al.~1998; Helmi et al.~2003), 
and photometric selection of quasars (Richards et al.~2002; Ross et al.~2012). The extension of the 
SDSS system to longer wavelengths
(the $y$ band at $\sim$1 micron) is driven by the increased effective redshift 
range achievable with the LSST due to deeper imaging, the desire to study sub-stellar 
objects, high-redshift quasars, and regions of the Galaxy that are obscured by
interstellar dust, and
the scientific opportunity enabled by modern CCDs with high quantum efficiency 
in the near infrared.  


\begin{figure}
\hskip -0.13in
\includegraphics[width=1.1\hsize,clip]{filters_y4.pdf}
\caption{The LSST bandpasses. The vertical axis shows the total throughput. The computation 
includes the atmospheric transmission (assuming an airmass of 1.2, % at an altitude of $\sim$56 deg.,
dotted line), optics, and the detector sensitivity.} 
\label{Fig:filters}
\end{figure}


The chosen filter complement corresponds to a design ``sweet spot''. We have 
investigated the possibility of replacing the $ugrizy$ system with a
filter complement that includes only five filters. For example, each filter
width could be increased by 20\% over the same wavelength range (neither a
shorter wavelength range, nor gaps in the wavelength coverage are desirable 
options), but this option is not satisfactory. Placing the red edge of the $u$ 
band blueward of the Balmer break allows optimal separation of stars and
quasars, and the telluric water absorption feature at 9500\AA\
effectively defines the blue edge of the $y$ band. Of the remaining four
filters ($griz$), the $g$ band is already quite wide. As a last option, the 
$riz$ bands could be redesigned as two wider bands. However, this option is also 
undesirable because the $r$ and $i$ bands are the primary bands for weak
lensing studies and for star/galaxy separation, and chromatic atmospheric
refraction would worsen the point spread function for a wider bandpass. 


\begin{figure}
\hskip -0.12in
\includegraphics[width=1.1\hsize,clip]{modtran1.pdf}
\caption{An example of determination of the atmospheric opacity by 
simultaneously fitting a three-parameter stellar model SED (Kurucz 1979) and 
six physical parameters of a sophisticated atmospheric model (MODTRAN, Anderson 
et al.~1999) to an observed F-type stellar spectrum ($F_\lambda$). The black 
line is the observed spectrum and the red line is the best fit. Note that the 
atmospheric water feature around 0.9-1.0 $\mu$m is exquisitely well fit. 
The components of the best-fit atmospheric opacity are shown in 
Fig.~\ref{Fig:modtran2}. Adapted from Burke et al.~(2007).}
\label{Fig:modtran1}
\end{figure}

\newpage
\subsection{ The Calibration Methods }



\begin{figure}
\includegraphics[width=1.0\hsize,clip]{modtran2.pdf}
\caption{The components of the best-fit atmospheric opacity used to
model the observed stellar spectrum shown in Fig.~\ref{Fig:modtran1}.
The atmosphere model (MODTRAN, Anderson et al.~1999) includes six
components: water vapor (blue), oxygen and other trace molecules
(green), ozone (red), Rayleigh scattering (cyan), a gray term
with a transmission of 0.989 (not shown) and an aerosol contribution
proportional to $\lambda^{-1}$ and extinction of 1.3\% at $\lambda$=0.675 \mic\
(not shown). The black line shows all six components combined.
Adapted from Burke et al.~(2007).}
\label{Fig:modtran2}
\end{figure}


Precise determination of the point spread function across each image, 
accurate photometric and astrometric calibration, and continuous monitoring 
of system performance and observing conditions will be needed to reach the 
full potential of the LSST mission. Extensive precursor data including the 
massive SDSS dataset and our own data obtained using telescopes close to 
the LSST site of Cerro Pach\'{o}n (e.g., the SOAR and Gemini South telescopes), 
as well as telescopes of similar aperture (e.g., Subaru), indicate that the 
photometric and astrometric accuracy will be limited not by our instrumentation 
or software, but rather by atmospheric effects. 
%Active optics will assure superb image quality. 

%The required 1\% photometric accuracy is driven by our requirements 
%on the photometric redshift accuracy, the separation of stellar populations, 
%the ability to detect low-amplitude variable objects, and the search for 
%systematic effects in type Ia supernova light curves. 
SDSS data and PS1 data 
taken in good photometric conditions have approached the LSST
requirement of 1\% photometric calibration
(Padmanabahn et al.~2008; Schlafly et al.~2012), although measurements with ground-based telescopes 
typically produce data with errors a factor of two or so larger. Analysis of
repeated SDSS scans obtained in varying observing conditions demonstrates that data
obtained in 
%traditionally 
non-photometric conditions can also be calibrated with
sufficient accuracy (Ivezi\'{c} et al.~2007b), as long as high-quality
photometric data also exist in the region. 

The LSST calibration plan builds on experience gained from the SDSS survey. 
The planned calibration process decouples the establishment of a stable and uniform internal 
relative calibration from the task of assigning absolute optical flux to 
celestial objects. 
%The latter effort requires determination of a small 
%number of factors appropriate for the accumulated multi-epoch data
%set. [I HAVE NO IDEA WHAT THIS SENTENCE ACTUALLY MEANS...]
Celestial sources will be used to define the internal photometric system and 
to monitor stability and uniformity of photometric data. There will be 
$>$100 main-sequence stars with $17<r<20$ per detector (14$\times$14 arcmin$^2$) 
even at high Galactic latitudes. Standardization of photometric scales will be 
achieved through direct observation of stars with well-understood spectral 
energy distributions (SEDs). 

While the primary source of data used for photometric calibration will be 
science images taken with the main telescope, these images alone will be 
insufficient to fully characterize instrumental and atmospheric temporal and 
spatial variations that can affect photometric measurements.  A
separate 1.5-meter calibration telescope will obtain spectra of
standard stars in LSST fields, calibrating the atmospheric throughput
as a function of wavelength  (Stubbs et al.~2007; 
see Figs.~\ref{Fig:modtran1} and \ref{Fig:modtran2}).  A tunable
monochromatic light source, calibrated with a photodiode, will measure
the relative throughput of the telescope plus camera system as a
function of wavelength (Stubbs \& Tonry 2006).  
%Auxiliary instrumentation, including a 1.5 m calibration telescope, will provide the 
%calibration parameters needed for image 
%processing, to calibrate the instrumental response of the LSST hardware
%(Stubbs \& Tonry 2006), and to measure the atmospheric optical depth as 
%a function of wavelength along the LSST line of sight (Stubbs et al.~2007; 
%see Figs.~\ref{Fig:modtran1} and \ref{Fig:modtran2}). 


\subsection{     The LSST  Reference Design    }

We briefly describe the reference design for the main LSST system components. 
Detailed discussion of the flow-down from science requirements to system 
design parameters, and extensive system engineering analysis can be 
found in the LSST Science Book (Ch.~2--3). 

\begin{figure}
\vskip -0.5in
\includegraphics[width=1.0\hsize,clip]{mirrors.pdf}
\vskip -0.5in
\caption{The LSST baseline optical design (modified three-mirror
  Paul-Baker) with its unique 
monolithic mirror: the primary and tertiary mirrors are positioned such 
that they form a continuous compound surface, allowing them to be polished 
from a single substrate.}
\label{Fig:optics}
\end{figure}


\begin{figure}
%\vskip -0.65in
\includegraphics[width=1.0\hsize,clip]{polishing.pdf}
%\vskip -0.65in
\caption{The polishing of the primary-tertiary mirror pair at the Richard F.Caris Mirror Lab at the University of Arizona Tucson. } 
\label{Fig:polishing}
\end{figure}




\begin{figure}
%\vskip -0.65in
\includegraphics[width=1.0\hsize,clip]{TMA_Image-Oct-2017.JPG}
%\vskip -0.65in
\caption{The baseline design for the 
LSST telescope.  The small focal ratio allows for a very squat
telescope, and thus a very stiff structure.  } 
\label{Fig:telescope}
\end{figure}



\subsubsection{ Telescope and Site}

The large LSST \'etendue is achieved in a novel three-mirror design (modified
Paul-Baker Mersenne-Schmidt system; Angel, Lesser \& Sarlot 2000) with a very fast $f$/1.234 beam. The optical 
design has been optimized to yield a large field of view (9.6 deg$^2$), 
with seeing-limited image quality, across a wide wavelength band (320--1050
nm). Incident light is collected by an annular primary mirror, having
an outer diameter of 8.4 m and inner diameter of 5.0 m, creating an effective filled aperture of 
$\sim$6.5 m in diameter. The collected light is reflected to a 3.4 m convex secondary, then onto
a 5 m concave tertiary, and finally  into the three refractive lenses of the camera (see Fig.~\ref{Fig:optics}).  
In broad terms, the primary-secondary mirror pair acts as a beam condenser, while the aspheric portion of 
the secondary and tertiary mirror acts as a Schmidt camera.  The 3-element refractive optics of the camera
correct for the chromatic aberrations induced by the necessity of a thick dewar window and flatten the
focal surface.  During design optimization, the primary and tertiary mirror surfaces were placed such that the primary's 
inner diameter coincides with the tertiary's outer diameter, thus making it possible to fabricate the mirror pair from a 
single monolithic blank using spin-cast borosilicate technology. The secondary mirror is fabricated from 
a thin 100 mm thick meniscus substrate, made from Corning's ultra-low expansion material. All 
three mirrors will be actively supported to control wavefront distortions 
introduced by gravity and environmental stresses on the telescope. 
The primary-tertiary mirror and the secondary mirror were 
cast\footnote{http://www.lsst.org/News/enews/m1m3-1004.html}  in 2008
and 2009.  The primary-tertiary mirror was polished by the Richard F.Caris Mirror Lab at the University of Arizona Tucson before being inspected and accepted by LSST in April 2015 \cite{Araujo16}. The  primary-tertiary mirror cell was fabricated by CAID in Tucson and is undergoing acceptance tests. The integration of the actuators and final tests with the mirror is scheduled later this fall and beginning of 2018. 
 

\begin{figure}
%\vskip -1.3in
\includegraphics[width=1.0\hsize,clip]{observatoryFull.pdf}
\includegraphics[width=1.0\hsize,clip]{ObservatoryFull_2017.pdf}
%\vskip -1.4in
\caption{Left: artist's rendering of the dome enclosure 
with the attached summit support building on Cerro Pach\'{o}n. The LSST calibration 
telescope is shown on an adjacent rise to the right.
Right: Status of the LSST Observatory as of Summer 2017. The main LSST telescope is on the right, waiting for the dome to be installed. The calibration telescope is on the left with its dome being installed.} 
\label{Fig:observatory}
\end{figure}

The LSST Observing Facility (Fig.~\ref{Fig:observatory}),
consisting of the telescope enclosure and summit support building, is being constructed atop Cerro Pach\'{o}n in northern Chile,
sharing the ridge with the Gemini South and SOAR telescopes\footnote{Coordinates listed in older versions 
of this paper were incorrect. We thank E. Mamajek for pointing this error to us.} 
% old, WRONG: (latitude: S 30$^\circ$ 10$'$ 20.1$"$; longitude: W 70$^\circ$ 48$'$ 0.1$"$; elevation: 2123 m; 
% NEW: from Mamajek (2012, arXiv:1210.1616, Table 3, GPS values): 
(latitude: S 30$^\circ$ 14$\arcmin$ 40.68$\arcsec$; longitude: W 70$^\circ$ 44$\arcmin$ 57.90$\arcsec$; elevation: 2652 m; 
Mamajek 2012).  The telescope enclosure houses a compact, stiff
telescope structure (see Fig.~\ref{Fig:telescope}) atop a 15 m high concrete pier
with a fundamental frequency of 8 Hz, that is crucial for achieving the required fast slew-and-settle times.  The height of the pier was set to place the telescope above the degrading 
effects of the turbulent ground layer.  Capping the telescope
enclosure is a 30 m diameter dome with extensive ventilation to reduce
dome seeing 
%(local air turbulence that can distort images)  
and to maintain a uniform thermal environment over the course of the night.  Furthermore, the summit support 
building has been oriented with respect to the prevailing winds to shed its turbulence away from the
telescope enclosure.  The summit support building includes a coating chamber for recoating the three LSST mirrors and 
clean room facilities for maintaining and servicing the camera.


\vskip 0.2in
\subsubsection{ Camera }


The LSST camera provides a 3.2 Gigapixel flat focal plane array, tiled by 189
4K$\times$4K CCD science sensors with 10 $\mu$m pixels (see Figs.~\ref{Fig:camera}
and \ref{Fig:fov}). This pixel count is a direct consequence of sampling the 
9.6 deg$^2$ field-of-view (0.64m diameter) with 0.2$\times$0.2 arcsec$^2$
pixels (Nyquist sampling in the best expected seeing of $\sim$0.4 arcsec). 
The sensors are deep depleted high resistivity silicon back-illuminated devices with 
a highly segmented architecture that enables the entire array to be read in 2 seconds. 
The detectors are grouped into 3$\times$3 rafts (see Fig.~\ref{Fig:raft}); each 
contains its own dedicated electronics. The rafts are mounted on a silicon carbide 
grid inside a vacuum cryostat, with an intricate thermal control system that maintains 
the CCDs at an operating temperature of 173 K. The entrance window to the
cryostat is the third (L3) of the three refractive lenses in the camera. The other
two lenses (L1 and L2) are mounted in an optics structure at the front of the camera 
body, which also contains a mechanical shutter, and a carousel assembly that holds 
five large optical filters. The sixth optical filter can 
replace any of the five via a procedure accomplished during daylight hours. 



\begin{figure}[t!]
%\hskip -1.7in
%\includegraphics[width=1.55\hsize,angle=90.0,clip]{camera2009.pdf}
%\vskip -2.5in
%\includegraphics[width=1.55\hsize,angle=90.0,clip]{camera2009.pdf}
\hskip 0.1in
\includegraphics[width=0.45\textwidth]{CameraFigureUpdated.pdf}
\caption{A cutaway view of LSST camera. The camera is positioned in the middle of the telescope.  The shutter (not shown) is positioned between the filter and lens L3.} 
\label{Fig:camera}
\end{figure}


\begin{figure}[ht]
\includegraphics[width=1.0\hsize,clip]{fov-updated-cropped.pdf}
\caption{The LSST focal plane. Each cyan square represents one
$4096\times4096$ pixel sensor. Nine sensors are assembled into a
raft; the 21 rafts are outlined in red. There are 189 science sensors, each 
with 16.8 megapixels, for a total of 3.2 gigapixels.} 
\label{Fig:fov}
\end{figure}

\begin{figure}[ht]
%\hskip -1.6in
%\includegraphics[width=1.5\hsize,angle=90.0,clip]{raft.pdf}
%\vskip -2.5in
\includegraphics[width=1.\hsize,clip]{raft-updated-cropped.pdf}
\caption{The LSST raft module with integrated front-end electronics
and thermal connections. Each raft (corresponding to the red squares
in Fig.~\ref{Fig:fov}) 
includes 9 sensors, and can be replaced.} 
\label{Fig:raft}
\end{figure}



\vskip 0.2in
\subsubsection{ Data Management }
\label{sec:dm}

The rapid cadence and scale of the LSST observing program will produce
approximately 15 TB per night of raw imaging data\footnote{For
  comparison, the volume of all imaging data collected over a decade
  and published in SDSS Data Release 7 (Abazajian et al.~2009) is approximately 16 TB.}. The large data volume, the real-time aspects, and the complexity of processing involved makes it impractical to defer the data reduction to the LSST end-users. Instead, the data collected by the LSST system will be automatically reduced to scientifically useful catalogs and images by the LSST Data Management (DM) system.
\\

The principal functions of the LSST Data Management system are to: 
\begin{itemize}
\item Process, in real time, the incoming stream of images generated by the camera system during observing by archiving raw images, generating alerts to new sources or sources whose properties have changed, and updating the relevant catalogs (``Level 1'' data products; \S~\ref{Sec:dp}).
\item Periodically process the accumulated survey data to provide a
  uniform photometric and astrometric calibration, measure the
  properties of all detected objects, and characterize objects based on their time-dependent behavior. The results of such a processing run form a {\em Data Release} (DR), which is a static, self-consistent data set suitable for use in performing scientific analyses of LSST data and publication of the results (the ``Level 2'' data products; \S~\ref{Sec:dp}). All data releases will be archived for the entire operational life of the LSST archive.
\item Facilitate the creation of added-value (``Level 3'';
  \S~\ref{Sec:dp}) data products, by providing suitable software,
  application programming interfaces (APIs),  
and computing infrastructure at the LSST data access centers.
\item Make all LSST data available through an interface that utilizes
community-based standards   to the maximum possible extent. Provide
  enough processing, storage, and network bandwidth to enable user
  analyses of the data without the need for petabyte-scale data
  transfers. 
\end{itemize}

Over the ten years of LSST operations and 11 data releases, this processing will result in a cumulative {\em processed} data size 
approaching 500 petabytes (PB) for imaging, and over 50 PB for the
catalog databases. The final data release catalog database alone is expected 
to be approximately 15 PB in size.
\\

\begin{figure}
%
% NOTE NOTE NOTE: The source of this figure is in DMsandwich.pptx.
% Edit that file and save it as PDF when an update is needed.
%
\hskip -0.2in
\includegraphics[width=1.1\hsize,clip]{DMsandwich.pdf}
\caption{The three-layered architecture of the data management system 
(application [red, top], middleware [purple, middle], and infrastructure [blue, bottom] layers) enables scalability, reliability, and evolutionary capability.} 
\label{Fig:DM1}
\end{figure}

The data management system is conceptually divided into three layers: an
infrastructure layer consisting of the computing, storage, and
networking hardware and system software; a middleware layer, which
handles distributed processing, data access, user interface, and
system operations services; and an applications layer, which includes
the data pipelines and products and the science data archives (see
Fig.~\ref{Fig:DM1}).

\begin{figure*}
%
% NOTE NOTE NOTE: The source of this figure is in DMX2.pptx
% Edit that file and save it as PDF when an update is needed.
%
\hskip 0.25in
\includegraphics[width=0.95\hsize,clip]{DMX2.pdf}
\caption{The LSST data flow from the mountain summit/base facility in
Chile to the data access center and archive centers in the U.S.}
\label{Fig:DM2}
\end{figure*}

Physically, the DM system components will span three key facilities on two 
continents: the Summit Facility at  Cerro Pach\'on (where the initial
detector cross-talk 
correction will be performed), the Base Facility in La Serena (which will serve
as a retransmission hub for data 
uploads to North America, as well as the data access center for the Chilean
community), and the central Archive Facility at the National Center
for Supercomputing Applications (NCSA) in Champaign, Illinois
All Level 1 (nightly) and Level 2 (data release) processing will take place at the
Archive Facility, which will also serve as a data access center
for the US community.

The data will be transported between the centers over existing high-speed optical fiber 
links from South America to the U.S. (see Fig.~\ref{Fig:DM2}).
Although the data processing center will have substantial computing 
power (e.g., the central facility will peak at $\sim 1.6$~petaflops of 
compute power), the continuation of current trends suggests that the center will 
not qualify for the top 500 list by the time of first light.
Hence, while LSST is making a novel use of advances in information technology, 
it is not taking the risk of pushing the expected technology to the limit, reducing
the overall risk to the project. 
\\

A novel aspect of the LSST DM system will be its ``Level 3"
capabilities, allowing the end-users to create, store, and share
custom data products not created by standard LSST processing. These
could be new catalogs created by simple post-processing of the LSST
data release catalogs, or entirely new data products generated by
running custom code on raw LSST data. The LSST software stack (described
below) will be made available to LSST end-users, as a basis on which
to quickly build such code. Approximately 10\% of the total
budget for the LSST Archive Center compute and storage capacity has
been reserved for end-user, Level 3, processing support
infrastructure\footnote{Furthermore, the data management system 
architecture will enable Level 3 tasks to "piggyback" onto annual 
Level 2 reprocessings, leveraging considerable I/O and
computing resources employed in the production of a data release.}. 

\subsubsection{The LSST software stack}
\label{sec:dmstack}

The {\em LSST Software Stack} is a well documented, state-of-the-art,
high-performance, scalable, multi-camera, open source, O/IR survey
data processing and analysis system, built to enable LSST survey data
reduction and the writing of custom, user-driven, code for Level 3
processing. It comprises 
all science pipelines needed to accomplish LSST data processing tasks
(e.g., calibration, single frame processing, coaddition, image
differencing, multi-epoch measurement, asteroid orbit determination,
etc.), the necessary data 
access and orchestration middleware, as well as the database and user
interface components. 

Algorithm development for the LSST software builds on the expertize
and experience of prior large astronomical surveys (including SDSS,
Pan-STARRS, DES,  
SuperMACHO, ESSENCE,  DLS, CFHTLS, and UKIDSS). The pipelines written 
for these surveys have demonstrated that it is possible to carry out
largely autonomous data 
reduction of large data sets, automated detection of sources and
objects, and the  
extraction of scientifically useful characteristics of those objects. 
While firmly footed in this prior history, the LSST software stack has 
largely been written anew, for reasons of performance, extendability, and
maintainability. All LSST codes have been designed and implemented 
following sofware engineering best practices, including modularity, clear definition
of interfaces, continuous integration, 
utilization of unit testing, a single set of documentation and coding
standards, and others. The primary implementation language is Python and, where
necessary for performance reasons, C++\footnote{All components implemented 
in C++ have been wrapped and exposed as Python modules to the rest of the system. Python truly is the ``native language of LSST''.}.

\begin{figure}
%
% NOTE NOTE NOTE: The source of this figure is in DMStripe82.pptx
%
\includegraphics[width=1.0\hsize,clip]{DMStripe82.jpg}
\caption{
A small region in the vicinity of globular cluster M2, taken from a coadd of SDSS Stripe 82 data produced with LSST software stack prototypes. The coaddition employs a novel ``background matching" technique that improves background estimation and preserves the diffuse structures in the resulting coadd. The full coadd can be browsed at \url{http://moe.astro.washington.edu/sdss}.} 
\label{Fig:DMStripe82}
\end{figure}


The LSST data management software has been prototyped for over eight
years. It has been exercised in eight data challenges, with increasing
degrees of complexity. Besides processing simulated LSST data
(\S~\ref{sec:imsim}), it has been used to process images from CFHTLS
and SDSS (Abazajian et al. 2009). As an example,
Figure~\ref{Fig:DMStripe82} shows a small region in the vicinity of M2
taken from a large coaddition of SDSS Stripe 82 data, generated with LSST
software stack prototypes. 
\\

The LSST software stack is free software, licensed under the terms of the GNU General
Public License, Version 3. The stack
prototype code and documentation are available via \url{http://ls.st/ug}.
Its open source nature, an open development
process, attention to software engineering, a long-term project
commitment and a modular design that can be modified for use with
other cameras 
may make it useful for the processing of imaging data beyond LSST.  

\vskip 0.4in
\subsection{Simulating the LSST System}


Throughout its design, construction and commissioning, the LSST needs
to be able demonstrate that it can achieve the requirements laid out
in the Science Requirements Documents (SRD) given its design and
as-delivered components, that the system can be calibrated to the
required level of fidelity, that the data management software can
extract the appropriate astrophysical signals, and that this can be
achieved with sufficient efficiency such that the telescope can
complete its primary objectives within the ten years of its duration
(including surveying 18,000 sq degrees of the sky, and completing the
Deep Drilling Fields, see \S \ref{Sec:minisurveys}). 


Realizing these objectives requires that the project can characterize
the performance of the LSST including the performance of the
opto-mechanical systems, the response of the detectors and their
electronics, and the capabilities of the analysis software. A
simulation framework provides such a capability; delivering a virtual
prototype LSST against which design decisions, optimizations
(including descoping), and trade studies can be evaluated (Connolly
et al.~2014).


The framework underlying the LSST simulations is designed to be
extensible and scalable (i.e. capable of being run on a single
processor or across many-thousand core compute clusters). It comprises
four primary components: a simulation of the survey scheduler
(\S~\ref{sec:opsim}),
databases of simulated astrophysical catalogs of stars, galaxies,
quasars and Solar System objects (\S~\ref{sec:catalogs}), a system for generating observations
based on the pointing of the telescope, and a system for generating
realistic LSST images of a given area of sky
(\S~\ref{sec:imsim}). Computationally intensive routines are written 
in $C/C^{++}$ with the overall framework and database interactions
using $Python$.  The purpose of this design is to enable the
generation of a wide range of data products for use by the
collaboration; from all-sky catalogs used in simulations of the LSST
calibration pipeline, to studies of the impact of survey cadence on
recovering variability, to simulated images of a single LSST focal
plane.


\subsubsection{ The LSST Operations Simulator \label{sec:opsim}}

The LSST Operations Simulator (Delgado et al.~2014) was developed to enable a 
detailed quantitative analysis of the various science tradeoffs described in 
this paper. It contains detailed models of site conditions, hardware and
software performance, and an algorithm for scheduling observations which will, 
eventually, drive the largely robotic observatory. 
Observing conditions include a model for seeing derived from an extensive body
of on-site MASS/DIMM (Multi-Aperture Scintillation Sensor and Differential
Image Motion Monitor) measurements obtained during site selection and
characterization (see Fig.~\ref{Fig:seeing}). It not only reproduces the 
observed seeing distribution, but includes 
the auto-correlation spectrum of seeing with time over intervals from minutes 
to seasons. Weather data are taken from ten years of hourly measurements at
nearby Cerro Tololo. The time history of site conditions is important if the 
simulation is to faithfully reproduce a sequence of observations. Thus, for
example, the simulator correctly represents the variation of limiting
magnitude between pairs of observations used to detect NEOs and the
correlation between, for example, seasonal weather patterns and observing
conditions at any given point on the sky.  In addition, down time for
observatory maintenance is also included. 

The signal-to-noise ratio of each 
observation is determined using a sky background model which includes the dark
sky brightness in each filter, the effects of seeing and atmospheric
transparency, and a detailed model for scattered light from the moon and/or 
twilight at each observation (Yoachim et al.~2014). The time taken to move from one observation to
the next is given by a detailed model of the camera, telescope, and dome. It 
includes such effects as the acceleration/deceleration profiles employed in 
moving the telescope, the dome, and the wind screen, 
% in altitude, azimuth, camera rotator, dome azimuth, and wind/stray light screen altitude, 
the time needed to damp vibrations excited by each slew, 
cable wrap, the time taken for active optics lock and correction as a function of 
slew distance, and the time for filter changes and focal plane readout. 

Observations are scheduled by a ranking algorithm. After a given exposure, all 
possible next observations are assigned a score which depends upon their locations, times,
and filters according to a set of scientific requirements which can vary with 
time and location. For example, if an ecliptic field has been observed in the
$r$ band, the score for another $r$-band observation of the same field will 
initially be quite low, but it will rise in time to peak about an hour after
the first observation, and decline thereafter. This algorithm results in
observations being acquired as pairs roughly an hour apart, which enables
efficient association of NEO detections. To ensure uniform 
sky coverage, fields with fewer previous observations will be scored more
highly than those which have already been observed more frequently.
 
Once all possible next observations have been scored for scientific
priority, their scores are modified according to observing conditions
(e.g., seeing, airmass, and sky brightness) and to criteria such as
low slew time to move from the current position, time required to
change filters, etc. The highest-ranked observation is then performed,
and the cycle repeats. The result of a simulator run is a detailed
history of which locations on the sky were observed when, in what
filter, and with what sky background, seeing and other observing
conditions.  It takes a few days to produce a decade-long simulation
using an average PC.

Results of the simulated surveys can be visualized and analyzed using
a Python-based package called the Metrics Analysis Framework (MAF;
Jones et al.~2014). MAF provides tools to analyze the properties of a
survey (e.g.\ the distribution of aimasses) through the creation of
functions or metrics that are applied to OpSim outputs. These metrics
can express the expected technical performance of the survey, such as
the number of visits per field or the integrated depth after 10 years,
as well as the science capabilities or a survey, such as the number of
supernovae detected or the number of supernovae with sufficient
observations that they can be easily classified.

\subsubsection{Catalog Generation}
\label{sec:catalogs}

The simulated astronomical catalogs (CatSim; Connolly et al.~2014) are
stored in an SQL database. This base catalog is queried using
sequences of observations derived from the Operations Simulator. Each
simulated pointing provides a position and time of the observation
together with the appropriate sky conditions (e.g. seeing, moon phase
and angle, sky brightness and sky transparency). Positions of sources
are propagated to the time of observation (including proper motions
for stars and orbits for Solar System sources). Magnitudes and source
counts are derived using the atmospheric and filter response functions
appropriate for the airmass of the observation and after applying
corrections for source variability.  The resulting catalogs are then
formatted for either output to users, or fed into an image
simulator.

The current version of the LSST simulation framework incorporates
galaxies derived from an N-body simulation of a $\Lambda$CDM
cosmology, quasars/AGNs, stars that match the observed stellar
distributions within our Galaxy, asteroids generated from simulations
of our Solar System, and a 3-D model for Galactic extinction.  Stellar
sources are based on the Galactic structure models of Juri\'{c} et
al.~(2008) and include thin-disk, thick-disk, and halo star
components. The distribution and colors of the stars match those
observed by SDSS. Each star in the simulation is matched to a template
spectral energy distribution (SED). Kurucz (1993) model spectra are
used to represent main-sequence F, G, and K stars as well as RGB
stars, blue horizontal branch stars, and RR Lyrae variables.  SEDs for
white dwarf stars are taken from Bergeron et al.~(1995).  SEDs for M,
L, and T dwarfs are generated from a combination of spectral models
and by stacking spectra from the SDSS (e.g., Cushing et al.~2005,
Bochanski et al.~2007, Burrows et al.~2006, Petterson \& Hawley 1989,
Kowalski et al.~2010). The adopted metallicity for each star is based
on a model from Ivezi\'{c} et al.~(2008a), and proper motions are
based on the kinematic model of Bond et al.~(2010).  Light curve
templates are assigned to a subset of the stellar population so that
variability may also be simulated. This assignment and variability are
matched to that observed by the Kepler satellite and augmented by
simulated distributions of RR-Lyrae and Cepheids. For Galactic reddening, a
value of $E(B-V)$ is assigned to each star using the three-dimensional
Galactic model of Amores \& Lepine (2005). To provide consistency with
the modeling of extragalactic fluxes in the simulations, the dust model in the Milky Way integrated
to 100 kpc is re-normalized to match the Schlegel et al.~(1998) dust maps.

Galaxy catalogs are derived from the Millennium simulations of de
Lucia et al.~(2006).  These models extend pure dark matter N-body
simulations to include gas cooling, star formation, supernovae and
AGN, and are designed to reproduce the observed colors, luminosities,
and clustering of galaxies as a function of redshift. To generate the
LSST simulated catalogs, a light cone, covering redshifts $0<z<6$, was
constructed from 58 simulation snapshots 500h$^{-1}$Mpc on a side. This
light cone extends to a depth of approximately $r=28$ and covers a
4.5$^\circ$$\times$4.5$^\circ$ footprint on the sky. Replicating this
catalog across the sky simulates the full LSST footprint. As with the
stellar catalog, an SED is fit to the colors of each source using
Bruzual \& Charlot (2003) spectral synthesis models. These fits are
undertaken separately for the bulge and disk components and, for the
disk, include inclination-dependent reddening. Morphologies are
modeled using two Sersic profiles. The bulge-to-disk ratio and disk
scale lengths are taken from de Lucia et al.~(2006). Half-light radii
for bulges are estimated using the empirical absolute-magnitude
vs. half-light radius relation given by Gonzalez et
al.~(2009). Comparisons between the redshift and number-magnitude
distributions of the simulated catalogs with those derived from deep
imaging and spectroscopic surveys showed that the de Lucia et
al.~models under-predict the density of sources at faint magnitudes
and high redshifts. To correct for these effects, sources are cloned
in magnitude and redshift space until their densities reflect the
average observed properties.

Quasar/AGN catalogs are generated using the Bongiorno et al.~(2007)
luminosity function for $M_B < -15$, over an area of 100
deg$^2$. Their observed SEDs are generated using a composite
rest-frame spectrum derived from SDSS data by Vanden Berk et
al.~(2001). The host galaxy is selected to have the closest match to
the preferred stellar mass and color at the AGN's redshift, following
the results from Xue et al.~(2010).  Each galaxy hosts at most one
AGN, and no explicit distinction is made between low-luminosity AGN and
quasars that dramatically outshine their host galaxies. The light
curve for each AGN is generated using a damped random walk model and
prescriptions given by MacLeod et al.~(2010).

Asteroids are simulated using the Solar System models of Grav et
al.~(2007). They include: Near Earth Objects (NEOs), Main Belt
Asteroids, the Trojans of Mars, Jupiter, Saturn, Uranus, and Neptune,
Trans Neptunian Objects, and Centaurs. Spectral energy distributions
are assigned using the C and S type asteroids of DeMeo et
al.~(2009). Positions for the 11 million asteroids in the simulation
are stored within the base catalog (sampled once per night for the ten
year duration of the LSST survey). We generate
accurate ephemerides of all asteroids falling within a given LSST
point using the $PyOrb$ software package (Granvik et
al.~2009). With typically 8000 sources per LSST field of view, this
procedure significantly reduces the computational resources
required to simulate asteroid ephemerides.



\subsubsection{Image Simulations}
\label{sec:imsim}

The framework described above provides a parametrized view of the sky
above the atmosphere. Images are simulated using two packages: GalSim
(Rowe et al.~2015), and Phosim (Peterson et al.~2015). Galsim is a
modular and open-source package that provides a library for simulating
stars and galaxies through a range of modern astronomical
telescopes. Point-spread-functions (PSFs) are treated as either
analytic functions or modelled from ray-traced optics. Convolutions by
the PSF can be applied to parameterized galaxy profiles (e.g.\
Sersic profiles) or to directly observed images. Operations are
applied in Fourier space to enable an effective trade-off between
speed of simulation and accuracy. GalSim is written in C++ with a
Python API and is integrated within the LSST CatSim framework.

Phosim is an open-source package that simulates images by drawing
photons from the spectral energy distribution of each source (scaled
to the appropriate flux density based on the apparent magnitude of a
source and accounting for the spatial distribution of light for
extended sources). Each photon is ray-traced through the atmosphere,
telescope and camera to generate a CCD image. The atmosphere is
modeled using a Taylor frozen screen approximation (with the
atmosphere described by six layers). The density fluctuations within
these screens are described by a Kolmogorov spectrum with an outer
scale (typically 10m to 200m). All screens move during an exposure
with velocities derived from NOAA measurements of the wind velocities
above the LSST site in Chile.  Typical velocities are on the order of
20 m s$^{-1}$, and are found to have a seasonable dependence that is
modeled when generating the screens. Each photon's trajectory is
altered due to refraction as it passes through each screen.


\begin{figure*}
\centerline{\includegraphics[width=0.98\textwidth,clip]{chip2014.jpg}}
%\vskip -1in
\caption{ A simulated image of a single LSST CCD using PhoSim
  (covering a $13.3\times13.3$ arcmin$^2$ region of the sky). The
  image is a color composite (Lupton et al.~2004) from a set of 30
  second $gri$ visits.}
\label{Fig:ImSimExample}
\end{figure*}

After the atmospheric refraction, the photons in PhoSim are reflected
and refracted by the optical surfaces within the telescope and
camera. The mirrors and lenses are simulated using geometric optics
techniques in a fast ray-tracing algorithm and all optical surfaces
include a spectrum of perturbations based on design tolerances. Each
optic moves according to its six degrees of freedom within tolerances
specified by the LSST system. Fast techniques for finding intercepts
on the aspheric surface and altering the trajectory of a photon by
reflection or wavelength-dependent refraction have been implemented to
optimize the efficiency of the simulated images. Wavelength and
angle-dependent transmission functions are incorporated within each of
these techniques, including simulation of the telescope spider.

Both GalSim and PhoSim model the propagation of photons through the
silicon of the detector. The conversion probability, refraction as a
function of wavelength and temperature, and charge diffusion within
the silicon are modeled for all photons. Photons are pixelated and the
readout process simulated including blooming, charge saturation,
charge transfer inefficiency, gain and offsets, hot pixels and
columns, and QE variations. 

An example of a simulated LSST image using PhoSim is shown in
Fig.~\ref{Fig:ImSimExample}.  

\section{    ANTICIPATED DATA PRODUCTS AND THEIR CHARACTERISTICS    }
\label{Sec:dataprod}

The LSST observing strategy is designed to maximize the scientific
throughput by minimizing slew and other downtime and by making appropriate
choices of the filter bands given the real-time weather conditions.
Using simulated surveys produced with the Operations Simulator described in \S~\ref{sec:opsim}, 
we illustrate predictions of LSST performance with two examples. 


\subsection{ The Baseline LSST Surveys }
\label{sec:baseline}

The fundamental basis of the LSST concept is to scan the sky deep, wide, and
fast, and to obtain a dataset that simultaneously satisfies the majority
of the science goals. We present here a specific realization, the
so-called ``universal cadence'', which yields the main deep-wide-fast
survey and meets our core science goals.  However, at this writing,
there is a vigorous discussion of cadence plans in the LSST community,
exploring variants and alternatives that enhances various specific
science programs, while maintaining the science requirements described
in the SRD.  

The main deep-wide-fast survey %(typical single visit depth of $r\sim24.5$)
will use about 90\% of the observing time. The remaining 10\% of the observing 
time will be used to obtain improved coverage of parameter space such as 
very deep ($r\sim26$) observations, observations with very short revisit 
times ($\sim$1 minute), and observations of ``special'' regions such as the 
Ecliptic, Galactic plane, and the Large and Small Magellanic Clouds. 
We are also considering a third type of survey, micro-surveys, that would 
use about 1\% of the time (which still represents 25 nights on a unique 
8m-class telescope). 

\subsubsection{ The Main Deep-Wide-Fast Survey }


\begin{figure}
%\includegraphics[width=1.0\hsize,clip]{rbandSky.pdf}
%\includegraphics[width=0.78\hsize,angle=90.0,clip]{rvisits_SciBook.pdf}
%\hskip -0.0in
\includegraphics[width=1.0\hsize,clip]{rBandWhite.png}
\vskip -0.01in
\caption{The distribution of the $r$ band visits on the sky for a simulated 
realization of the baseline cadence. The sky is shown in the equal-area Mollweide 
projection in equatorial coordinates (the vernal equinoctial point is in the center, and 
the right ascension is increasing from right to left). The number of visits for 
a 10-year survey, normalized to the SRD design value of 184, is color-coded according
to the legend. The two regions with smaller number of visits than the main survey 
(``mini-surveys'') are the Galactic plane (arc on the right) and the region around the 
South Celestial Pole (bottom). The so-called ``northern Ecliptic region'' (upper left)
has received more visits than the main survey in this particular simulation (in
order to increase completeness for moving objects by increasing the coverage of
the Ecliptic plane). Deep drilling fields, with a much higher number of visits than
the main survey, are also visible as small circles. The fields were dithered on 
sub-field scales and pixels with angular resolution of $\sim$30 arcmin were used 
to evaluate and display the coverage.} 
\label{Fig:rbandSky}
\end{figure}

The observing strategy for the main survey will be optimized for the homogeneity
of depth and number of visits. In times of good seeing and at low airmass, preference 
is given to $r$-band and $i$-band observations. As often as possible, each field will be 
observed twice, with visits separated by 15-60 minutes. This strategy will provide motion 
vectors to link detections of moving objects in the Solar System, and fine-time sampling 
for measuring short-period variability. The ranking criteria also ensure that the 
visits to each field are widely distributed in position angle on the sky and 
rotation angle of the camera in order to minimize systematic effects in galaxy shape 
determination.

The universal cadence provides most of LSST's power for detecting Near Earth 
Objects (NEO) and naturally incorporates the southern half of the ecliptic 
within its 18,000 square degrees (the northern half lies above the desired airmass 
limits, $X\la1.5$). NEO sample completeness for the smallest bodies ($\sim$140m in 
diameter, per the Congressional NEO mandate) is greatly enhanced, however, by the 
addition of a crescent on the sky within 10 degrees of the northern ecliptic
(see Fig.~\ref{Fig:rbandSky}). Thus, we plan to extend
%the ``northern Ecliptic proposal'' (here ``proposal'' refers to an observing program) extends
 the universal cadence to this region using the
$r$ and $i$ filters only, along 
with more relaxed limits on airmass and seeing. Relaxed limits on airmass and 
seeing are also adopted for $\sim$700 deg$^2$ around the South Celestial 
Pole, allowing coverage of the Large and Small Magellanic Clouds.

Finally, the universal cadence proposal excludes observations in a region of 
1,000 square degrees around the Galactic Center, where the high stellar
density leads to a confusion limit at much brighter magnitudes than those 
attained in the rest of the survey. Within this region, the Galactic Center
proposal provides 30 observations in each of the six filters, distributed 
roughly logarithmically in time (it may not be necessary to use the
$u$ and $g$ filters for this heavily extincted region). 

The resulting sky coverage for the LSST baseline
cadence, based on detailed operations simulations, is shown for the 
$r$ band in Fig.~\ref{Fig:rbandSky}. The anticipated total number of visits 
for a ten-year LSST survey is about 2.8 million ($\sim$5.6 million 15-second long
exposures). The per-band allocation of these visits is shown in Table 1.



\subsubsection{ Mini-surveys}
\label{Sec:minisurveys}

Although the uniform treatment of the sky provided by the universal cadence
proposal can satisfy the majority of LSST scientific goals, roughly 10\%
of the time will be allocated to other strategies that significantly enhance the 
scientific return.  These surveys aim to extend the parameter space accessible 
to the main survey by going deeper or by employing different time/filter
sampling. 

As an example of a mini-survey, consider a program that uses one hour of
observing time per night to observe a relatively small region of sky to
substantially greater depth in individual visits. Accounting for
read-out time and filter changes, it could obtain about 50 consecutive
15-second exposures in each of four filters in an hour. If a field is visited
every two days over four months, about 600 deg$^2$ can be observed with this 
cadence over 10 years. Taking weather into account, the selected fields would 
each have on average about 40 hour-long sequences of 200 exposures each. Each 
observation in a sequence would have an equivalent 5-$\sigma$ depth of
$r\sim24.5$, and each filter subsequence when coadded would be 2 magnitudes 
deeper than the main survey visits ($r\sim26.5$). When all 40 sequences and 
the main survey visits are coadded, they would extend the depth to $r\sim28$. 

This data set would be excellent for a wide variety of science programs. The 
individual sequences would be sensitive to 1\% variability on hourly time 
scales, allowing discovery of planetary eclipses. If these fields were selected 
at Galactic latitudes of $|b|\sim30$ deg, they would include about 10 million 
stars with $r<21$ observed with signal-to-noise ratio above 100 in each visit.
When subsequences from a given night were coadded, they would 
provide dense time sampling to a faint limit of $r\sim26.5$ (assuming observations
in 4 bands, every 2 days over 120 days, and accounting for weather losses), and thus 
enable deep searches 
for SN, trans-Neptunian objects, and other faint transient, moving and 
variable sources.  For example, the SN sample
would be extended to redshifts of $z\sim1.2$, with more densely sampled light
curves than obtained from the universal cadence. Such sequences would also
serve as excellent tests of our photometric calibration procedures. 

The LSST has already selected four distant extragalactic survey fields\footnote{For
details, see http://www.lsst.org/News/enews/deep-drilling-201202.html}
that the project guarantees to observe as Deep Drilling Fields with deeper coverage 
and more frequent temporal sampling than provided by the standard LSST observing 
pattern. These fields (Elias S1, XMM-LSS, Extended Chandra Deep Field-South, and
COSMOS) are  well-studied survey fields with substantial existing multiwavelength 
coverage and other positive attributes. These four fields are only the first chosen 
for deep-drilling observations; more such fields will be chosen later.

The baseline universal cadence is by no means the definitive plan for the entire
survey. Rather, it represents a proof of concept that it is indeed possible to 
design a universal cadence which addresses a wide variety of science goals in a nearly 
optimal way. We are undertaking a vigorous and systematic research effort to explore 
the enormously large parameter space of possible surveys. The
scientific commissioning period 
will be used to test the usefulness of various observing modes and to explore 
alternative strategies. Proposals from the community
%, through the science collaborations (see \S~\ref{Sec:community}), 
for specialized cadences (such as mini-surveys and
micro-surveys) will also be considered.  



\begin{table}
\caption{The Parameters From Eqs.~\ref{ggg} and \ref{m5}}
\begin{tabular}{|r|r|r|r|r|r|r|}
\hline  
                           &   $u$  &   $g$   & $r$   &  $i$  & $z$  & $y$  \\
\hline  
   $m_{\rm sky}^a$ &   22.9    & 22.3    & 21.2    & 20.5    & 19.6    &  18.6  \\
   $\theta^b$       &   0.77     &  0.73     & 0.70    & 0.67    &  0.65   &  0.63  \\
   $\gamma^c$   &   0.037   & 0.038    & 0.039   & 0.039   & 0.040   & 0.040 \\
    $k_m^d$        &    0.451  &  0.163   &  0.087  &  0.065   &  0.043   &  0.138 \\
    $C_m^e$        &   22.92   & 24.29    & 24.33   & 24.20   & 24.07   & 23.69 \\
    $m_5^f$         &   23.68     &   24.89    & 24.43     &  24.00   & 24.45    & 22.60  \\
 $\Delta C^{\infty,g}_m$  &  0.67   &  0.21     &  0.11   &  0.08   &   0.05  &  0.04  \\ 
 $\Delta C_m(2)^h$       &  0.24   &  0.09     &   0.05  &  0.04   &   0.03  &  0.02 \\   
   $\Delta m_5^i$ &   0.21    & 0.15   & 0.14     &  0.13   & 0.13    & 0.15  \\
\hline                         
\end{tabular}
  \\ \vskip 0.05in
  $^a$ The expected median zenith sky brightness at Cerro Pach\'on, derived from
           the median dark sky brightness observed by SDSS (AB mag arcsec$^{-2}$). \\
  $^b$ The expected delivered median zenith seeing (arcsec). For larger
           airmass, $X$, seeing is proportional to $X^{0.6}$. \\
  $^c$ The band-dependent parameter from Eq.~\ref{ggg}. \\
  $^d$ Adopted atmospheric extinction. \\
  $^e$ The band-dependent parameter from Eq.~\ref{m5}. \\
  $^f$ The typical 5$\sigma$ depth for point sources at zenith, assuming exposure time of 
          2$\times$15 sec, and observing conditions as listed. For larger
          airmass the 5$\sigma$ depth is brighter; see the bottom row. \\
  $^g$ The loss of depth due to instrumental noise (assuming 9 e$^-$ per pixel and readout, 
       and two readouts per visit). \\
  $^h$ Additive correction to $C_m$ when exposure time is doubled from its fiducial value 
          to 60 sec. \\
  $^i$ The loss of depth at airmass of $X=1.2$ due to seeing degradation 
                 and increased atmospheric extinction. \\
\end{table}






\subsection{  Detailed Analysis of Simulated Surveys  } 

As examples of analysis enabled by the Operations Simulator, we describe 
determination of the completeness of the LSST NEO sample, and estimation 
of errors expected for trigonometric parallax and proper motion measurements. 
In both examples, the conclusions crucially depend on assumed signal-to-noise
ratios, described next.

\subsubsection{  Expected Photometric Signal-to-Noise Ratio } 

The output of operations simulations is a data stream consisting of 
a position on the sky and the time of observation, together with 
observing conditions such as seeing and sky brightness. The expected 
photometric error (the inverse of the signal-to-noise ratio) for a single visit
can be written as
\begin{equation}
         \sigma_1^2 = \sigma_{sys}^2 + \sigma_{rand}^2,
\end{equation}
where $\sigma_{rand}$ is the random photometric error and $\sigma_{sys}$ is 
the systematic photometric error (includes errors due to, e.g., imperfect
modeling of the point spread function, but does not include errors in 
absolute photometric zeropoint). The calibration system and procedures
are designed to maintain $\sigma_{sys}<0.005$ mag. Based on 
SDSS experience (Sesar et al.~2007), the random photometric error for
point sources, as
a function of magnitude, is well described\footnote{Eq.~\ref{ggg} can 
be derived from $\sigma_{rand}=N/S$, where $N$ is noise and $S$ is signal, 
and by assuming that $N^2 = N_o^2 + \alpha S$. The constants $N_o$ and 
$\alpha$ can be expressed in terms of a single unknown constant $\gamma$ 
by using the condition that $\sigma_{rand}=0.2$ for $m=m_5$.} by
\begin{equation}
\label{ggg}
  \sigma_{rand}^2 = (0.04-\gamma)\, x + \gamma \, x^2 \,\,\, {\rm (mag^2),}
\end{equation}
with $x \equiv 10^{0.4\,(m-m_5)}$. Here $m_5$ is the 5$\sigma$ depth (for
point sources) in a given band, and $\gamma$ depends on the sky 
brightness, readout noise, etc. 
%Using the LSST exposure time 
%calculator\footnote{Available at http://dls.physics.ucdavis.edu/etc.} 
%(Gee et al.~2007), 
Detailed determination of the system throughput yields the values of $\gamma$ 
listed in Table 2. The 5$\sigma$ depth for point sources is determined from 
\begin{eqnarray}
\label{m5}
  m_5 = C_m + 0.50\,(m_{sky}-21) + 2.5\,\log_{10}(0.7/\theta) +  \nonumber \\
        + 1.25\,\log_{10}(t_{vis}/30) - k_m(X-1) \phantom{xxxxx}
\end{eqnarray}
where $m_{sky}$ is the sky brightness (AB mag arcsec$^{-2}$), $\theta$ is 
the seeing (FWHM, in arcsec), $t_{vis}$ is the exposure time (seconds),
$k$ is the atmospheric extinction coefficient, and $X$ is airmass. 


The constants $C_m$ depend on the overall throughput of the instrument
and are computed using currently the best available throughput estimates for
optical elements and sensors. The resulting $C_m$ values are listed in Table 2
and in all six bands they imply single visit depths (also listed in Table 2) in the range 
between minimum and design specification values from the Science Requirements
Document listed in Table 1. 
The differences in performance between LSST and, for example, SDSS are easily 
understood\footnote{SDSS data 
typically reach a 5$\sigma$ depth for point sources of $r=22.5$ 
with an effective aperture of $D=2.22$ m, an exposure time of $t_{vis}=54$ 
sec, the median $r$ band sky brightness of $r_{sky}=20.9$ mag arcsec$^{-2}$, 
the median seeing of $\theta=1.5$ arcsec, and the median airmass of $X=1.3$.
%%% fix numbers: (aperture and throughout) 
In comparison, the LSST loss of depth is 0.32 mag due to shorter exposures,
while the gains are 1.17 mag due to larger aperture, 0.83 mag due to better 
seeing, 0.20 mag due to fainter sky, for the net gain of $\sim$1.9 mag.}.

The structure of eq.~\ref{m5} nicely illustrates decoupling between the system 
sensitivity which is fully absorbed into $C_m$ and observing conditions
(system deployment parameters) specified by $m_{sky}$, $\theta$, $t_{vis}$, $k_m$ 
and $X$. The computation of $C_m$ listed in Table 2 assumed instrumental noise of 
9 e$^-$ per pixel and per readout, which has by and large negligible effect on $m_5$ in 
all bands except for the $u$ band. This loss of depth due to instrumental noise, $\Delta C^{\infty}_m$,
is listed in Table 2; it also corresponds to additive correction to $C_m$ when the
exposure time $t_{vis} \rightarrow \infty$. To predict $5\sigma$ depths for 
exposure time $\tau$ times longer that the fiducial $t_{vis} = 30$ sec., the 
following correction has to be added to $C_m$ listed in Table 2:
\begin{equation}
\label{eq:DCm} 
 \Delta C_m(\tau) = \Delta C^\infty_m - 1.25\,\log_{10}\left[1 + {10^{(0.8 \, \Delta C^\infty_m)} - 1 \over \tau}  \right].
\end{equation}
Of course, $\Delta C_m(\tau=1)=0$. Again, this effect is only substantial in the $u$ 
band, as demonstrated by $\Delta C_m(\tau = 2)$ listed in Table 2.  

The loss of depth at the airmass of $X=1.2$ due to seeing degradation 
and increased atmospheric extinction is listed in the last row in Table 2. Note
that the uncertainty of limiting depth predictions due to unpredictable solar 
activity (which influences the night sky brightness, Patat 2008) is about 
0.1-0.2 mag. 




\subsubsection{   The NEO Completeness Analysis    } 
\label{Sec:NEOc}
To assess the LSST completeness for PHAs, the PHA 
population is represented by a size-limited complete sample of 800 true
PHAs whose orbital elements are taken from the Minor Planet Center.
The simulated baseline survey is used to determine which PHAs are present in
each exposure and at what signal-to-noise ratio they were observed. In 
addition to  seeing, atmospheric transparency, and sky background effects
(see eq.~\ref{m5}), the signal-to-noise computation takes into account losses 
due to non-optimal filters and object trailing. Using SDSS observations
of asteroids (Ivezi\'c et al.~2001), we adopt the following mean colors to 
transform limiting (AB) magnitudes in LSST bandpasses (listed in Table 2)
to an `effective'' limiting magnitude in the standard $V$ band: 
$V-m=(-2.1, -0.5, 0.2, 0.4, 0.6, 0.6)$ for $m=(u,g,r,i,z,y)$. Due to 
very red $V-u$ colors, and the relatively bright limiting magnitude in the $y$ 
band, the smallest objects are preferentially detected in the $griz$ bands.
The correction for trailing is implemented by subtracting from the right-hand 
side of eq.~\ref{m5}
\begin{equation}
 \Delta m_5^{\rm trailing} = 1.25\,\log_{10}\left(1+0.028{v \,t_{vis} \over \theta}\right),
\end{equation}
where the object's velocity, $v$, is expressed in deg.~day$^{-1}$. For the nominal
exposure time of 30 seconds and $\theta=0.7$ arcsec, the loss of limiting 
magnitude is 0.14 mag for $v=0.25$ deg.~day$^{-1}$, typical for objects in the main 
asteroid belt, and 0.43 mag for $v=1.0$ deg.~day$^{-1}$, typical of NEOs passing
near Earth. 

For a survey that has a completeness to NEOs above 60\%, each
additional one magnitude of depth for a given survey cadence increases
the completeness by another 10\%.  An object's orbit is considered to be 
determined if the object was detected on at least three nights during a single 
lunation, with a minimum of two visits per night. The same criterion
has been used in NASA studies,
%\footnote{The NASA 2007 NEO study is available from
%http://neo.jpl.nasa.gov/neo/report2007.html.}, 
%[WE'VE GIVEN THIS REFERENCE IN A FOOTNOTE EARLIER]
and is confirmed as 
reliable by a detailed analysis of orbital linking and determination using
the Moving Object Processing System (MOPS) code (Jedicke et al.~2005) developed by the Pan-STARRS project (and 
adopted by LSST in a collaborative effort with Pan-STARRS). The MOPS software
system and its algorithms are significantly more advanced than
anything previously 
fielded for this purpose to date. Realistic MOPS simulations show 
$>$99\% linking efficiency across all classes of Solar System objects. 
For the LSST baseline cadence, objects
counted as having well-determined orbits are observed on 20 different nights on average over ten
years. A more stringent requirement that an object must be detected on at least 
five nights per lunation decreases the completeness by typically 3\%. 
The completeness is also a function of the assumed size distribution of NEOs, the flatter the distribution, the 
higher the completeness, and different assumptions contribute a systematic uncertainty of about 2\%. 

The LSST baseline cadence provides orbits for 82\% of PHAs larger than 140
meters after 10 years of operations.  This can be increased to 84\%
completeness (90\% for sizes above 200 meters) with minor changes to
the cadence, such as 
requiring that all observations North of $\delta =+5^\circ$) are
obtained in the $r$ band.  The completeness curve as a function of an object's size is shown 
in Fig.~\ref{Fig:Cneo} (lower curve). This cadence spends 5\% of the total 
observing time on NEO-optimized observations north of $\delta = +5^\circ$.



\begin{figure}
\includegraphics[width=1.0\hsize,clip]{Cneo.pdf}
\caption{Completeness of the LSST survey for PHAs brighter than a given absolute
magnitude (related to the size of the object and albedo; 
$H$=22 mag is equivalent to a typical 140m asteroid and $H$=24 mag is
equivalent to a 50m asteroid). Two scenarios are shown: the lower curve is the 
10-year long baseline survey with a 5\% NEO optimization, and it reaches a 
completeness of 84\%. The upper dashed curve results from spending 15\% of the 
observing time in an NEO optimized mode, and running the survey for 12 years.  
It meets the 90\% completeness level for 140m objects mandated by the Congress.} 
\label{Fig:Cneo}
\end{figure}

Various adjustments to the baseline cadence can boost the completeness for
140m and larger PHAs to 90\%. We find that such variations can have an unacceptably 
large impact on other science programs, if the 90\% completeness is to be reached 
within the 10 year survey lifetime. However, with a minor adjustment of the 
baseline cadence, such that 15\% of the time is spent north of $\delta
= +5^\circ$ to reach
fainter limiting magnitudes, this completeness level can be reached
with a 12 year 
survey, and with a negligible effect on other science goals. The completeness 
curve as a function of an object's size for such a modified cadence is shown in 
Fig.~\ref{Fig:Cneo} (upper curve).

Our analysis assumes that no NEOs are known prior to LSST. Currently known
NEOs do not have a significant impact on this calculation. 
%However, if a precursor 
%survey, such as Pan-STARRS 4, operated for three years prior to LSST, the time to
%fulfill the Congressional mandate by LSST could be shortened by about a year.


\subsubsection{The Expected Accuracy of Trigonometric Parallax and Proper Motion Measurements } 
\label{sec:astrom}

Given the observing sequence for each sky position in the main survey, we
generate a time sequence of mock astrometric measurements. The assumed astrometric 
accuracy is a function of signal-to-noise ratio. Random astrometric errors per
visit are modeled as $\theta/SNR$, with $\theta=700$ mas and $SNR$ determined using
eq.~\ref{m5}. The estimated proper motion and parallax accuracy at the bright end
($r<20$) is driven by systematic errors due to the atmosphere. Systematic
errors of 10 mas are added in quadrature, and are assumed to be {uncorrelated} 
between different observations of a given object. Systematic and random
errors become similar at about $r=22$, and there are about 100 stars per LSST 
sensor (0.05 deg$^2$) to this depth (and fainter than the LSST saturation limit at
$r\sim16$) even at the Galactic poles. 

Data from the Subaru telescope indicate that systematic errors of 
10 mas on spatial scales of several arcminutes are realistic. Even a drift-scanning 
survey such as SDSS delivers uncorrelated systematic errors (dominated by seeing 
effects) at the level of 20-30 mas (measured from repeated scans; Pier et al.~2003);
the expected image quality for LSST will be twice as good as for SDSS. Furthermore, 
there are close to 1000 galaxies per sensor with $r<22$, which will provide exquisite 
control of systematic astrometric errors as a function of magnitude, color and other 
parameters, and thus enable absolute proper motion measurements.



\begin{table}[b!]
\caption{The expected proper motion, parallax and accuracy for a 10-year long baseline survey.}
\begin{tabular}{|l|c|c|c|c|c|}
\hline  
    $r$   &  $\sigma^a_{xy} $  & $\sigma^b_\pi$  &   $\sigma^c_\mu$   &  $\sigma^d_1$  &  $\sigma^e_C$  \\
    mag &       mas            &      mas  & mas/yr &   mag   &    mag  \\
\hline  
       21 &  11  &  0.6  &  0.2   &   0.01  &   0.005 \\
       22 &  15  &  0.8  &  0.3   &   0.02  &   0.005 \\
       23 &  31  &  1.3  &  0.5   &   0.04  &   0.006 \\
       24 &  74  &  2.9  &  1.0   &   0.10  &   0.009 \\
\hline                         
\end{tabular}
\\ \vskip 0.05in
  $^a$ Typical astrometric accuracy (rms per coordinate per visit); \\
  $^b$ Parallax accuracy for 10-year long survey; \\
  $^c$ Proper motion accuracy for 10-year long survey; \\
  $^d$ Photometric error for a single visit (two 15-second exposures); \\
  $^e$ Photometric error for coadded observations (see Table 1). \\
\end{table}


The astrometric transformations for a given CCD and exposure, and 
proper motion and parallax for all the stars from a given CCD, are simultaneously
solved for using an iterative algorithm. The astrometric transformations from
pixel to sky coordinates are modeled using low-order polynomials and standard
techniques developed at the U.S. Naval Observatory (Monet et al.~2003). The expected 
proper motion and 
parallax errors for a 10-year long baseline survey, as a function of apparent 
magnitude, are summarized in Table 3. Blue stars (e.g., F/G stars) fainter than 
$r\sim23$ will have about 50\% larger proper motion and parallax errors than 
given in the table due to decreased numbers of $z$ and $y$ band detections. The 
impact on red stars is smaller due to a relatively small number of observations 
in the $u$ and $g$ bands, but extremely red objects, such as L and T dwarfs, 
will definitely have larger errors, depending on details of their spectral 
energy distributions.  After the first three years of the survey, 
{the proper motion errors will be about five times as large, and parallax
errors will be about twice as large,} as the values given in Table 3; the errors
scale as $t^{-3/2}$ and $t^{-1/2}$, respectively. This error behavior is 
a strong independent argument for a survey lifetime of at least 10 years 
(c.f. \S 2).  





For comparison with Table 3, the SDSS-POSS proper motion measurements have an 
accuracy of $\sim$5 mas yr$^{-1}$ per coordinate at $r=20$ (Munn et al.~2004). Gaia
is expected to deliver parallax errors of 0.3 mas and proper motion errors of 
0.2 mas yr$^{-1}$ at its faint end at $r\sim20$ (Perryman et al.~2001). Hence, LSST will smoothly 
extend Gaia's error vs.\ magnitude curve 4 magnitudes fainter.


\subsection{             Data Products                    } 
\label{Sec:dp}

Data collected by the LSST telescope and camera will be automatically processed to {\em data products} -- catalogs, alerts,
and reduced images -- by the LSST Data Management system
(\S~\ref{sec:dm}). These products are designed to be sufficient to
enable a large majority of LSST science cases, without the need to
work directly with the raw pixels.  We give a high-level overview of
the LSST data products here; further details may be found in the LSST
Data Products Definition Document (Juri\'{c} et al. 2013). 

\vskip 1em

Two major categories of data products will be produced and delivered by LSST DM:
\begin{itemize}
\item {\bf Level 1 data products}, designed to support the discovery,
  characterization, and rapid follow-up of time-dependent phenomena
  (``transient science''). These will be generated continuously every
  observing night, by detecting and characterizing sources in images
  differenced against deep templates. They will include alerts to
  objects that were newly discovered, or have changed brightness or
  position at a statistically significant level. The alerts to such
  events will be published within 60   seconds of observation.\\ 
\\
In addition to transient science, Level 1 data products will support
discovery and follow-up of objects in the Solar System. Objects with
motions sufficient to cause trailing in a single exposure will be
identified and flagged as such when the alerts are broadcast. Those
that are not trailed will be identified and linked based on their
motion from observation to observation, over a period of a few
days. Their orbits will be published within 24 hours of
identification. The efficiency of linking (and thus the completeness
of the resulting orbit catalog) will depend on the final observing
cadence chosen for LSST, as well as the performance of the linking
algorithm (\S~\ref{Sec:NEOc}).
\item {\bf Level 2 data products} are designed to enable systematics- and flux-limited science, and will be made available in annual Data Releases. These will include the (reduced and raw) single-epoch images, deep coadds of the observed sky, catalogs of objects detected in LSST data, catalogs of sources (the detections and measurements of objects on individual visits), and catalogs of ``forced sources" (measurements of flux on individual visits at locations where objects were detected by LSST or other surveys). LSST data releases will also include fully reprocessed Level 1 data products, as well as all metadata and software necessary for the end-user to reproduce any aspect of LSST data release processing.\\
\\
A noteworthy aspect of LSST Level 2 processing is that it will largely
rely on {\bf multi-epoch model fitting}, or {\bf \em MultiFit}, to
perform near-optimal characterization of object properties. That is,
while the coadds will be used to perform object {\em detection}, the
{\em measurement} of their properties will be performed by
simultaneously fitting (PSF-convolved) models to single-epoch
observations. An extended source model -- a constrained linear
combination of two S\'ersic profiles -- and a point source model with
proper motion -- will generally be 
fitted to each detected object\footnote{For performance reasons, it is
  likely that only the point source model will be fitted in the most
  crowded regions of the Galactic plane.}.\\ 
\\
Secondly, for the extended source model fits, the LSST will
characterize and store the shape of the associated likelihood surface
(and the posterior), and not just the maximum likelihood values and
covariances. The characterization will be accomplished by sampling,
with up to $\sim$200 (independent) likelihood samples retained for
each object. For storage cost reasons, these samples
may be retained only for those bands of greatest interest for
weak lensing studies. 

\end{itemize}


While a large majority of science cases will be adequately served by
Level 1 and 2 data products, a limited number of highly specialized
investigations may require custom, user-driven, processing of LSST
data. This processing will be most efficiently performed at the 
LSST Archive Center, given the size of the LSST data set and the
associated storage and computational challenges. To enable such use
cases, the LSST DM system will devote the equivalent of 10\% of its
processing and storage capabilities to creation, use, and federation
of {\bf Level 3} (user-created) data products. It will also allow the
science teams to use the LSST database infrastructure to store and
share their results. 

To further enable user-driven Level 3 processing, the LSST software
stack has been explicitly architected with reusability and
extendability in mind, and will be made available to the LSST user
community (\S~\ref{sec:dmstack}). This will allow the LSST users to
more rapidly develop custom Level 3 processing codes, leveraging 15+
years of investment and experience put into LSST codes. In addition to
executing such customized codes at the LSST data centers, LSST users
will be able to run it on their own computational resources as well.\\

We have described that approximately 10\% of the observing time will
be devoted to mini-surveys that do not follow the LSST baseline
cadence (\S~\ref{Sec:minisurveys}). The data products for these
programs will be generated using the same processing system and
exhibit the same general characteristics of Level 1 and 2 data
products, but these data may be reduced on a somewhat different
timescale.  



\B{
\subsection{Data Mining Challenges}

The characterization (unsupervised machine learning) and classification (supervised machine learning) of 
massive, multivariate data catalogs such as those generated by the LSST are major research challenges for 
data-intensive astronomy (Tyson et al.~2008b; Ivezi\'{c} et al.~2008b;
Bloom et al.~2008; Borne 2008; Ivezi\'c et al.~2014). To address these questions, the statistics and machine-learning research 
communities are collaborating with LSST scientists to develop new algorithms that will enable the full 
scientific potential of the LSST, including:
\begin{itemize}
\item Rapid characterizations and probabilistic classifications for
  the million sources
          detected in difference images each night.
\item Identification of unusual classes of astronomical sources using outlier detection techniques that are 
          robust to noise and image processing defects.
\item Characterization of novel and unexpected behavior in the time domain from time series data.
\item Measurements of the clustering of stars and galaxies (including higher order statistics) using fast 
          algorithms for point processes.
\item The application of dimensionality-reduction techniques to determine important physical correlations 
          within large multi-variate catalogs.
\item Model or hypothesis testing that can verify existing (or generate new) astronomical hypotheses with 
          strong statistical confidence, using millions of training samples.
\end{itemize}

%The broad range of science that will benefit from statistically
%rigorous and computationally efficient algorithms has led to the
%creation of the Informatics and Statistical Science Research
%Collaboration for the LSST (see \S~\ref{Sec:community}) . This 
%collaboration's goal is to develop, implement, and validate data
%mining algorithms that will scale to the size and complexity of the
%LSST data.  
}




\begin{figure}
\hskip -0.8in
\includegraphics[width=1.5\hsize,clip]{panels1_2.pdf}
\caption{A comparison of $\sim7.5\times7.5$ arcmin$^2$ images of
the same area of sky (centered on $\alpha$=9$^h$ 20$\arcmin$ 47$\arcsec$ and 
$\delta$=30$^\circ$ 8$\arcmin$ 12$\arcsec$) obtained by the SDSS (top, $r<22.5$) and 
the Deep Lens Survey (bottom, $r<24.5$). These are gri composites,
colorized following Lupton et al.~(2004).  The depth gain for the bottom image
is mostly due to the lower surface brightness limit, which is also responsible 
for the apparent increase of galaxy sizes. LSST will obtain $\sim$100 $gri$ 
color images to the same depth ($\sim$200 for the $riz$ composites) of each point 
over half the Celestial sphere (18,000 deg$^2$, equivalent to 1.15 million $\sim7.5\times7.5$
arcmin$^2$ regions), and with better seeing. After their coaddition, the final 
image will be another $\sim3$ mag deeper (a faint limit of $r=27.5$ for point 
sources).} 
\label{Fig:panels1}
\end{figure}

\begin{figure}
\includegraphics[width=1.0\hsize,clip]{panels2.pdf}
\caption{A comparison of angular resolution for $20\times20$ arcsec$^2$ images obtained 
by the SDSS (top, median seeing of 1.5 arcsec) and expected from LSST (bottom,
seeing of 0.7 arcsec). The images show a lensed SDSS quasar (SDSS J1332+0347,
Morokuma et al.~2007); the bottom image was taken with Suprime-cam at Subaru. 
Adapted from Blandford et al.~(2008).} 
\label{Fig:panels2}
\end{figure}



\section{                EXAMPLES OF LSST SCIENCE PROJECTS                    }
\label{Sec:science}


The design and optimization of the LSST system leverages its unique capability 
to scan a large sky area to a faint flux limit in a short amount of time. 
The main product of the LSST system will be a multi-color $ugrizy$ image of about 
half the sky to unprecedented depth ($r\sim27.5$). For a comparison, the best 
analogous contemporary dataset is that of SDSS, which provides $ugriz$ images
of about a quarter of the sky to $r\sim22.5$, with twice as large seeing
(see Figs.~\ref{Fig:panels1} and \ref{Fig:panels2}). A major advantage of LSST 
is the fact that this deep sky map will be produced by taking hundreds of
shorter exposures (see Table 1). Each sky position within the main survey area 
will be observed about 1000 times, with time scales spanning seven orders of 
magnitude (from 30 sec to 10 years), and produce over {\it a trillion 
photometric measures} of celestial sources.

It is not possible to predict all the science that LSST data will enable.
We now briefly discuss a few projects to give a flavor of anticipated studies,
organized by the four science themes that drive the LSST design 
(although some projects span more than one theme). 
\B{For an in-depth discussion of LSST science cases, we refer the reader to the 
LSST Science Book.}
 
\vskip 0.3in
\subsection{Probing Dark Energy and Dark Matter }

A unique aspect of LSST as a probe of dark energy and dark matter is
the use of multiple cross-checking probes that reach unprecedented
precision (see Fig.~\ref{Fig:DEellipses}). Any given probe constrains
degenerate combinations of cosmological parameters, and each probe is
affected by different systematics, thus the combination of probes
allows systematics to be calibrated out and for degeneracies to be
broken.  Dark energy manifests itself in two ways.  The first is the
relationship between redshift and distance (the Hubble diagram), or
equivalently the expansion rate of the Universe as a function of
cosmic time.  The second is the rate at which matter clusters with
time.  
Structure formation involves a balance between
gravitational attraction of matter over-densities and the rapid
expansion of the background.  Thus, quantifying the rate of growth of
structures from early times until the present provides additional
tests of the energy contents of the Universe and their interactions.

The joint analysis of LSST weak lensing and BAO is
particularly powerful in constraining the dynamical behavior of dark
energy, i.e., how it evolves with cosmic time or redshift (Hu \& Jain
2004; Zhan 2006).  By
simultaneously measuring the growth of large-scale structure, and
luminosity and angular distances as functions of redshift (via weak
lensing, BAO, SN, and cluster counting), LSST data can reveal whether
the recent cosmic acceleration is due to dark energy or modified
gravity (Lue, Scoccimarro \& Starkman 2004; Knox, Song \& Tyson 2006;
Ishak, Upadhye \& Spergel 2006; Jain \& Zhang 2008; Oguri \& Takada
2011; Jain et al.~2013; Weinberg et al.~2013).

Over a broad range of accessible redshifts, the simple linear model
for the dark energy equation of state ($w = w_0 + w_a(1-a)$) is a poor representation of more
general dark energy theories. Barnard et al.~(2008) have shown that in a high-dimensional dark energy model space, 
LSST data could lead to a hundred- to thousand-fold increase in precision over 
precursor experiments about to be undertaken, thereby confirming its status as 
a premier Stage IV experiment in the sense of Albrecht et al.~(2006). 
%Fits of multi-parameter dark energy models to the complementary LSST
%probes of dark energy will constrain alternative models of gravity.

The power and accuracy of LSST dark energy and dark matter probes are
a result of the enormous samples that LSST will have, including 
several billion galaxies and millions of Type Ia
supernovae. At $i < 25.3$ (SNR${}>20$ for point sources), the
photometry of galaxies will be of high enough quality to give
photometric redshifts with an accuracy ($\sigma/(1+z)$) of 1-2\% (only
10\% of the sample will have redshift errors larger than 4\%); the
sample to this limit will include about four billion galaxies.  At a
slightly brighter cut, there will be about 40 galaxies arcmin${-2}$
with shapes measured well enough for weak lensing measurements.  
%The nominal high-SNR galaxy sample defined by $i<25.3$ (SNR$>$20 for point
%sources) will include four billion galaxies (56 arcmin$^{-2}$; the surface density of galaxies 
%used for weak lensing analysis will be about 40 arcmin$^{-2}$) with a mean 
%photometric redshift accuracy of 1-2\% (relative error for $1+z$), and with 
%only 10\% of the sample with redshift errors larger than 4\%. 
The median redshift for 
this sample will be $z\sim$1.2, with the third quartile at $z\sim2$. 
%For a subsample of 2 billion galaxies further constrained by flux limits in the 
%$g$ and $z$ bands, the photometric redshift errors will be about two times 
%smaller. 
%I SIMPLY DON'T BELIEVE THIS.
It will be possible to further improve photometric redshift calibration
by cross-correlating the photometric sample with redshift surveys of
galaxies and quasars in the same fields (Newman 2008; Matthews \&
Newman 2010; Menard et al.~2013).  


\begin{figure}
%\includegraphics[width=1.0\hsize,clip]{DEellipses1.pdf}
%\includegraphics[width=0.95\hsize,clip]{DEellipses2.pdf}
\includegraphics[width=1.0\hsize,clip]{cswb.pdf}
\caption{
Constraints on the dark energy equation of state ($w = w_0 +
w_a(1-a)$) from LSST cosmological probes.  The various ellipses assume
constraints from BAO (dashed line), cluster counting (dash-dotted line), 
supernovae (dotted line), WL (solid line), joint BAO and WL 
(green shaded area), and all probes combined (yellow shaded area).
The BAO and WL results are based on galaxy--galaxy, galaxy--shear,
and shear--shear power spectra only. 
Adding other probes such as strong lensing time delay
and higher-order galaxy and shear statistics will further improve 
the constraints.} 
\label{Fig:DEellipses}
\end{figure}



\begin{figure}
\includegraphics[width=1.0\hsize,clip]{cls.pdf}
\caption{The lensing shear auto power spectra constructed from 5 redshift bins 
($l$ is the multipole moment of the distribution on the sky). Only the 5 auto-power 
spectra of each redshift bin among the available 15 cospectra are displayed, and the 
solid curves show the predictions for the concordance $\Lambda$CDM model. The boxes 
show the expected 1-$\sigma$ measurement error (in bins of $\Delta l/l=0.2$) from the full LSST 
10-year survey due to the sample (i.e., cosmic) variance which dominates at about 
$l< 100$, and intrinsic ellipticities which dominate at $l> 1000$ (Zhan 2006).} 
\label{Fig:wlPk}
\end{figure}


\begin{figure}
\includegraphics[width=1.0\hsize,clip]{bao.pdf}
\caption{Simulations of the ratio of the measured galaxy power spectrum to 
a featureless reference power spectrum in various redshift bins (shifted vertically 
for clarity), for the full LSST survey using conservative photometric redshift
errors. The several peaks visible in each curve are the signature of baryon acoustic 
oscillations (Eisenstein et al.~2005; Cole et al.~2005; Anderson et al.~2014). LSST will measure the 
angular diameter distance over this redshift range with an accuracy of $\sim1$\% 
(Zhan \& Knox 2006).} 
\label{Fig:bao}
\end{figure}



\begin{figure}
%\includegraphics[width=1.0\hsize,clip]{dgcon.pdf}
\includegraphics[width=1.0\hsize,clip]{dges.pdf}
\caption{Marginalized $1\sigma$ errors on the comoving distance 
(open triangles) and growth factor (open circles) parameters from 
the joint analysis of LSST BAO and WL (galaxy--galaxy, galaxy--shear,
and shear--shear power spectra) with a 
conservative level of systematic uncertainties in the photometric redshift error 
distribution and additive and multiplicative errors in the shear and 
galaxy power spectra. The maximum multipole used for WL is 
2000, and that for BAO is 3000 [with the additional requirement that 
$\Delta_\delta^2(\ell/D_{A};z) < 0.4$].
The growth parameters
%, $G_0 \ldots G_{14}$, 
are evenly spaced in 
$\log(1+z)$ between $z = 0$ and 5, and the distance parameters
%$D_1 \ldots D_{14}$, 
start at $z_1 = 0.14$.
% (see text for details).
The error of each distance (growth) parameter is marginalized 
over all the other parameters including growth (distance) parameters. The joint constraints on 
distance are relatively insensitive to the assumed systematics
(Zhan, Knox \& Tyson 2009).} 
\label{Fig:bao2}
\end{figure}


The main LSST observables in the context of dark energy and matter will be

\begin{itemize}
\item Measurements of the two-point auto- and cross-correlation of
  shear across multiple redshift bins (Fig.~\ref{Fig:wlPk}). There will be 
over 50 of these auto and cross power spectra constructed from the
shape catalog with excellent photometric redshifts
%correlation of shear of a galaxy at one redshift bin with another galaxy in 
%a different redshift bin, averaged over all pairs of billions of
%galaxies; 
(Jain \& Taylor 2003).
\item Higher-order shear and galaxy statistics that can improve dark energy 
constraints and provide self-calibration of various systematics (Takada \& Jain 2004; 
Dolney, Jain \& Takada 2006; Huterer et al.~2006). They are also probes of both
primordial non-Gaussianities and those caused by non-linear structure.
\item Similarly, weak lens magnification tomography (Morrison et al.~2012) offers a 
complementary probe of a mix of cosmic geometry and growth of dark matter structure.
\item The galaxy-mass correlation function of billions of galaxies (as a function of their 
photometric redshift) probes the growth of dark matter large-scale structure and is a 
diagnostic of the underlying cosmology (Choi et al.~2012). Combining this with the 
galaxy-velocity power spectrum estimated from spectroscopic observations of millions of galaxies 
could probe General Relativity at high redshift (Reyes et al.~2007).
\item Baryon Acoustic Oscillations in the galaxy angular correlation functions.
The standard ruler of the sound horizon at decoupling which is imprinted on the mass 
distribution at all redshifts provides a direct way of measuring the angular diameter
distance as a function of redshift (Fig.~\ref{Fig:bao}; Eisenstein, Hu \& Tegmark 1998;
Cooray et al.~2001; Blake \& Glazebrook 2003; Hu \& Haiman 2003; Linder 2003; Seo \& 
Eisenstein 2003). LSST photo-z BAO will achieve percent level precision on the angular 
diameter distance at $\sim$10 redshifts logarithmically spaced between z = 0.4 to 3.6 
with this CMB-calibrated standard ruler. When combined with CMB 
and weak lensing (WL) cosmic shear, this combination of probes yields tight constraints on the 
dynamical behavior of dark energy (Fig.~\ref{Fig:bao2}). In particular, high-redshift BAO data can break 
the degeneracy between curvature and dark energy, and constrain $\Omega_k$ to within 
0.001.
%, which is about ten times better than the most accurate current
%result based on WMAP and SDSS data (Komatsu et al.~2009; Percival et
%al.~2010).  
%I TOOK THIS OUT; THIS CONSTRAINT IS COMPARABLE TO THAT OF PLANCK.
\item Supernovae. The two LSST observing programs are complementary: the main survey will 
obtain light curves in six bands and photometric redshifts of about 300,000 Type
Ia supernovae per year (permitting a search for a ``third parameter'' which,
if not understood, may introduce systematic error in luminosity evolution; 
Wood-Vasey et al.~2007), and the rapid sampling ``mini-survey'' of selected
areas will yield well-sampled light curves of tens of thousands of supernovae
to a limiting redshift beyond one (leading to an independent test of dark energy 
dynamics; Riess et al.~2007). 
\item Unique constraints on anisotropy of cosmological parameters over the sky using
SN, WL and BAO, which will test fundamental cosmological assumptions of homogeneity and isotropy.
For example, WL$+$BAO and SN can separately probe anisotropy of dark energy (Zhan, Knox \& Tyson 2009).
\item The shape of the power spectrum of dark matter fluctuations measured by
the LSST weak lensing maps will constrain the sum of neutrino masses with an accuracy 
of 0.04 eV or better (Cooray 1999; Song \& Knox 2004; Hannestad, Tu \& Wong 2006). 
The current best limit, as derived from a combination of WMAP, the SDSS power spectrum, 
SN, and the Ly$\alpha$ forest, is 0.17 eV (Seljak et al.~2006).
\item Hundreds of millions of galaxy-galaxy lenses will provide the needed statistics to probe dark matter 
halo profiles and substructure (Mandelbaum et al.~2006). The image fluxes in several thousand well-measured
strongly lensed quasars will enable constraints of the dark matter mass function on small scales (Dalal \& Kochanek 2002).
\item The abundance of galaxy clusters as a function of mass and redshift is sensitive to cosmological parameters
(SciBook, Ch.~13). LSST will produce a large catalog of clusters detected through their member galaxy population 
out to and beyond redshift unity.  In addition, LSST will identify optical counterparts and provide deep optical
imaging for clusters detected in other wavebands (e.g., Staniszewski et al.~2009). 
\item The distribution of strongest WL shear peaks with redshift is
  closely related to the galaxy cluster abundance, but has the
  potential to be a more robust and more directly calibratable statistic. This is a simultaneous 
probe of the universal mass function and the growth of structure, and provides a useful 
constraint on dark energy (Wang et al.~2005). LSST will find over 200,000 such 
shear peaks on galaxy cluster scales. 
\item LSST will discover several hundred galaxy clusters that produce multiple-image lenses of background objects.
Cluster mass reconstruction based on the multiple image positions
 can probe the cluster inner mass profile, and can provide a separate test of cosmology, especially 
in cases with strongly lensed background objects at different redshift (Porciani \& Madau 2000; Oguri \& Kawano 2003).
\item Time delays of galaxy-scale lensed quasars will allow one to measure Hubble's constant 
(e.g. Suyu et al.~2010) in hundreds or thousands of systems; sub-percent level precision in 
$H(z)$ should be achievable (Coe \& Moustakas 2009), providing a further independent dark energy probe. 
Time delays for quasars multiply lensed by clusters as a function of redshift are an independent test
of dark energy (Kundi\'{c} et al.~1997). The natural timescale (many months to years) is well matched
to the LSST survey (Oguri \& Marshall 2010). 
\end{itemize}

\subsection{Taking an Inventory of the Solar System}


The small bodies of the Solar System, such as main-belt asteroids,
the Trojan populations of the giant planets and the Kuiper Belt objects,
offer a unique insight into its early stages because they provide
samples of the original solid materials of the solar nebula. 
Understanding these populations, both physically and in their number 
and size distribution, is a key element in testing various theories of
Solar System formation and evolution. 

The baseline LSST cadence will result in orbital parameters for several
million objects; these will be dominated by main-belt asteroids, with 
light curves and multi-color photometry for a substantial fraction of detected objects. 
This dataset will yield 10 to 100 times more objects than are currently
available with orbits, colors, and variability information. LSST is capable of reaching the Congressional target 
completeness of 90\% for PHAs larger than 140 m (\S~\ref{Sec:NEOc}), and will detect over 30,000 TNOs 
brighter than $r\sim24.5$ using its baseline cadence. LSST will be capable
of detecting objects like Sedna to beyond 100 AU, thus enabling {\it in situ} exploration
far beyond the edge of the Kuiper belt at $\sim$50 AU. Because most of these
objects will be observed several hundred times, accurate orbital elements, 
colors, and variability information will also be available.


\begin{figure}
\includegraphics[width=1.0\hsize,clip]{asteroids.pdf}
\caption{An example of color-based asteroid taxonomy. The figure
shows the distribution of asteroids in the proper semi-major axis vs. $\sin(i)$
plane for 45,000 asteroids with colors measured by SDSS (Parker et al.~2008). 
The color of each dot is representative of the object's color.
Note the strong correlation between asteroid families (objects in distinct regions
of the plane) and colors. There are
at least five different taxonomic types distinguishable with SDSS measurements;
LSST color measurements of asteroids will be more than twice as accurate
and will increase the number of objects by roughly two orders of magnitude.} 
\label{Fig:asteroids}
\end{figure}

\begin{figure}
\includegraphics[width=1.0\hsize,clip]{Af9.pdf}
\caption{The LSST detection limits for distant Solar System objects as
  a function of distance.
Moving objects with diameters as small as 100m in the main asteroid belt and 
100km in the Kuiper Belt (TNOs) will be detected in individual visits,
depending on the albedo. Specialized deeper observations  
(see \S~\ref{Sec:minisurveys}) will detect TNOs as small as 10 km. Adapted from 
Jones et al.~(2007).} 
\label{Fig:Af9}
\end{figure}



The following are some examples of the LSST science opportunities in
Solar System science: 
\begin{itemize}
\item Studies of the distribution of orbital elements for over 5 million main-belt 
asteroids as a function of color-based taxonomy (see Fig.~\ref{Fig:asteroids})
and size; size distributions of asteroid families (Parker et al.~2008) and their
correlations with age (Jedicke et al.~2004; Nesvorn\'{y} et al.~2005); dynamical 
effects (Bottke et al.~2001); and studies of object shapes and spin states using 
light curve inversion techniques (Pravec \& Harris 2000; Durech et al.~2009). 
\item Studies of transient activity in asteroids (active asteroids or main belt comets, 
Hsieh \& Jewitt 2006),  due to mass loss with associated extended
morphology.  Only a few such objects are currently known (Jewitt et
al.~2011; Jewitt 2012); LSST
will increase the sample of such objects to $\sim$100. 
\item Studies of the distribution of orbital elements for about 100,000 NEOs as a 
function of color and size (Rabinowitz 1993; Dandy, Fitzsimmons \& Collander-Brown 2002);
correlations with the analogous distributions for 
main-belt objects, and studies of object shapes and structure using light curves. 
\item Studies of the distribution of orbital elements for close to 300,000 Jovian Trojan 
asteroids as a function of color and size (Jewitt, Trujillo \& Luu 2000; Yoshida \& 
Nakamura 2005; Szabo et al.~2007); the search for dynamical families (Kne\v{z}evi\'{c} 
\& Milani 2005); studies of shapes and structure using light curves. 
\item Studies of the distribution of orbital elements for about 30,000 TNOs (see 
Fig.~\ref{Fig:Af9}) as a function of color and size; the search for dynamical families
(Marcus et al.~2011); studies of shapes and structure using light curves (Duncan \& Levison 1995;
Trujillo, Jewitt \& Luu 2001; Gladman et al.~2001; Bernstein et al.~2004; 
Elliot et al.~2005; Jones et al.~2006; Doressoundiram et al.~2007).
\item An unbiased and complete census of both Jupiter-family and Oort-cloud
comets; six-band sub-arcsecond spatial profiles to a faint surface brightness
limit; temporally resolved activity (Lowry et al.~1999; A'Hearn 2004). 
\item Searching for objects with perihelia out to several hundred AU. For example, an object 
like Sedna (Brown, Trujillo \& Rabinowitz 2004) would be detectable at 130 AU. This will result 
in a much larger, well-understood sample of inner Oort Cloud objects like Sedna and 2012 VP113
 (Trujillo \& Sheppard 2014).  Studying the distribution of their orbits (in particular including any 
clustering in the argument of perihelion) will constrain models of the birth environment of the 
Solar System and any potential nearby stellar passages (Kaib \& Quinn 2008; Levison et al.~2010;
Brasser et al.~2012). 
\item Mapping the propagation of interplanetary coronal mass ejections using induced 
 activity in a large sample of comets at different heliocentric distances
(SciBook Ch.~5).
\end{itemize}



\subsection{ Exploring the Transient Optical Sky }


\begin{figure}
\hskip -0.1in
\vskip -0.1in
%\includegraphics[width=1.1\hsize,clip]{shri2.pdf}
\includegraphics[width=1.0\hsize,clip]{taumv.pdf}
%\vskip -0.1in
\caption{The phase space of cosmic explosive and eruptive transients as
represented by their absolute $V$ band peak brightness and the 
event timescale (adapted from Kulkarni et al.~2007). LSST will open up large 
regions of this phase space for systematic exploration by extending
time-volume space over 100 imes over existing surveys.} 
\label{Fig:shri}
\end{figure}

Time domain science will greatly benefit from LSST's unique capability to simultaneously
provide large area coverage, dense temporal coverage, accurate color information,
good image quality, and rapid data reduction and classification. Since LSST extends 
time-volume space more than one hundred times over current surveys (e.g., Morokuma et al.~2008), 
the most interesting science may well be the discovery of new classes of objects.  
There are many projects that LSST data products will enable:

\begin{itemize}
\item Detection and measurement of gamma-ray burst afterglows and transients 
      (e.g., Zhang \& M\'{e}sz\'{a}ros 2004; Zhang et al.~2006; Kann et al.~2010) to high redshift ($\sim$7.5).
  \item Studies of optical bursters (those varying faster than 1 mag hr$^{-1}$) to $r\sim25$ mag. 
\item Gravitational micro-lensing in the Local Group and beyond (de Jong, Kuijken \& H\'{e}raudeau 2008).
\item Studies of known and unusual SN populations and parametrization of their light curves 
         (e.g., Hoeflich, Wheeler \& Thielemann 1998; Wang et al.~2003; Howell et al.~2007;
          Kowalski et al.~2008; Hicken et al.~2009; Foley~2012).
\item Studies of dwarf novae, including their use as probes of stellar populations and 
      structure in the Local Group (Neill \& Shara 2005; Shara 2006; Shen \& Bildsten 2009).
\item A deep search for new populations of novae and supernova progenitors 
      (Smartt 2009; Thompson et al.~2009; Smith et al.~2011; see Fig.\ref{Fig:shri}). 
\item Search for stellar tidal disruptions by nuclear supermassive black
       holes (Evans \& Kochanek 1989; Gezari et al.~2008; Strubbe \& Quataert 2009; Bloom et al.~2011; Gezari 2012), 
       as well as binary supermassive black holes in the in-spiral phase (e.g., Cuadra et al.~2009).
       Measurements of rates as function of galaxy type, redshift, and level of nuclear activity
       to understand the apparent diversity of these events in terms of  total power, effective temperature, 
       and jet launching efficiency. 
\item Discovery and characterization of thousands of hot Jupiters 
  in exoplanetary systems via the transit method (Wright et al.~2012). 
LSST will extend the extrasolar planet census to larger distances within the Galaxy, thus enabling detailed studies 
of planet frequency as a function of stellar metallicity and parent population (e.g., Hartman et al.~2009; Bayliss \& 
Sackett 2011). The out-of-transit variability of exoplanet host stars will also provide characterization of the system 
via flaring behavior and stellar age via gyrochronology, the latter helping constrain theories of tidal evolution and 
migration in giant planets. 
\item A study of quasar variability using accurate, multicolor light
  curves for a few million  
quasars, leading to constraints on the accretion physics and nuclear environments (de Vries, Becker 
\& White 2003; Vanden Berk et al.~2004, MacLeod et al.~2010). Relations between quasar variability 
      properties and luminosity, redshift, 
      rest-frame wavelength, time scale, color, radio-jet emission, black-hole 
      mass, and Eddington-normalized luminosity will be defined with massive
      statistics, including the potential to detect rare but important events such as 
      jet flares and obscuration events. Microlensing events will also be monitored in the $\sim$4000 gravitationally-lensed 
      quasars discovered by LSST and used to measure the spatial structure of quasar accretion disks.
\item The superb continuum light curves of AGN will enable economical ``piggyback" 
      reverberation-mapping efforts using spectroscopy of emission lines (Doron \& Eliran 2012). These results 
      will greatly broaden the luminosity-redshift plane of reverberation-mapped AGNs
      with black-hole mass estimates. For LSST data alone, the inter-band continuum lags 
      will provide useful structural information. 
\item Optical identification for transients and variables detected in other electromagnetic wavebands, 
from gamma rays to radio. Examples include optical and gamma ray variability in blazars (Hovatta et al.~2014), 
radio transients associated with tidal disruption flares (Giannios \& Metzger 2011), and radio counterparts to 
supernovae and GRBs (Gal-Yam et al.~2006). Deep optical observations with LSST may also help illuminate the 
nature of the still-mysterious Lorimer Bursts (Lorimer et al.~2007; Thornton et al.~2013).
\item Optical identification of counterparts to non-electromagnetic sources, such as gravitational waves and
neutrino detections (LIGO\footnote{http://www.ligo.caltech.edu}, ICECUBE\footnote{http://icecube.wisc.edu}). 
LSST's unique ability to characterize the faint variable sky over large areas will be important not only for detection 
of associated sources but for eliminating potential false positives (Nissanke, Kasliwal \& Georgieva 2013; 
Metzger \& Berger 2012).  
\end{itemize}



\subsection{Mapping the Milky Way }

The LSST will map the Galaxy in unprecedented detail, and by doing so revolutionize the fields of Galactic 
Astronomy and Near-field Cosmology. The great detail with which the Milky Way can be studied complements 
the statistical power of extra-galactic observations.  The overarching goal of near-field cosmology is to use 
spatial, kinematic, and chemical data sets of stars to reveal the structure and evolution history of the Milky Way
and its environment. LSST will reveal this fossil record in great detail and provide a Rosetta Stone for extragalactic 
astronomy by setting the context within which we interpret these
broader data sets. Moreover, different candidate supersymmetric
particle dark matter models predict different mass clustering on small
scales, and thus different mass functions for subhalos of the Milky
Way.  Thus the LSST census of faint satellites and stellar streams in
the halo will offer a unique means to constrain the 
particle nature of dark matter.  


\begin{figure}
%\includegraphics[width=1.0\hsize,clip]{smolcic.pdf}
%\hskip -1.0in
\includegraphics[width=1.0\hsize,clip]{MarlaUGR.pdf}
%\vskip -0.3in
\caption{The $g-r$ vs. $u-g$ color-color diagram for about a million points sources 
from SDSS Stripe 82 area. Accurate multi-color photometry 
contains information that can be used for source classification and determination of 
detailed stellar properties such as effective temperature and metallicity. LSST will 
enable such measurements for billions of stars.} 
\label{Fig:FeH}
\end{figure}


The LSST will produce a massive and exquisitely accurate photometric and astrometric data set for about 20 billion 
Milky Way stars. The coverage of the Galactic plane will yield data for numerous star-forming 
regions, and the $y$ band data will penetrate through the interstellar dust layer. Photometric metallicity 
measurements (see Figs.~\ref{Fig:FeH} and \ref{Fig:FeH3}) will be available for about 200 million main-sequence 
F/G stars which will sample the halo to distances of 100 kpc (Ivezi\'{c} et al.~2008a; An et al.~2013). No other 
existing or planned survey will provide such a powerful dataset to
study the outer halo: Gaia
is flux limited at $r=20$, and the Dark Energy Survey (Rossetto et al.~2011) and Pan-STARRS both 
lack observations in the $u$ band, necessary for estimating metallicity. The LSST in its standard surveying mode will 
be able to detect RR Lyrae variables (pulsating stars and standard candles) and classical novae (exploding stars 
and standard candles) at a distance of 400 kpc and hence explore the extent and structure of our  halo out to 
half the distance to the Andromeda galaxy. Thus, the LSST will enable studies of the distribution of main-sequence 
stars beyond the presumed edge of the Galaxy's halo (see Fig.~\ref{Fig:halo}), of their metallicity distribution
throughout most of the halo, and of their kinematics beyond the thick disk/halo boundary. LSST will also obtain 
direct distance measurements via trigonometric parallax below the hydrogen-burning limit for a representative
thin-disk sample.

% the SkyMapper (Keller et al.~2001) which is flux limited to $r=22.6$
% (but will nicely complement SDSS in the Southern hemisphere),

In addition to the study of hydrogen burning stars, LSST will uncover the largest sample of stellar remnants to date. 
Over 97\% of all stars eventually exhaust their fuel and cool to become white dwarfs. Given the age of the Galactic 
halo, a significant fraction of the mass in this component may reside in these remnant stars (e.g., Alcock et al.
2000; Tisserand et al.~2007) and therefore their discovery directly constrains the Galactic mass budget.  These large 
populations of disk and halo white dwarfs will provide unprecedented constraints on the luminosity function of
these stars, which will directly yield independent ages for the Galactic disk and halo (e.g., through the initial-final mass
relation, Kalirai et al.~2008).

The sky coverage of LSST naturally targets both field stars and star clusters.  To date, no systematic survey of the stellar 
populations of Southern hemisphere clusters has been performed (e.g., such as the CFHT Open Star Cluster Survey, or 
the WIYN Open Star Cluster Survey in the North; Kalirai et al.~2001; Mathieu 2000).  Multiband imaging of these co-eval, 
co-spatial, and iso-metallic systems will provide vital insights into fundamental stellar evolution.  For example, the depth 
of LSST will enable construction of  luminosity and mass functions for nearby open clusters down to the hydrogen burning
limit and beyond.  Variations in the initial mass function will be studied as a function of environment (e.g., age and metallicity).  
The wide-field coverage will also allow us to track how the stellar populations in each cluster vary as a function of radius, 
from the core to beyond the tidal radius. Fainter remnant white dwarfs will be uncovered in both open and globular clusters
(the nearest of which are all in the South), thereby providing a crucial link to the properties of the now evolved stars in each 
system.



\begin{figure}
%\includegraphics[width=1.0\hsize,clip]{panelsLSST.pdf}
\includegraphics[width=0.49\textwidth,clip]{panelsLSST.png}
\caption{
The median metallicity map for 2.5 million main-sequence F-type stars within 10 kpc 
from the Sun (adapted from Ivezi\'{c} et al.~2008a). The metallicity is estimated using 
$u-g$ and $g-r$ colors measured by SDSS. The position and size of the mapped 
region, relative to the rest of the Milky Way, is illustrated in the top right 
corner, where the same map is scaled and overlaid on an image of the Andromeda 
galaxy. The gradient of the median metallicity is essentially parallel
to the $Z$ axis, except in the Monoceros stream region, as marked. LSST 
will extend this map out to 100 kpc, using a sample of over 100 million 
main-sequence F stars.} 
\label{Fig:FeH3}
\end{figure}


\begin{figure}
\hskip -0.5in
\includegraphics[width=1.3\hsize,clip]{BethMWhalo.pdf}
\caption{A predicted spatial distribution of stars out to 150 kpc from the center of the Milky Way, 
from Bullock and Johnston (2005).  LSST will resolve main sequence turnoff stars out to 300 kpc, ten times 
more volume than shown here, enabling a high-fidelity spatial map over the entire observed virial volume.  
Overlaid on this prediction is the observed SDSS stellar number density map based on main sequence stars 
with $0.10 < r-i < 0.15$ (Juri\'{c} et al.~2008).  This map extends up to $\sim$ 20 kpc from the Sun, with 
the white box showing a scale of 20 kpc across and the left side aligned with the Galactic center.  
The revolutionary Galaxy map provided by SDSS is only complete to $\sim$40 kpc, or only $\sim$1\% of
the virial volume.  However, the outermost reaches of the stellar halo are predicted to bear the most unique 
signatures of our Galaxy's formation (Johnston et al.~2008;  Cooper et al.~2010).   LSST will be the only survey 
capable of fully testing such predictions.}
\label{Fig:halo}
\end{figure}

 

In summary, the LSST data will revolutionize studies of the Milky Way and the entire Local Group. We list a few specific 
Galactic science programs that LSST will enable:

\begin{itemize}
\item High-resolution studies of the distribution of stars in the outer halo
          in the six-dimensional space spanned by position, metallicity and proper
          motions (e.g., Girard et al.~2006; Bell et al.~2008; Juri\'{c} et al.~2008;
          Ivezi\'{c} et al.~2008a; Bond et al.~2010).
\item The most complete search possible for halo streams, Galaxy satellites and intra-Local Group 
          stars (e.g., Belokurov et al.~2007a; Walsh, Willman, \& Jerjen 2009; Bochanski et al.~2014).
\item Deep and highly accurate color-magnitude diagrams for over half of the known
          globular clusters, including tangential velocities from proper motion
          measurements (An et al.~2008; Casetti-Dinescu et al.~2007).
\item Mapping the metallicity, kinematics and spatial profile of the Sgr dwarf tidal
          stream (e.g., Ibata et al.~2001; Majewski et al.~2003; Law, Johnston \& Majewski 2005; Belokurov et al.~2014) 
          and the Magellanic stream (Zaritsky et al.~2004).
\item The measurement of the internal motions of Milky Way dwarf 
          galaxies, thereby constraining their density profiles and
	  possibly the nature of dark matter (e.g., Walker \&
	  Pe{\~n}arrubia 2011). 
\item Detailed constraints on the formation and evolution of the populations within the Galactic Bulge, as traced by the spatial 
          distribution, motion, and chemistry of $\sim$10$^{7-8}$ of its stars (e.g., Hill et al.~2011; Ness et al.~2014).
\item Studies of the clumpiness of the gravitational potential in the Galaxy using
          fragile wide-angle binaries selected with the aid of trigonometric and
          photometric parallaxes, and common proper motion (e.g., Yoo, Chanam\'{e} \& Gould 2004; Longhitano \& Binggeli 2010).
\item Detailed studies of variable star populations; 2\% or better accurate
          multicolor light curves will be available for a sample of at least 50
          million variable stars (Sesar et al.~2007), enabling studies of
          cataclysmic variables, eclipsing binary systems, and rare types of variables.
\item Discovery of rare and faint high proper motion objects: probing the
          faint end of the stellar mass function (Lepine 2008; Finch et al.~2010), and searching for
          free-floating planet candidates (Lucas \& Roche 2000; Luhman 2014).
\item Direct measurement of the faint end of the stellar luminosity function
          using trigonometric parallaxes (Reid et al.~2002) and a complete census of the
          solar neighborhood to a distance of 100 pc based on trigonometric parallax measurements for objects as faint as 
          $M_r=17$ ($\sim$L5 brown dwarfs). For example, LSST will deliver 10\% or better distances for a sample of about 2,500 stars
          with 18$<M_r<$19. % Comment: (note there are about 400 brown dwarfs with good parallaxes and Gaia will see about 1000, 
                                          % but most of these will be early L).
\item The separation of halo M sub-dwarfs from disk M dwarfs, using the $z-y$ color which is sensitive to their rich molecular band 
          structure (West et al.~2011; Bochanski et al.~2013).
\item Studies of white dwarfs using samples of several million objects, including the determination of the halo white dwarf luminosity 
          function (SciBook Ch.~6). 
\item Measurements of physical properties of stars using large samples of eclipsing binary stars (Stassun et al.~2013). 
\item High-resolution three-dimensional studies of interstellar dust using 5-color
          SEDs of main sequence stars (Planck Collaboration 2011; Berry et al.~2012; Green et al.~2014).
%\item Planetary transits: the data set may include a mini survey of 600
%         deg$^2$ of sky which will collect about 40 hour-long sequences of 200
%          observations each over a 4-month period (see \S~\ref{Sec:minisurveys}).
%          There would be an additional 800 observations over 10 years of these
%          same fields as part of the main survey. With about 10 million or more stars
%          in the sample (depending on where the fields are placed), this would
%          provide excellent base material to study the planet frequency as a
%          function of stellar type, metallicity, and distance from the Galactic plane
%          (e.g., Charbonneau et al.~2000; Konacki et al.~2003).
\item A census of AGB stars in the Galaxy by searching for resolved envelopes and optical  identifications of IR counterparts
         (e.g., from the WISE survey), and by using long-term variability and color selection (Ivezi\'{c} 2007c).
\item A complete census of faint populations in nearby star forming regions using
          color and variability selection (e.g., Briceno et al.~2005).
\end{itemize}


\begin{figure}
\vskip -1in
\includegraphics[width=1.0\hsize,clip]{musycSDSS.pdf}
\vskip -1.9in
\includegraphics[width=1.0\hsize,clip]{musyc.pdf}
\vskip -1in
\caption{
A comparison of an SDSS image (2$\times$4 arcmin$^2$ $gri$ composite) showing a typical galaxy at 
a redshift of $\sim$0.1 (top) with a similar $BVR$ composite image of the same field obtained by the MUSYC survey (bottom; 
Gawiser et al.~2006). The MUSYC image is about 4 mag deeper than the SDSS image (and about 1 mag less deep 
than the anticipated LSST 10-year coadded data). Note the rich surface brightness structure seen in the MUSYC 
image that is undetectable in the SDSS image.} 
\label{Fig:musyc}
\end{figure}


\subsection{  Additional Science Projects}

The experience with any large survey (e.g., SDSS, 2MASS, GALEX, to name but a 
few) is that much of their most interesting science is often unrelated to 
the main science drivers, and is often unanticipated at the time the survey is 
designed. LSST will enable far more diverse science than encompassed by the 
four themes that drive the system design. We list a few anticipated major 
programs.

\begin{itemize}
\item Detailed studies of galaxy formation and evolution using their distribution in 
luminosity-color-morphology space as a function of redshift. For example, LSST will 
enable studies of the rest-frame UV emission, similar to those based on GALEX data 
for local galaxies, to a redshift of $\sim$2 for an unprecedentedly large number of
galaxies. These studies project onto many axes:
\begin{itemize}
  \item the evolution of the galaxy luminosity function with redshift, as a function of 
        morphology and color;
  \item the evolution of the galaxy color distribution over a wide range of rest-frame 
        wavelengths, and as a function of luminosity and morphology; 
  \item bulge-disk decomposition, as a function of luminosity and color, over 
        a large redshift range; 
  \item detailed distribution of satellite galaxies in luminosity-color-morphology space 
        as a function of luminosity, color, and morphology of the primary galaxy; 
  \item correlations of luminosity, color and morphology with local environment, and
        as a function of redshift (see  Figs.~\ref{Fig:musyc} and \ref{Fig:cowan});
  \item the properties of galaxy groups as a function of cosmic time.
\end{itemize}
\item AGN census to very faint luminosity and large redshift limit
  (Ivezi\'{c} et al.~2013), yielding 10-40 million objects (see Fig.~\ref{Fig:panels3}). By reaching substantially further 
      down the AGN luminosity function over a very large solid angle, LSST data 
      will test evolutionary cosmic downsizing scenarios, and lead 
      to a much clearer understanding of black-hole growth during the first Gyr. For
      example, LSST should discover $\sim$1000 AGNs at $z\sim6-7.5$,
      representing a dramatic increase over present samples 
      (Brandt et al.~2007; see also SciBook Ch.~10).
\item LSST data will allow good constraints on AGN lifetimes, or at least the timescales over which 
         they make distinct accretion-state transitions, due to large sample
         size and survey lifetime (e.g., Martini \& Schneider 2003).
\item The first wide field survey of ultra low surface brightness galaxies, with 
      photometric redshift information. The currently available samples are highly
      incomplete, especially in the Southern Hemisphere (see Fig.~7 in Belokurov et al.~2007a). 
\item Search for strong gravitational lenses to a faint surface
  brightness limit (e.g., Bartelmann et al.~1998; Tyson, Kochanski \&
  Dell'Antonio 1998; Belokurov et al.~2007b), which can be used to
  explore the dark matter profiles of galaxies (e.g., Treu et
  al.~2006).
\end{itemize}


\begin{figure}
\includegraphics[width=1.0\hsize,clip]{cowan.pdf}
\caption{A comparison of the distribution of galaxies in  
luminosity--color--density space measured by SDSS (left) and a model based
on the Millennium simulation (right). The linearly-spaced contours outline
the distribution of a volume-limited sample of galaxies in the plotted diagram, and
the color-coded background shows the median environmental density (computed
using the ten nearest neighbors) for galaxies
with the corresponding luminosity and color. Such multi-variate distributions
encode rich information about formation and evolution of galaxies. Galaxies 
detected by SDSS are representative of the low-redshift Universe (the median 
redshift is $\sim$0.1). The LSST will enable such studies as a function of 
redshift, to $z\sim$2. Adapted from Cowan \& Ivezi\'{c} 
(2008).} 
\label{Fig:cowan}
\end{figure}


\begin{figure}
%\includegraphics[width=1.0\hsize,clip]{panels3.pdf}
\includegraphics[width=1.0\hsize,clip]{panels3.png}
\caption{The LSST will deliver AGN sky densities of 1000-4000 deg$^{-2}$, which is 
competitive with the deepest pencil-beam surveys to date, such as the Chandra
Deep Fields (top panel). The total LSST AGN yield, selected using colors and
variability, should be well over 10 million objects. 
The bottom panel shows the expected distribution of these objects in the 
absolute magnitude vs. redshift plane, color-coded by the probability for
an object to be
detected as variable after 1 year of observations. Note that quasars will
be detected to their formal luminosity cutoff ($M< -23$) even at redshifts
of $\sim$5. Adapted from Brandt et al.~(2007).} 
\label{Fig:panels3}
\end{figure}

\subsubsection{  Synergy with other projects }

LSST will not operate in isolation and will greatly benefit from other precursor and coeval 
data at a variety of wavelengths, depths, and timescales. For example,
in the optical, most of the Celestial 
Sphere will be covered to a limit several magnitudes fainter than LSST saturation 
($r\sim16$), first by the combination of SDSS, PS1, the Dark Energy Survey (Flaugher 2006) and SkyMapper (Keller et al.~2007), 
and then by the Gaia survey. The SkyMapper survey will obtain imaging data in the southern
sky to similar depths as SDSS, the PS1 surveys provided multi-epoch data deeper 
than SDSS in the northern sky, and the Dark Energy Survey will scan 
$\sim$5000 deg$^2$ to similar depth as Pan-STARRS in the southern sky. Despite the lack of 
the $u$ band data and a depth about a magnitude shallower, the Pan-STARRS surveys
will represent a valuable complement to LSST in providing North sky coverage to a limit 
fainter than that of SDSS and SkyMapper. LSST and Gaia will 
be highly complementary datasets for studying the Milky Way in the multi-dimensional space of 
three-dimensional positions, proper motions and metallicity (Ivezi\'{c}, Beers \& Juri\'{c} 2012). 
The Gaia survey will provide calibration checks at the bright end for proper 
motions and trigonometric parallax measurements by LSST, and LSST will extend the 
Gaia survey by four magnitudes. The LSST data stream will invigorate subsequent 
investigations by numerous other telescopes that will provide 
additional temporal, spectral and spatial resolution coverage. 

WFIRST and Euclid will carry out wide-field imaging surveys in the
near-infrared, giving highly complementary photometry to LSST.  The
resulting galaxy SEDs should give rise to even better photometric
redshifts, as well as tighter constraints on stellar masses and star
formation histories crucial for galaxy evolution studies.  The weak
lensing analyses from space and from the ground will also be highly
complementary, and will provide crucial cross-checks of one another.  

LSST will also enable multi-wavelength studies of faint optical
sources using gamma-ray, X-ray, IR and radio data.  For example, the
SDSS detected only 1/3 of all 20cm FIRST sources (Becker et al.~1995)
because it was too shallow by $\sim$4 mag for a complete optical
identification. Similarly, deep optical data are required for
identification of X-ray sources (Brandt \& Hasinger 2005).

LSST will provide a crucial complementary capability to space 
experiments operating in other wavebands, such as the ongoing
%Black Hole Finder Einstein Probe 
NuSTAR Mission (Harrison et al.~2013)
%\footnote{http://www.nustar.caltech.edu/}, 
and the {\it Fermi} 
Gamma-ray Space Telescope (e.g., Atwood et al.~2010). 
%The Laser Interferometer Space Antenna (LISA) and the 
The Laser Interferometer Gravitational 
Wave Observatory (LIGO) may detect ultracompact binaries and black hole mergers through the 
gravitational wave outburst that should be emitted during the final stages of such events. 
LSST has the potential to measure the electromagnetic signal that accompanies the gravitational wave emission, 
thereby providing an accurate position on the sky for the system, which is 
crucial for subsequent observations. LSST will also add new value to the archives for 
billion-dollar class space missions such as Chandra, XMM-Newton, Spitzer, Herschel,
etc., because deep optical multi-color data will enable 
massive photometric  studies of sources from these missions;
any areas of the sky -- whether by design or by serendipity -- in which past, present, or future 
multiwavelength surveys overlap with LSST sky coverage, will be further promoted by LSST 
investigations to ``optical plus multiwavelength Selected Areas''.   
Last but not least, the huge samples of various astronomical source
populations will yield extremely rare objects for investigations by powerful
facilities such as JWST (Gardner et al.~2006) and the next generation
of 20-40 meter telescopes.  
%GSMT (Kan \& Antebi 2003).

In summary, the diversity of science enabled by LSST will be 
unparalleled, extending from the physics of gravity and the
early Universe to the properties of ``killer'' asteroids. While
there are other projects that aim to address some of the same
science goals, no other project matches this diversity and 
LSST's potential impact on society in general. 


\section{   COMMUNITY INVOLVEMENT   }
\label{Sec:community}

LSST has been conceived as a public facility: the database that it will
produce, and the associated object catalogs that are generated from that
database, will be made available to the U.S. and Chilean scientific communities,
to international partners and to the public at large, with no proprietary period. 
We are working with international partners 
to make LSST data products available worldwide. The LSST data management 
system will provide user-friendly tools to access this database and to support
user-initiated queries, run on LSST computers, either at the archive facility 
or at the data access centers. We expect that many, perhaps even the majority,
of LSST discoveries will come from research astronomers with no formal
affiliation to the project, from students, and from interested amateurs, 
intrigued by the accessibility to the Universe that this facility uniquely 
provides. For example, because of its interest in the large public data access 
of the LSST program, Google has joined the LSST team. 

The SDSS provides a good example for how the scientific 
community can be effective in working with large, publicly available
astronomical data sets. The SDSS has published a series of large incremental
data releases via a sophisticated database, roughly once per year, together with 
a paper describing the content of each data release, and extensive on-line 
documentation giving instructions on downloading the catalogs and image data
(see http://www.sdss.org). The overwhelming majority of the almost
6000 refereed papers based 
on SDSS data to date have been written by scientists from outside 
the project, and  include many of the most high-profile results that have come 
from the facility. 

Nevertheless, it is clear that many of the highest priority LSST science
investigations will require organized teams of professionals working together
to optimize science analyses and to assess the importance of systematic 
uncertainties on the derived results. To meet this need, eleven science
collaborations have been established by the project in core science
areas. For example, the LSST Dark Energy Science Collaboration includes
members with interests in the study of dark energy and related topics in 
fundamental physics with LSST data. As of the time of this contribution, there are 
over 700 participants in these collaborations, mostly from LSST member institutions. 
%Periodic calls for applications for membership are issued to the
%professional  community at large\footnote{See
%http://www.noao.edu/lsst/collab\_prop/Scicollab.htm}. 
One may contact the chairs of each science collaboration to learn the
application process to join; all those at US and Chilean institutions,
as well as named individuals from institutions in other countries
which have signed Memoranda of Agreement to contribute to LSST
operations costs are eligible.  
It is anticipated that the LSST science community  will perform its
analyses using  
LSST computational facilities, and the system has been sized accordingly. 

As we design our observing strategies, we are actively seeking and implementing
input by the LSST science community.  The LSST science collaborations
in particular have helped develop the LSST science case and continue
to provide advice on how to optimize their science with choices in
cadence, software,  
and data systems. The Science Collaborations are expected to play a
role  
in the system optimization during the commissioning period. We will continue 
this review when the survey starts, and will maintain our design of flexible 
cadence structure.

The Science Advisory Committee (SAC), chaired by Michael Strauss, provides a formal, and 
two-way, connection to the external science community served by LSST. This committee takes 
responsibility for policy questions facing the project and also deals with technical topics of interest 
to both the science community and the LSST Project. The SAC minutes and notes are available 
publically. Membership of this committee is as follows: N. Brandt
(Penn State), H. Ferguson (STScI), 
C. Hirata (Ohio State), L. Hunter (UC Sant Cruz), J. Kalirai (STScI), B. Jain (UPenn), M. Kasliwal (Carnegie), 
D. Kirkby (UC Irvine), R. Malhotra (U Arizona), R. Mandelbaum
(Carnegie Mellon),
D. Minniti (U Catolica, Chile), R. Mu\~noz (U. Chile), 
M. Strauss (Princeton), L. Walkowicz (Adler), B. Willman (Haverford), and M. Wood-Vasey 
(U Pittsburgh). 



\section{   EDUCATIONAL AND SOCIETAL IMPACTS    }
\label{Sec:impact}


\B{
The impact -- and enduring societal significance -- of the LSST will exceed its direct contributions 
to advances in physics and astronomy.  LSST is uniquely positioned to have high impact with the interested 
public and K-12 educational programs. Engaging the public in LSST activities has, therefore, been part of 
the project design from the beginning. 
With its open data policy, survey mode of operations and 
data products that offer vast potential for discovery, LSST facilitates the active engagement of a 
broad community with pathways to lifelong learning.  LSST will contribute to the national goals of 
enhancing science literacy and increasing the global competitiveness of the US science and technology 
workforce.  We intend to develop in our audience the awareness that they are part of a dynamic universe 
that changes on many time scales.  

We will use three primary techniques to accomplish these goals: visualization, utilization of real data, 
and active engagement in the research process. These techniques will be executed via three primary components: 
a dynamic, immersive public web presence featuring LSST discoveries; a physical presence in classrooms 
and science centers; and an emphasis on citizen science, that is, participation in the research process by 
non-specialists.  LSST's ``movie of the Universe'' will offer a new window on the sky for curious minds of 
all ages.  It is not unreasonable to anticipate tens of millions of public users browsing the LSST 
sky,
%\footnote{At the angular resolution of LSST, it would take over 3
%million  HDTV sets (with 1080$\times$1920  pixel resolution) to
%display a single-epoch  LSST image of half the Celestial Sphere (and
%about 1000 such  images will be  obtained over 10 years).}, 
%THIS IS REPEATED IN THE MAIN TEXT NEAR THE END OF THE PAPER.
exploring discoveries as they are broadcast and monitoring objects 
of interest.  For users just browsing, interfaces such as Sky in Google Earth or Microsoft's World Wide Telescope 
will serve images.  For those wanting a deeper experience, LSST will provide a value-added interface for 
web-based immersive delivery of dynamic multi-media content related to LSST discoveries.

LSST data can become a key part of projects emphasizing student-centered research in middle school, high 
school, and undergraduate settings. Web-based instructional materials, on-line professional development, and 
software tools will facilitate both formal and lifelong learning opportunities.  Citizen science is an integral 
component of the overall LSST EPO program, allowing us to actively engage a large and broad community in 
the excitement of exploration.  LSST EPO expects to operate at least one ``Citizen Science'' project at all times and provide 
properly constructed data products, tools, and interfaces to facilitate efforts outside LSST.   A few dozen 
``power users'' at major science centers, which in turn will reach 10-20 million participants annually, will incorporate 
LSST skymaps, discoveries,  and scientists into displays and live sky shows.  We will provide a specialized data 
access interface to expedite access at digital planetariums; kiosks bridging museum experiences to classroom 
and lifelong learning will be developed. This 
involvement and active participation will allow LSST to fulfill its public responsibility and extend its scientific 
potential –- a truly transformative idea for the 21$^{st}$ century telescope system.
}



\section{SUMMARY AND CONCLUSIONS} 
\label{Sec:conclusions}

Until recently, most astronomical investigations have focused on small 
samples of cosmic sources or individual objects. Over the past decade, 
however, advances in technology have made it possible to move beyond the 
traditional observational paradigm and to undertake large-scale sky 
surveys, such as SDSS, 2MASS, GALEX and many others. This observational 
progress, based on a synergy of advances in telescope construction, detectors, 
and above all, information technology, has had a dramatic impact on nearly all 
fields of astronomy, many areas of fundamental physics, and society in 
general. The LSST builds on the experience of these surveys and addresses
the broad goals stated in several nationally endorsed reports by the U.S. 
National Academy of Sciences. The 2010 report ``New Worlds, New Horizons 
in Astronomy and Astrophysics'' by the Committee for a Decadal Survey of Astronomy and 
Astrophysics
%\footnote{http://www.nap.edu/catalog.php?record\_id=12951}
ranked LSST as its top priority for large ground-based programs.

The realization of the LSST involves extraordinary engineering and 
technological challenges: the fabrication of large, high-precision optics; 
construction of a huge, highly-integrated array of sensitive, wide-band 
imaging sensors; and the operation of a data management facility 
handling tens of terabytes of data each day. The design and development 
effort, which has been underway since 2006 and will continue through the 
onset of construction, includes structural, thermal, and optical analyses 
of all key hardware subsystems, vendor interactions to determine 
manufacturability, explicit prototyping of high-risk elements, prototyping 
and development of data management systems, and extensive systems engineering 
studies. Over 200 technical personnel at a range of institutions are currently 
engaged in this program. 

In September 2013, LSST passed the NSF Final Design Review for construction, 
and in May 2014 the National Science Board approved the project.  
The fabrication of the primary/tertiary mirror has been completed. 
In 2014 LSST transitioned from the design and development phase to
construction, and the Associated Universities for Research in
Astronomy (AURA) has formal responsibility for the LSST project.  
The U.S. Department of Energy is supporting the cost of constructing the 
camera. LSST has high visibility in the high-energy physics community,
both at universities and government laboratories. The project is scheduled to 
begin regular survey operations by 2022
(it will take about five years from the start of the federal construction phase
to full system integration and the beginning of the commissioning period). 
The construction cost of LSST is being borne by the US National Science
Foundation, the Department of Energy, generous contributions from several 
private foundations and institutions, and the member institutions of the 
LSST Corporation. The LSST budget includes a significant EPO program. 

The LSST survey will open a movie-like window on objects that 
change brightness, or move, on timescales ranging from 10 seconds to 10 years.
The survey will have a raw data rate of about 15 TB per night (about the same as one
complete Sloan Digital Sky Survey per night), and will collect over 100 PB
of data over its lifetime, resulting in an incredibly rich and extensive
public archive that will be a treasure trove for breakthroughs in many areas 
of astronomy and physics. About 20 billion galaxies and a similar number of stars
will be detected -- for the first time in history, the number of cataloged 
celestial objects will exceed the number of living people! About a thousand
observations of each position across half of the Celestial Sphere will
represent the greatest movie of all time: assuming HDTV resolution (2.1
Mpix/frame) with 30 frames per second, and 270 color images per position
(e.g., the $ugr$ and $izy$ color-composites, c.f. Table 1), it would take 
about eleven uninterrupted months to view this color movie (equivalent
to about 4,000 feature films). 


%The integrated cost of LSST will be {\$638M} for construction (NSF and DOE contributions), 
%and {\$37M} per year for operations (in 2014 U.S. dollars). 

%Beyond the improved image quality, LSST will be much more cost 
%efficient than the current state-of-the-art massive optical survey, the 
%SDSS. For example, for each \$100 spent, the SDSS obtained 
%single-epoch $ugriz$ photometry for $\sim$200 sources and spectra for 
%two sources. In contrast, for \$100 LSST will obtain $ugrizy$ 
%photometry for about 4,000 sources (to a limit 5 magnitudes deeper 
%than SDSS) and a 800-point multi-color light curve for $\sim$400 sources. Compared
%to the SDSS, LSST will produce three orders of magnitude larger data volume with
%an increase in total cost of less than a factor of ten.

LSST has been conceived as a public facility: the database that it will
produce, and the associated object catalogs that are generated from that
database, will be made available to the U.S. and Chilean
scientific communities and to the public at large with no proprietary period. 
We are working with prospective foreign partners to make all of the LSST data more broadly 
available worldwide.  As of 2014, scientists in 67 institutions from
28 countries have expressed interest in contributing significantly to
the LSST operating costs in exchange for data access.  The software
which creates the LSST database will be open source.  
LSST will  be a significant milestone in the globalization of the information revolution.
The vast relational database of 32 trillion observations of 40 billion objects
will be mined for the unexpected and used for precision experiments in astrophysics.
LSST will be in some sense an internet telescope:
the ultimate network peripheral device to explore the Universe, and
a shared resource for all humanity. 

\vskip 0.15in
%\section*{ACKNOWLEDGMENTS}
\centerline{ACKNOWLEDGMENTS}

In 2003, the LSST Corporation was formed as a non-profit 501(c)3 Arizona corporation 
with headquarters in Tucson, AZ.  Membership has since expanded to
more than forty institutional members 
including Adler Planetarium, Argonne National Laboratory, Brookhaven National Laboratory, California Institute of Technology,  
Carnegie Mellon University, Chile,  Columbia University, Cornell University, Drexel University, 
Fermi National Accelerator Laboratory, George Mason University, Google Inc., Harvard-Smithsonian 
Center for Astrophysics, Institut de Physique Nuclaire et de Physique des Particules, Johns Hopkins 
University, Kavli Institute for Particle Astrophysics and Cosmology -- Stanford University, Las 
Cumbres Observatory Global Telescope Network, Inc., Lawrence Livermore National Laboratory, 
Los Alamos National Laboratory,  National Optical Astronomy
Observatory, National Radio Astronomy Observatory, Northwestern
University, Oxford University, 
Princeton University, Purdue University, Research Corporation for Science Advancement, Rutgers 
University, SLAC National Accelerator Laboratory, Space Telescope Science Institute, 
Texas A\&M University, The Pennsylvania State University, The University of Arizona, University 
of California at Davis, University of California at Irvine, University College London, University of 
Illinois at Urbana-Champaign, University of Michigan, University of Pennsylvania, University of 
Pittsburgh, University of Washington, and Vanderbilt and Fisk Universities. 

LSST is a public-private partnership.  Design and development activity is
supported by in part the National Science Foundation under Scientific
Program Order No. 9 (AST-0551161), Scientific Program Order No. 1
(AST-0244680) through Cooperative Agreement AST-0132798, and grant AST-0441069. 
Portions of this work are supported by the Department of Energy under contract
DE-AC02-76SF00515 with the Stanford Linear Accelerator Center, contract
DE-AC02-98CH10886 with Brookhaven National Laboratory, contract
DE-AC52-07NA27344 with Lawrence Livermore National Laboratory, and grant
DE-FG02-07ER41505. A portion of this work was conducted at the Jet Propulsion 
Laboratory, California Institute of Technology, under a contract with NASA.
Gifts by the Charles Simonyi Fund for Arts and Sciences,
and Bill Gates, and grants by W. M. Keck Foundation and the TABASGO Foundation
are gratefully acknowledged. Additional funding has been provided by private gifts, grants 
to universities, and in-kind support at Department of Energy laboratories and other 
LSSTC Institutional Members.  

NOAO is operated by the Association of Universities for Research in Astronomy,
Inc. (AURA) under cooperative agreement with the National Science Foundation.

This work was supported in part by National Science Foundation Grant No. PHYS-1066293 
and the hospitality of the Aspen Center for Physics.

We thank Chris Borden, Tim Knauer and Branimir Sesar for valuable comments.


\appendix{\bf Version History }

\vskip 0.06in
Version 1.0 (May 15, 2008): the first posting. 
\vskip 0.06in
Version 2.0 (June 7, 2011): acknowledged the Decadal Survey 2010 report; updated construction schedule;
updated expected performance in Table 2; added sections on LSST simulations and data mining;
updated several figures; updated references; expanded author list. 
\vskip 0.06in
Version 3.0 (August 26, 2014): acknowledged the start of federal construction;
updated system description and science examples, updated several figures; refreshed references;
expanded author list. 




 % References
\vskip 0.3in
\begin{thebibliography}{}

\bibitem[()]{} Abazajian, K.N., Adelman-McCarthy, J.K., Ag\"ueros,
  M.A. et al.~2009, Astrophysical Journal Supplements, 182, 543

\bibitem[()]{} A'Hearn, M.F. 2004, in Comets II, M.C. Festou, H.U. Keller, and H.A. Weaver (eds.), 
             University of  Arizona Press, Tucson, p.17-22 

\bibitem[()]{} Aihara, H., Allende Prieto, C., An, D., et al.~2011,
  Astrophysical Journal Supplement, 195, 26

%\bibitem[()]{} Albrecht, A. \& Bernstein, G. 2007, Physical Review D, 75, 3003

\bibitem[()]{} Albrecht, A., Bernstein, G., Cahn, R. 2006, Report of
  the Dark Energy Task Force, arXiv:0609591

\bibitem[()]{} Alcock, C., Allsman, R.A., Alves, D.R., et al.~2000, Astrophysical Journal, 542, 281  

\bibitem[()]{} Amores, E.B. \& L\'{e}pine, J.R.D. 2005, Astronomical Journal, 130, 659

\bibitem[()]{} An, D., Johnson, J.~A., Clem, J.~L., et al.~2008, Astrophysical Journal Supplement, 179, 326

\bibitem[()]{} An, D., Beers, T.~C., Johnson, J.~A., et al.~2013, Astrophysical Journal, 763, 65

\bibitem[()]{} Anderson, G.P., Berk, A., Acharya, P.K., et al.~1999, in ``Optics in
             Atmospheric Propagation and Adaptive Systems III'',  A. Kohnle \& J.D. Gonglewski,
             eds., Proceedings of the SPIE, 3866, 2

\bibitem[()]{} Anderson, S.F., Haggard, D., Homer, L., et al.~2005, Astronomical Journal, 130, 2230

\bibitem[()]{} Anderson, L., Aubourg, \'E., Bailey, S. et al.~2014,
  Monthly Notices of the Royal Astronomical Society, 441, 24

\bibitem[()]{} Angel, R., Lesser, M. \& Sarlot, R. 2000, in  ``Imaging the Universe in Three 
             Dimensions: Astrophysics with Advanced Multi-wavelength Imaging Devices'',
             W. Bruegel and J. Bland-Hawthorn, eds., ASP Conference Series, 195, 81

\bibitem[()]{} Atwood, W.B., Abdo, A.A., Ackerman, M., et al., 2009, Astrophysical Journal, 697, 1071

\bibitem[()]{} Barnard, M., Abrahamse, A., Albrecht, A., Bozek, B. \& Yashar, M., 2008, Physical Review D, 78, 043528

\bibitem[()]{} Bartelmann, M. \& Schneider, P. 2001, Physics Reports, 340, 291
	
\bibitem[()]{} Bartelmann, M., Huss, A., Colberg, J.M., Jenkins, A. \& Pearce, F.R. 1998, Astronomy \& 
             Astrophysics, 330, 1

\bibitem[()]{} Batista, V., Beaulieu, J.P., Gould, A., et al.~2014, Astrophysical Journal, 780, 54 

\bibitem[()]{} Bayliss, D.D.R. \& Sackett, P.D. 2011, Astrophysical Journal, 743, 103	

\bibitem[()]{} Beaulieu, J.P., Bennett, D.P., Fouqu\'{e}, P., et al.~2006, Nature, 439, 437

\bibitem[()]{} Becker, A.C., Wittman, D.M., Boeshaar, P.C., et al.~2004, Astrophysical Journal, 611, 418

\bibitem[()]{} Becker, R.H., White, R.L., \& Helfand, D.J. 1995, Astrophysical Journal, 450, 559

\bibitem[()]{} Bell, E.F., Zucker, D.B., Belokurov, V. et al.~2008, Astrophysical Journal, 680, 295
	
\bibitem[()]{} Belokurov, V., Zucker, D.B., Evans, N.W., et al.~2007a, Astrophysical Journal, 654, 897

\bibitem[()]{} Belokurov, V., Evans, N.W., Moiseev, A., et al.~2007b, Astrophysical Journal, 671, L9

\bibitem[()]{} Belokurov, V., Koposov, S.~E., Evans, N.W., et al.~2014, MNRAS, 437, 116

\bibitem[()]{} Bergeron, P., Wesemael, F. \& Beauchamp, A. 1995, Publications of the Astronomical Society
                   of the Pacific, 107, 1047

\bibitem[()]{} Bernstein, G.M., Trilling, D.E., Allen, R.L., et al.~2004, Astronomical Journal, 128, 1364

\bibitem[()]{} Berry, M., Ivezi\'c, \v Z., Sesar, B., et al.~2012, Astrophysical Journal, 757, 166

\bibitem[()]{} Blake, C. \& Glazebrook, K. 2003, Astrophysical Journal, 594, 665

\bibitem[()]{} Blandford, R.D., Marshall, P., Oguri, M.,et al.~2008, 
                           American Astronomical Society, AAS Meeting \#211, \#137.07
	
\bibitem[()]{} Bloom, J.S., Starr, D.L., Butler, N.R. et al.~2008, Astronomische Nachrichten, 329, 284

%\bibitem[()]{} Bloom, J.S., Perley, D.A., Li, W., et al.~2009, Astrophysical Journal, 691, 723

\bibitem[()]{} Bloom, J.S., Giannios, D., Metzger, B.D., et al.~2011, Science, 333, 203

\bibitem[()]{} Bochanski, J.J., West, A.A., Hawley, S.L. \& Covey, K.R. 2007, Astronomical Journal, 133, 531

\bibitem[()]{} Bochanski, J.J., Savcheva, A., West, A.A. \& Hawley, S.L. 2013, Astronomical Journal, 145, 40

\bibitem[()]{} Bochanski, J.J., Willman, B., West, A.A., Strader, J. \& Chomiuk, L. 2014, Astronomical Journal, 147, 76

\bibitem[()]{} Bond, N.A.,  Ivezi\'{c}, \v{Z}., Sesar, B., et al.~2010, Astrophysical Journal, 716, 1

\bibitem[()]{} Bongiorno, A., Zamorani, G., Gavignaud, I., et al.~2007, Astronomy \& Astrophysics, 472, 443 

\bibitem[()]{} Borne, K. 2008, Astronomische Nachrichten, 329, 255

%\bibitem[()]{} Borne, K., Laher, R., Ivezi\'{c}, \v{Z}., et al.~2009, BAAS, 41, 372

\bibitem[()]{} Bottke, W.F., Vokrouhlick\'{y}, D., Bro\v{z}, M., Nesvorn\'y, D. \& Morbidelli, A. 2001, 
             Science, 294, 1693

\bibitem[()]{} Brandt, N., Anderson, S.F., Ballantyne, D.R., et al.~2007, American Astronomical 
             Society, AAS Meeting \#211, \#137.09

\bibitem[()]{} Brandt, N. \& Hasinger, G. 2005, ARA\&A, 43, 827

\bibitem[()]{} Brasser, R., Duncan, M.~J., Levison, H.~F., Schwamb, M.~E. \& Brown, M.~E. 2012, Icarus, 217, 1

\bibitem[()]{} Briceno, C., Calvet, N., Hernandez, J., et al.~2005, Astronomical Journal, 129, 907

\bibitem[()]{} Brown, M.E., Trujillo, C. \& Rabinowitz, D. 2004, Astrophysical Journal, 617, 645

\bibitem[()]{} Bruzual, G. \& Charlot, S. 2003, MNRAS, 344, 1000

\bibitem[()]{} Budav\'{a}ri, T., Connolly, A.J., Szalay, A.S., et al.~2003, Astrophysical Journal, 595, 59

\bibitem[()]{} Bullock, J.S. \& Johnston, K.V. 2005,  Astrophysical Journal, 635, 931

\bibitem[()]{} Burke, D., Axelrod, T., Bartlett, J., et al.~2007, 
                      American Astronomical Society, AAS Meeting \#211, \#137.23

\bibitem[()]{} Burrows, D., Sudarsky, D. \& Hubeny, I. 2006, Astrophysical Journal, 640, 1063

\bibitem[()]{} Carollo, D., Beers, T.C., Lee, Y.S., et al.~2007, Nature, 450, 1020	

\bibitem[()]{} Casetti-Dinescu, D.I., Girard, T.M., Herrera, D., et al.~2007, Astronomical Journal, 134, 195
	
\bibitem[()]{} Charbonneau, D., Brown, T.M., Latham, D.W. \& Mayor, M. 2000, Astrophysical Journal, 529, L45

\bibitem[()]{} Choi, A., Tyson, J.A., Morrison, C.B., et al.~2012, Astrophysical Journal, 759, 101 

\bibitem[()]{} Choi, J.Y., Han, C., Udalski, A., et al.~2013, Astrophysical Journal, 768, 129

\bibitem[()]{} Chou, M.Y., Majewski, S.~R., Cunha, K., et al.~2007, Astrophysical Journal, 670, 346

\bibitem[()]{} Coe, D. \& Moustakas, L.A. 2009, Astrophysical Journal, 706, 45 

\bibitem[()]{} Cole, S., Percival, W.J., Peacock, J.A., et al.~2005, Monthly Notices of the Royal 
             Astronomical Society, 362, 505

\bibitem[()]{} Connolly, A.J., Angeli, G.Z., Chandrasekharan, S., et al.~2014, Proc. SPIE 9150, 915014

\bibitem[()]{} Cooper, A.P, Cole, S., Frenk, C.S., et al.~2010, MNRAS, 406, 744

\bibitem[()]{} Cooray, A.R. 1999, Astronomy \& Astrophysics, 348, 31
	
\bibitem[()]{} Cooray, A.R., Hu, W., Huterer, D. \& Joffre, M. 2001, Astrophysical Journal, 557, L7 

\bibitem[()]{} Cowan, N. \& Ivezi\'{c}, \v{Z}. 2008, Astrophysical Journal, 674, L13

\bibitem[()]{} Cuadra, J., Armitage, P.J., Alexander, R.D., Begelman, M.C. 2009, MNRAS, 393, 1423 

\bibitem[()]{} Cushing, M.C., Rayner, J.T. \& Vacca, W.D. 2005, Astrophysical Journal, 623, 1115

\bibitem[()]{} Dalal, N. \& Kochanek, C.S. 2002, Astrophysical Journal, 575, 25

\bibitem[()]{} Dandy, C.L., Fitzsimmons, A. \& Collander-Brown, S.J. 2002, Icarus, 163, 363  

\bibitem[()]{} Delgado, F., Saha, A., Chandrasekharan, S., et al.,  2014,  SPIE, 9150, 915015

\bibitem[()]{} De Lucia, G., Springel, V., White, S.D.M., Croton, D. \& Kauffmann, G. 2006, MNRAS, 366, 499
	
\bibitem[()]{} de Bruijne, J.H.J. 2012, Astrophysics and Space Science, 341, 31

\bibitem[()]{} de Jong, J.T.A., Kuijken, K.H. \& H\'{e}raudeau, P. 2008, Astronomy \& Astrophysics, 478, 755

\bibitem[()]{} DeMeo, F.E., Binzel, R.P., Slivan, S.M. \& Bus, S.J. 2009, Icarus, 202, 160

\bibitem[()]{} de Vries, W.H., Becker, R.H., \& White, R.L. 2003, Astronomical Journal, 126, 1217

\bibitem[()]{} Djorgovski, S.G., Mahabal, A.A., Drake, A.J., et al.~2012, Proceedings of the IAU, 285, 141

\bibitem[()]{} Dolney, D., Jain, B. \& Takada, M. 2006, MNRAS, 366, 884

\bibitem[()]{} Doressoundiram, A., Peixinho, N., Moullet, A., et al.~2007, Astronomical Journal, 134, 2186

\bibitem[()]{} Doron, C. \& Eliran, D. 2012, Astrophysical Journal, 747, 62

\bibitem[()]{} Drake, A.J., Beshore, E., Catelan, M., et al.~2010, arXiv:1009.3048 

\bibitem[()]{} Drake, A.J., Djorgovski, S.G., Mahabal, A., et al.~2009, Astrophysical Journal, 696, 870

\bibitem[()]{} Duncan, M.J. \& Levison, H.F. \& Budd, S.M. 1995, Astronomical Journal, 110, 3073

\bibitem[()]{} Durech, J., Kaasalainen, M., Warner, B.D., et al.~2009, Astronomy \& Astrophysics, 493, 291

\bibitem[()]{} Eisenstein, D.J., Zehavi, I., Hogg, D.W., et al.~2005, Astrophysical Journal, 633, 560

\bibitem[()]{} Eisenstein, D.J., Hu, W. \& Tegmark, M. 1998, Astrophysical Journal, 504, L57

\bibitem[()]{} Elliot, J.L., Kern, S.D., Clancy, K.B., et al.~2005,  Astronomical Journal, 129, 1117

\bibitem[()]{} Evans, C.R. \& Kochanek, C.S. 1989, Astrophysical Journal, 346, L13

\bibitem[()]{} Finch, C.~T., Zacharias, N. \& Henry, T.~J. 2010, Astronomical Journal, 140, 844

\bibitem[()]{} Flaugher, B. 2006, Proc. of SPIE, Vol. 6269, 62692C

\bibitem[()]{} Foley, R.J. 2012, Astrophysical Journal, 748, 127

\bibitem[()]{} Fukugita, M., Ichikawa, T., Gunn, J.E., et al.~1996, Astronomical Journal, 111, 1748

\bibitem[()]{} Gal-Yam, A., Ofek, E.O., Poznanski, D., et al.~2006, Astrophysical Journal, 639, 331

\bibitem[()]{} Gardner, J.P., Mather, J.C., Clampin, M., et al.~2006, Space Science Reviews, 123, 485

\bibitem[()]{} Garnavich, P.M., Smith, R.C., Miknaitis, G., et al.~2005, 
             American Astronomical Society, AAS Meeting \#205, \#108.18 	

\bibitem[()]{} Gawiser, E., van Dokkum, P.G., Herrera, D., et al.~2006, Astrophysical Journal Supplement, 162, 1 

%\bibitem[()]{} Gee, P., Tyson, J.A., Pinto, P., et al.~2007, 
%             American Astronomical Society, AAS Meeting \#211, \#137.28
	
\bibitem[()]{} Gezari, S. 2012, European Physical Journal Web of Conferences, 39, 3001

\bibitem[()]{} Gezari, S., Basa, S., Martin, D.C., et al.~2008, Astrophysical Journal, 676, 944

\bibitem[()]{} Gezari, S., Chornock, R., Rest, A., et al.~2012, Nature, 485, 217 

\bibitem[()]{} Giannios, D. \& Metzger, B.D. 2011, MNRAS, 416, 2102

\bibitem[()]{} Girard, T.M., Korchagin, V.I., Casetti-Dinescu, D.I., et al.~2006, Astronomical 
             Journal, 132, 1768

\bibitem[()]{} Gladman, B., Kavelaars, J.J., Petit, J-M., et al.~2001, Astronomical Journal, 122, 1051

\bibitem[()]{} Gonz\'{a}lez, J.E., Lacey, C.G., Baugh, C.M., Frenk, C.S. \& Benson, A.J. 2009, MNRAS, 397, 1254

\bibitem[()]{} Graham, M.J., Djorgovski, S.G., Drake, A.J., et al.~2014, MNRAS, 439, 703 

\bibitem[()]{} Granvik, M., Virtanen, J., Oszkiewicz, D. \& Muinonen, K. 2009, Meteoritics \& Planetary Science, 44, 1853

\bibitem[()]{} Grav, T., Jedicke, R., Denneau, L., Holman, M.J.  \& Spahr, T. 2007, BAAS, 211, 4721

\bibitem[()]{} Green, G.~M., Schlafly, E.~F., Finkbeiner, D.~P., et al.~2014, Astrophysical Journal, 783, 114 

\bibitem[()]{} Gunn, J.E., Carr, M., Rockosi, C., et al.~1998, Astronomical Journal, 116, 3040

\bibitem[()]{} Gunn, J.E., Siegmund, W.A., Mannery, E.J., et al.~2006, Astronomical Journal, 131, 2332

\bibitem[()]{} Hannestad, S., Tu, H. \& Wong, Y.Y. 2006, Journal of Cosmology and Astroparticle 
             Physics, 6, 25

\bibitem[()]{} Harrison, F.A., Craig., W.W., Cristensen, F.E. et
  al.~2013, Astrophysical Journal, 770, 103

\bibitem[()]{} Hartman, J.D., Gaudi, B.S., Holman, M.J., et al.~2009, Astrophysical Journal, 695, 336

\bibitem[()]{} Helmi, A., Ivezi\'{c}, \v{Z}., Prada, F., et al.~2003, Astrophysical Journal, 586, 195

\bibitem[()]{} Hicken, M., Wood-Vasey, W.M., Blondin, S., et al.~2009, Astrophysical Journal, 700, 1097

\bibitem[()]{} Hill, V., Lecureur, A., G{\'o}mez, A., et al.~2011, Astronomy \& Astrophysics, 534, 80 

\bibitem[()]{} Hoeflich, P., Wheeler, J.C. \& Thielemann, F.K. 1998, Astrophysical Journal, 495, 617

\bibitem[()]{} Hovatta, T., Pavlidou, V., King, O.G., et al.~2014, MNRAS, 439, 690	 

\bibitem[()]{} Howell, D.A., Sullivan, M., Conley, A. \& Carlberg, R. 2007, Astrophysical Journal, 667, 37

\bibitem[()]{} Hsieh, H.H., \& Jewitt, D. 2006, Science, 312, 561

\bibitem[()]{} Hu, W. \& Tegmark, M. 1999, Astrophysical Journal, 514. L65

\bibitem[()]{} Hu, W. \& Haiman, Z. 2003, Physical Review D, 68, 063004

\bibitem[()]{} Hu, W. \& Jain, B. 2004, Physical Review D, 70, 3009

\bibitem[()]{} Huterer, D., Takada, M., Bernstein, G. \& Jain, B. 2006, MNRAS, 366, 101
	
\bibitem[()]{} Ibata, R., Irwin, M., Lewis, G.F. \& Stolte, A. 2001, Astrophysical Journal, 547, L133
	
\bibitem[()]{} Ishak, M., Upadhye, A. \& Spergel, D.N. 2006, Physical Review D, 74, 043513

\bibitem[()]{} Ivezi\'c, \v Z., Connolly, A.J., Vanderplas, J.T., and
  Gray, A. 2014, Statistics, Data Mining, and Machine Learning in
  Astronomy, Princeton University Press

\bibitem[()]{} Ivezi\'{c}, \v{Z}., Tabachnik, S., Rafikov, R., et al.~2001, Astronomical Journal, 122, 2749
	
\bibitem[()]{} Ivezi\'c, \v Z., Lupton, R.H., Schlegel, D., et al.~2003, astro-ph/0309075

\bibitem[()]{} Ivezi\'{c}, \v{Z}, Tyson, J.A., Juri\'{c}, M. et al.~2007a, Proceedings of IAU Symposium 
             236. Edited by G.B. Valsecchi and D. Vokrouhlick\'{y}. Cambridge: Cambridge University 
             Press, 353 (also astro-ph/0701506)

\bibitem[()]{} Ivezi\'{c}, \v{Z}., Smith, J. A., Miknaitis, G., et al.~2007b, Astronomical Journal, 134, 973

\bibitem[()]{} Ivezi\'{c}, \v{Z}. 2007c, ASP Conference Series, Vol. 378, 485 (also astro-ph/0701507)

\bibitem[()]{} Ivezi\'c, \v Z., Sesar, B., Juri\'{c}, M., et al.~2008a, Astrophysical Journal, 684, 287

\bibitem[()]{} Ivezi\'c, \v Z., Axelrod, T., Becker, A. C., et al.~2008b, in American Institute of
                           Physics Conference Series, ed. C. A. L. Bailer-Jones, Vol. 1082, 359 (also arXiv:0810.5155)

\bibitem[()]{} Ivezi\'c, \v Z., Beers, T.C. \& Juri\'{c}, M. 2012, ARA\&A, 50, 251

\bibitem[()]{} Ivezi\'c, \v Z., Brandt, W.N., Fan, X., et al.~2013, arXiv:1312.3963

\bibitem[()]{} Jain, B. \& Taylor, A. 2003, Physical Review Letters, 91, 1302	

\bibitem[()]{} Jain, B. \& Zhang, P. 2008, Physical Review D, 78, 063503 
 
\bibitem[()]{} Jain, B., Vikram, V. \& Sakstein, J. 2013, Astrophysical Journal, 779, 39

\bibitem[()]{} Jedicke, R., Nesvorn{\'y}, D., Whiteley, R., Ivezi{\'c}, {\v Z}., \& Juri{\'c}, M. 2004, 
             Nature, 429, 275 

\bibitem[()]{} Jedicke, R., Denneau, L., Grav, T., et al.~2005, American Astronomical Society, AAS 
             Meeting \#207, \#121.02

\bibitem[()]{} Jee, M.J. \& Tyson, J.A. 2011, PASP, 123, 596

\bibitem[()]{} Jewitt, D.C., Trujillo, C.A. \& Luu, J.X. 2000, Astronomical Journal, 120, 1140

\bibitem[()]{} Jewitt, D.C., Stuart, J.S., \& Li, J. 2011, Astronomical Journal, 142, 28 

\bibitem[()]{} Jewitt, D.C. 2012, Astronomical Journal, 143, 66

\bibitem[()]{} Johnston, K.~V., Bullock, J.~S., Sharma, S., et al.~2008, Astrophysical Journal, 689, 936

\bibitem[()]{} Jones, R.L., Chesley, S., Connolly, A.J., et al.~2007, American Astronomical Society, AAS 
             Meeting \#211, \#137.14 

\bibitem[()]{} Jones, R.L., Gladman, B., Petit, J.-M., et al.~2006, Icarus, 185, 508

\bibitem[()]{}  Jones, R.~L., Yoachim, P., Chandrasekharan, S., 2014, SPIE, 9149, 91490B

\bibitem[()]{} Juri\'{c}, M., Ivezi\'c, \v Z., Brooks, A., et al.~2008, Astrophysical Journal, 673, 864

\bibitem[()]{} Juri\'{c}, M., et al.~2013, ``LSST Data Products Definition Document'' \url{http://ls.st/fl5}, LSST Document Handle LSE-163

\bibitem[()]{} Kaib, N.~A. \& Quinn, T. 2008, Icarus, 197, 221 

\bibitem[()]{} Kaiser, N., Aussel, H., Burke, B.E., et al.~2002, in ``Survey and Other Telescope 
             Technologies and Discoveries'', Tyson, J.A. \& Wolff, S.,
             eds. Proceedings of the SPIE, 4836, 154 

\bibitem[()]{} Kalirai, J.S., Hansen, B.M.S., Kelson, D.D., et al.~2008, Astrophysical Journal, 676, 594

\bibitem[()]{} Kalirai, J.S., Richer, H.B., Fahlman, G.G., et al.~2001, Astronomical Journal, 122, 257
	
%\bibitem[()]{} Kan, F.W. \& Antebi, J. 2003, in ``Future Giant
%Telescopes'', Angel, J.R.P. \& Gilmozzi, R., eds., Proceedings of the
%SPIE, 4840, 485 

\bibitem[()]{} Kann, D.A., Klose, S., Zhang, B., et al.~2010, Astrophysical Journal, 720, 1513 
 
%\bibitem[()]{} Kashikawa, N., Takata, T., Ohyama, Y., et al.~2003, Astronomical Journal, 125, 53

\bibitem[()]{} Kaspi, S., Brandt, W.N., Maoz, D., et al.~2007, Astrophysical Journal, 659, 997

\bibitem[()]{} Keller, S.C., Schmidt, B.P., Bessell, M.S., et al.~2007, Publications of the Astronomical 
             Society of Australia, 24, 1

\bibitem[()]{} Kne\v{z}evi\'{c}, Z. \& Milani, A. 2005, Highlights of Astronomy, Ed. O. Engvold. San 
             Francisco, CA, p.~758

\bibitem[()]{} Knox, L., Song, Y., Tyson, J.A. 2006, Physical Review D, 74, 3512 

\bibitem[()]{} Komatsu, E., Dunkley, J., Nolta, M.R., et al.~2009, Astrophysical Journal, 180, 330 
	
%\bibitem[()]{} Konacki, M., Torres, G., Jha, S. \& Sasselov, D.D. 2003, Nature, 421, 507

\bibitem[()]{} Kowalski, A., Hawley, S.L., Holtzman, J.A., Wisniewski, J.P. \& Hilton, E.J. 2010, Astrophysical Journal, 714, L98

\bibitem[()]{} Kowalski, M., Rubin, D., Aldering, G., et al.~2008,  Astrophysical Journal, 686, 749

\bibitem[()]{} Kulkarni, S., Ofek, E.O., Rau, A., et al.~ 2007, Nature, 447, 458

\bibitem[()]{} Kundi\'{c}, T., Turner, E.L., Colley, W.N., et al.~1997, Astrophysical Journal, 482, 75

\bibitem[()]{} Kurucz, R.L. 1979, Astrophysical Journal Supp., 40, 1

\bibitem[()]{} Kurucz, R.L. 1993, ``CD-ROM No.13”, Cambridge, Mass., Smithsonian Astrophysical Observatory

\bibitem[()]{} Laureijs, R., Amiaux, J., Ardiuni, S. et al.~2011, arXiv:1110.3193

\bibitem[()]{} Law, D.R., Johnston, K.V. \& Majewski, S.R. 2005,  Astrophysical Journal, 619, 807

\bibitem[()]{} Law, N.M., Kulkarni, S.R., Dekany, R.G., et al.~2009, PASP, 121, 1395

\bibitem[()]{} Lenz, D.D., Newberg, J., Rosner, R., et al.~1998, Astrophysical Journal Supp., 119, 121

\bibitem[()]{} Lepine, S. 2008, Astronomical Journal, 135, 2177

\bibitem[()]{} Levi, M., Bebek, C., Beers, T., et al. 2013, arXiv:1308.0847

\bibitem[()]{} Levison, H.~F., Duncan, M.~J., Brasser, R. \& Kaufmann, D.~E. 2010, Science, 329, 187

\bibitem[()]{} Linder, E.V. 2003, Physical Review D, 68, 083504
	
\bibitem[()]{} Lintott, C., Schawinski, K., Bamford, S. 2011, MNRAS,
  410, 166

\bibitem[()]{} Longhitano, M. \& Binggeli, B. 2010, Astronomy \& Astrophysics, 509, 46

\bibitem[()]{} Lorimer, D.R., Bailes, M., McLaughlin, M.A., Narkevic, D.J. \& Crawford, F. 2007, Science, 318, 777	

\bibitem[()]{} Lowry, S.C., Fitzsimmons, A., Cartwright, I.M. \& Williams, I.P. 1999, Astronomy \& 
             Astrophysics, 349, 649

\bibitem[()]{} LSST Dark Energy Science Collaboration 2012, arXiv:1211.0310

\bibitem[()]{} LSST Science Book, Version 2.0,
  http://www.lsst.org/lsst/SciBook, arXiv:0912.0201

\bibitem[()]{} Lucas, P.W. \& Roche, P.F. 2000, MNRAS, 314, 858
	
\bibitem[()]{} Lue, A., Scoccimarro, R. \& Starkman, G. 2004, Physical Review D, 69, 4005

\bibitem[()]{} Luhman, K.L. 2014,  Astrophysical Journal, 786, 18 

\bibitem[()]{} Lupton, R., Blanton, M.R., Fekete, G., Hogg, D.W.,
  O'Mullane, W., Szalay, A., and Wherry, N. 2004, Publications of the
  Astronomical Society of the Pacific, 116, 133

\bibitem[()]{} MacLeod, C.L., Ivezi\'{c}, \v{Z}., Kochanek, C.S., et al.~2010, Astrophysical Journal, 721, 1014

\bibitem[()]{} Mahabal, A.A., Djorgovski, S.G., Drake, A.J., et al.~2011, Bulletin of the Astronomical Society of India, 39, 387

\bibitem[()]{} Majewski, S.R., Skrutskie, M.F., Weinberg, M.D. \& Ostheimer, J.C. 2003, 
             Astrophysical Journal, 599, 1082

\bibitem[()]{} Mamajek, E.E. 2012, arXiv:1210.1616

\bibitem[()]{} Mandelbaum, R. Seljak, U., Kauffmann, G., Hirata, C.M. \& Brinkmann, J. 2006, MNRAS, 368, 715

\bibitem[()]{} Marcus, R.A., Ragozzine, D., Murray-Clay, R.A. \& Holman, M.J. 2011, Astrophysical Journal, 733, 40

\bibitem[()]{} Martin, D.C., Fanson, J., Schiminovich, D., et al.~2006, Astrophysical Journal, 619, L1

\bibitem[()]{} Martini, P. \& Schneider, D.P. 2003, Astrophysical Journal, 597, 109

\bibitem[()]{} Mathieu, R. 2000, Proceedings from ASP Conference, 198, R. Pallavicini, G. Micela, and 
             S. Sciortino, eds. 517

\bibitem[()]{} Matthews, D.J. \& Newman, J.A. 2010, Astrophysical Journal, 721, 456

\bibitem[()]{} M\'{e}nard, B., Scranton, R., Schmidt, S., et al.~2013, arXiv:1303.4722

\bibitem[()]{} Metzger, B.D. \& Berger, E. 2012, Astrophysical Journal, 746, 48

\bibitem[()]{} Monet, D.G., Levine, S.E., Canzian, B., et al.~2003, Astronomical Journal, 125, 984

\bibitem[()]{} Morokuma, T., Inada, N., Oguri, M., et al.~2007, Astronomical Journal, 133, 214

\bibitem[()]{} Morokuma, T., Mamoru D., Naoki Y., et al.~2008, Astrophysical Journal, 676, 163	

\bibitem[()]{} Morrison, C.B., Scranton, R., M\'{e}nard, B., et al.~2012, MNRAS, 426, 2489

\bibitem[()]{} Munn, J.A., Monet, D.G., Levine, S.E., et al.~2004, Astronomical Journal, 127, 3034
	
\bibitem[()]{} Neill, J.D. \& Shara, M.M. 2005, Astronomical Journal, 129, 1873

\bibitem[()]{} Nemiroff, R.J. 2003, Astronomical Journal, 125, 2740

\bibitem[()]{} Ness, M., Debattista, V.~P., Bensby, T., et al.~2014, Astrophysical Journal, 787, L19

\bibitem[()]{} Nesvorn\'{y}, D., Jedicke, R., Whiteley, R. \& Ivezi\'{c}, \v{Z}. 2005, Icarus 173, 132

\bibitem[()]{} Newman, J., 2008, Astrophysical Journal, 684, 88

\bibitem[()]{} Nissanke, S., Kasliwal, M. \& Georgieva, A. 2013,  Astrophysical Journal, 767, 124 

\bibitem[()]{} Oguri, M. \& Kawano, Y. 2003, MNRAS, 338, L25 

\bibitem[()]{} Oguri, M., \& Marshall, P.J. 2010, MNRAS, 405, 2579

\bibitem[()]{} Oguri, M. \& Takada, M. 2011, Physical Review D, 83, 023008

\bibitem[()]{} Padmanabhan, N., Schlegel, D., Finkbeiner, D., et al.~2008, Astrophysical Journal, 674, 1217

\bibitem[()]{} Parker, A., Ivezi\'{c}, \v{Z}, Juri\'{c}, M., et al.~2008, Icarus, 198, 138

\bibitem[()]{} Patat, F. 2008, A\&A, 481, 575

\bibitem[()]{} Percival, W.J., Reid, B.A., Eisenstein, D.J., et al.~2010, MNRAS, 401, 2148 

\bibitem[()]{} Perlmutter, S., Aldering, G., Goldhaber, G., et al.~1999, Astrophysical Journal, 517, 565

\bibitem[()]{} Perryman, M.A.C., de Boer, K.S., Gilmore, G., et al.~2001; Astronomy \& 
             Astrophysics, 369, 339

\bibitem[()]{} Peterson, J.R., Jernigan, J.G., Kahn, S.M., et al.~2015, Astrophysical Journal Supplement, 218, 1

\bibitem[()]{} Pettersen, B.R. \& Hawley, S.L. 1989, Astronomy \& Astrophysics, 217, 187

\bibitem[()]{} Pier, J.R., Munn, J.A., Hindsley, R.B., et al.~2003, Astronomical Journal, 125, 1559

\bibitem[()]{} Pinto, P.A., Smith, R.C., Garnavich, P.M., et al.~2005, 
              American Astronomical Society, AAS Meeting \#205, \#108.20

\bibitem[()]{} Planck Collaboration 2011,  Astronomy \& Astrophysics, 536, 23 

\bibitem[()]{} Porciani, C. \& Madau, P. 2000, Astrophysical Journal, 532, 679

\bibitem[()]{} Pravec, P. \& Harris, A.W. 2000, Icarus, 148, 12

\bibitem[()]{} Rabinowitz, D.L. 1993,  Astrophysical Journal, 407, 412

\bibitem[()]{} Reid, I.~N., Gizis, J.~E. \& Hawley, S.~L. 2002,  Astronomical Journal, 124, 2721 

\bibitem[()]{} Reyes, R., Mandelbaum, R., Seljak, U., et al.~2010, Nature, 464, 256

\bibitem[()]{} Richards, G.T., Fan, X., Newberg, H.J., et al.~2002, Astronomical Journal, 123, 2945

\bibitem[()]{} Riess, A.G., Filippenko, A.V., Challis, P., et al.~1998, Astronomical Journal, 116, 1009

\bibitem[()]{} Riess, A.G., Strolger, L-G., Casertano, S., et al.~2007, Astrophysical Journal, 659, 98

\bibitem[()]{} Ross, N.P., Myers, A.D., Sheldon, E.S. et al.~2012,
  Astrophysical Journal (Supplements), 199, 3

\bibitem[()]{} Rossetto, B.~M., Santiago, B.~X., Girardi, L., et al.~2011, Astronomical Journal, 141, 185

\bibitem[()]{} Rowe, B., Jarvis, M., Mandelbaum, R., et al.~2015, Astronomy and Computing,  10

\bibitem[()]{} Rowe, J.F., Bryson, S.T., Marcy, G.W., et al.~2014, Astrophysical Journal, 784, 45

\bibitem[()]{} Schlafly, E.F., Finkbeiner, D.P., Juri\'c, M. et
  al.~2012, Astrophysical Journal, 756, 158

\bibitem[()]{} Schlegel, D.J., Finkbeiner, D.P. \& Davis, M. 1998, Astrophysical Journal, 500, 525

\bibitem[()]{} Scranton, R., Connolly, A., Krughoff, S., et al.~2007, astro-ph/0709.0752

\bibitem[()]{} Schwamb, M.~E., Brown, M.~E., Rabinowitz, D.~L. \& Ragozzine, D. 2010, Astrophysical Journal, 720, 1691

\bibitem[()]{} Seljak, U., Makarov, A., McDonald, P., et al.~2005,  Physical Review D, 71, 3515

\bibitem[()]{} Seo, H-J. \& Eisenstein, D.J. 2003, Astrophysical Journal, 598, 720

\bibitem[()]{} Sesar, B., Ivezi\'{c}, \v{Z}, Lupton, R.H., et al.~2007, Astronomical Journal, 134, 2236

\bibitem[()]{} Sesar, B., Ivezi\'{c}, \v{Z}, Grammer, S.H., et al.~2010, Astrophysical Journal, 708, 717

\bibitem[()]{} Shara, M.M. 2006, Astronomical Journal, 131, 2980

\bibitem[()]{} Sharma, S., Johnston, K.~V., Majewski, S.~R., Bullock, J. \& Mu{\~n}oz, R.~R. 2011,  
                      Astrophysical Journal, 728, 106

\bibitem[()]{} Shen, K.J. \& Bildsten, L. 2009, Astrophysical Journal, 692, 324

\bibitem[()]{} Shields, G.A. \& Bonning, E.W. 2008, Astrophysical Journal, 682, 758 

\bibitem[()]{} Skrutskie, M.F., Cutri, R.M., Stiening, R., et al.~2006, Astronomical Journal, 131, 1163.

\bibitem[()]{} Smartt, S.J. 2009, ARA\&A, 47, 63

\bibitem[()]{} Smith, N., Li, W., Silverman, J.M., Ganeshalingam, M. \& Filippenko, A.V. 2011, MNRAS, 415, 773

\bibitem[()]{} Song, Y-S. \& Knox, L. 2004, Physical Review D, 70, 63510

\bibitem[()]{} Spergel, D, Gehrels, N., Breckinridge, J. et al.~2013, arXiv:1305.5422

%\bibitem[()]{} Spergel, D.N., Verde, L., Peiris, H.V., et al.~2007, Astrophysical Journal Supp., 148, 175

\bibitem[()]{} Staniszewski, Z., Ade, P.A.R., Aird, K.A., et al.~2009,  Astrophysical  Journal, 701, 32

\bibitem[()]{} Stassun, K., Paegert, M., De Lee, N.~M. \& Cargile, P. 2013, American Astronomical Society, 
                      AAS Meeting \#221, \#116.01

\bibitem[()]{} Strubbe, L.E. \& Quataert, E. 2009, MNRAS, 400, 2070

\bibitem[()]{} Stubbs, C.W. \& Tonry, J., 2006, Astrophysical Journal, 646, 1436

\bibitem[()]{} Stubbs, C.W., High, F.W., George, M.R., et al.~2007, PASP, 119, 1163 

\bibitem[()]{} Suyu, S.H., Marshall, P.J., Auger, M.W., et al.~2010, Astrophysical Journal, 711, 201

\bibitem[()]{} Szab\'o, Gy.M., Ivezi\'c, \v{Z}., Juri\'c, M., \& Lupton, R. 2007, MNRAS, 377, 1393

\bibitem[()]{} Takada, M., Ellis, R.S., Chiba, M. et al.~2014, PASJ,
  66, 1

\bibitem[()]{} Takada, M. \& Jain, B. 2004, MNRAS, 348, 897

\bibitem[()]{} Thompson, T.A.,Prieto, J.L., Stanek, K.Z., et al.~2009, Astrophysical Journal, 705, 1364

\bibitem[()]{} Thornton, D., Stappers, B., Bailes, M., et al.~2013, Science, 341, 53

\bibitem[()]{} Tisserand, P., Le Guillou, L., Afonso, C. et al.~2007, Astronomy \& Astrophysics, 469, 387

\bibitem[()]{} Tokovinin, A. 2002, PASP, 114, 1156

\bibitem[()]{} Treu, T., Koopmans, L.V., Bolton, A.S., Burles, S., and
  Moustakas, L.A. 2006, Astrophysical Journal, 640, 662

\bibitem[()]{} Trujillo, C.A., Jewitt, D.C. \& Luu, J.X. 2001, Astronomical Journal, 122, 457

\bibitem[()]{} Trujillo, C.A. \& Sheppard, S.~S. 2014, Nature, 507, 471

\bibitem[()]{} Tyson, J.A., Kochanski, G.P. \& Dell'Antonio, I.P. 1998, Astrophysical Journal, 498, L107

\bibitem[()]{} Tyson, J.A., Roat, C., Bosch, J. \& Wittman, D. 2008a, Astronomical 
             Data Analysis Software and Systems XVII, ASP Conference Series, 
             J. Lewis, R. Argyle, P. Bunclarck, D. Evans, and E. Gonzales-Solares, eds.     

\bibitem[()]{} Tyson, J. A., and LSST collaboration, 2008b, see 
                 http://universe.ucdavis.edu/docs/LSST\_petascale\_challenge.pdf 

\bibitem[()]{} Vanden Berk, D.E., Richards, G.T., Bauer, A., et al.~2001, Astronomical Journal, 122, 549

\bibitem[()]{} Vanden Berk, D.E., Wilhite, B.C., Kron, R.G., et al.~2004, Astrophysical Journal, 601, 692

\bibitem[()]{} Walker, M.~G. \& Pe{\~n}arrubia, J. 2011, Astrophysical Journal, 742, 20

%\bibitem[()]{} Walsh, S.M., Jerjen, H. \& Willman, B. 2007, Astrophysical Journal, 662, L83

\bibitem[()]{} Walsh, S.M., Willman, B. \& Jerjen, H. 2009, Astronomical Journal, 137, 450

\bibitem[()]{} Wang, L., Goldhaber, G., Aldering, G. \& Perlmutter, S. 2003, Astrophysical Journal, 590, 944

\bibitem[()]{} Wang, S., Haiman, Z., May, M. \& Kehayias, J. 2005, astro-ph/0512513

\bibitem[()]{} Weinberg, D.~H., Bullock, J.~S., Governato, F., Kuzio de Naray, R. \& Peter, A.~H.~G., 2013, arXiv:1306.0913

\bibitem[()]{} Weinberg, D.H., Mortonson, M.J., Eisenstein, D.J.,
  Hirata, C., Riess, A.G., \& Rozo, E. 2013, Physics Reports, 530, 87

\bibitem[()]{} West, A.~A., Bochanski, J.~J., Bowler, B.~P., et al.~2011, Astronomical Society of the Pacific 
                  Conference Series, 448, 531

\bibitem[()]{} Wood-Vasey, W.M., Miknaitis, G., Stubbs, C.W., et al.~2007, Astrophysical Journal, 666, 694 

\bibitem[()]{} Wright, J.T., Marcy, G.W., Howard, A.W., et al.~2012, Astrophysical Journal, 753, 160

\bibitem[()]{} Xue, Y.Q., Brandt, W.N., Luo, B., et al, 2010,  Astrophysical Journal, 720, 368 

\bibitem[()]{} Yoachim, P., Coughlin, M.,Angeli, G., et al, 2014, SPIE, 9910, 99101A

\bibitem[()]{} Yoo, J., Chanam\'{e}, J. \& Gould, A. 2004, Astrophysical Journal, 601, 311

\bibitem[()]{} York, D.G., Adelman, J., Anderson, S., et al.~2000, Astronomical Journal, 120, 1579 

\bibitem[()]{} Yoshida, F. \& Nakamura, T. 2005, Astronomical Journal, 130, 2900 

\bibitem[()]{} Zaritsky, D., Harris, J., Thompson, I. B., \& Grebel, E. K., 2004, Astronomical Journal, 128, 1606

\bibitem[()]{} Zhan, H. 2006, Journal of Cosmology \& Astroparticle Physics, 8, 8

\bibitem[()]{} Zhan, H. \& Knox, L. 2006,  Astrophysical Journal, 644, 663

\bibitem[()]{} Zhan, H. \& Knox, L. \& Tyson, J.A. 2009,  Astrophysical Journal, 690, 923
	
\bibitem[()]{} Zhang, B. \& M\'{e}sz\'{a}ros, P. 2004, International Journal of Modern Physics A, 19, 2385

\bibitem[()]{} Zhang, B., Fan, Y.Z., Dyks, J., et al.~2006,  Astrophysical Journal, 642, 354

\end{thebibliography}


\end{document}




% latex lsst; dvips -Ppdf -o lsst.ps lsst.dvi; ps2pdf14 lsst.ps lsst.pdf
