\section{  EDUCATIONAL AND SOCIETAL IMPACTS    }
\label{Sec:impact}

\B{
The impact -- and enduring societal significance -- of the LSST will exceed its direct contributions 
to advances in physics and astronomy.  LSST is uniquely positioned to have high impact with the interested 
public and K-12 educational programs. Engaging the public in LSST activities has, therefore, been part of 
the project design from the beginning. 
With its open data policy, survey mode of operations and 
data products that offer vast potential for discovery, LSST facilitates the active engagement of a 
broad community with pathways to lifelong learning.  LSST will contribute to the national goals of 
enhancing science literacy and increasing the global competitiveness of the US science and technology 
workforce.  We intend to develop in our audience the awareness that they are part of a dynamic universe 
that changes on many time scales.  

We will use three primary techniques to accomplish these goals: visualization, utilization of real data, 
and active engagement in the research process. These techniques will be executed via three primary components: 
a dynamic, immersive public web presence featuring LSST discoveries; a physical presence in classrooms 
and science centers; and an emphasis on citizen science, that is, participation in the research process by 
non-specialists.  LSST's ``movie of the Universe'' will offer a new window on the sky for curious minds of 
all ages.  It is not unreasonable to anticipate tens of millions of public users browsing the LSST 
sky,
%\footnote{At the angular resolution of LSST, it would take over 3
%million  HDTV sets (with 1080$\times$1920  pixel resolution) to
%display a single-epoch  LSST image of half the Celestial Sphere (and
%about 1000 such  images will be  obtained over 10 years).}, 
%THIS IS REPEATED IN THE MAIN TEXT NEAR THE END OF THE PAPER.
exploring discoveries as they are broadcast and monitoring objects 
of interest.  For users just browsing, interfaces such as Sky in Google Earth or Microsoft's World Wide Telescope 
will serve images.  For those wanting a deeper experience, LSST will provide a value-added interface for 
web-based immersive delivery of dynamic multi-media content related to LSST discoveries.

LSST data can become a key part of projects emphasizing student-centered research in middle school, high 
school, and undergraduate settings. Web-based instructional materials, on-line professional development, and 
software tools will facilitate both formal and lifelong learning opportunities.  Citizen science is an integral 
component of the overall LSST EPO program, allowing us to actively engage a large and broad community in 
the excitement of exploration.  LSST EPO expects to operate at least one ``Citizen Science'' project at all times and provide 
properly constructed data products, tools, and interfaces to facilitate efforts outside LSST.   A few dozen 
``power users'' at major science centers, which in turn will reach 10-20 million participants annually, will incorporate 
LSST skymaps, discoveries,  and scientists into displays and live sky shows.  We will provide a specialized data 
access interface to expedite access at digital planetariums; kiosks bridging museum experiences to classroom 
and lifelong learning will be developed. This 
involvement and active participation will allow LSST to fulfill its public responsibility and extend its scientific 
potential –- a truly transformative idea for the 21$^{st}$ century telescope system.
}

